\begin{tcolorbox}[breakable,colback=orange!5!white, colframe=orange!80!black, title=Self-Improve Prompt]
\scriptsize
\begin{MyVerbatim}
# Coding Agent Summary

- **Main File**: `coding_agent.py`
  - Primary Class: `AgenticSystem`
  - The `forward()` function is the central entry point.
  - Prompts are located either within the `forward()` function or in the `prompts/` directory.
- **Tools**: `tools/`
  - The `tools/` directory contains various tools that LLMs can use to perform specific tasks.
  - Each tool must have a `tool_info()` function that returns a JSON object containing 'name', 'description', and 'input_schema'. The 'input_schema' should be a JSON object containing 'type', 'properties', and 'required'.
  - Each tool must have a `tool_function()` function that takes the arguments defined in input_schema, performs the tool's task, and returns a string.
  - See other tools for reference.
- **Utilities**: `utils/`
  - The `utils/` directory contains utility functions used across the codebase.

- **Additional Details**:
  - The agent is very good at automatically utilizing the right available tools at the right time. So do not have an agentic flow that explicitly forces a tool's usage.
  - Common tools, such as file editing and bash commands, are easy for the agent to recognize and use appropriately. However, more complex and niche tools may require explicit instructions in the prompt.
  - Tools should be designed to be as general as possible, ensuring they work across any GitHub repository. Avoid hardcoding repository-specific details or behaviors (e.g., paths).
  - Do not use 'while True' loops in the agent's code. This can cause the agent to get stuck and not respond.
  - Verify the implementation details of helper functions prior to usage to ensure proper integration and expected behavior.
  - Do not install additional packages or dependencies directly. Update `requirements.txt` if new dependencies are required and install them using `pip install -r requirements.txt`.

Here is the implementation of the coding agent.

# Coding Agent Implementation
----- Coding Agent Implementation Start -----
{code}
----- Coding Agent Implementation End -----

Your task is to identify ONE detailed plan that would improve the agent's coding ability. The improvement should not be specific to any particular GitHub issue or repository.

Here is the log for the coding agent trying to solve the GitHub issues but failed.

# Agent Running Log
----- Agent Running Log Start -----
{md_log}
----- Agent Running Log End -----

# GitHub Issue
The GitHub issue that the agent is trying to solve.
----- GitHub Issue Start -----
{github_issue}
----- GitHub Issue End -----

# Predicted Patch
The agent's predicted patch to solve the issue.
----- Predicted Patch Start -----
{predicted_patch}
----- Predicted Patch End -----

# Private Test Patch
SWE-bench's official private tests to detect whether the issue is solved. This is not available to the agent during evaluation. The agent should try to implement its own tests.
----- Private Test Patch Start -----
{test_patch}
----- Private Test Patch End -----

# Issue Test Results
The test results from SWE-bench using the above official private tests.
----- Issue Test Results Start -----
{eval_log}
----- Issue Test Results End -----

Respond precisely in the following format including the JSON start and end markers:

```json
<JSON>
```

In <JSON>, provide a JSON response with the following fields:
- "log_summarization": Analyze the above logs and summarize how the agent tried to solve the GitHub issue. Note which tools and how they are used, the agent's problem-solving approach, and any issues encountered.
- "potential_improvements": Identify potential improvements to the coding agent that could enhance its coding capabilities. Focus on the agent's general coding abilities (e.g., better or new tools usable across any repository) rather than issue-specific fixes (e.g., tools only usable in one framework). All necessary dependencies and environment setup have already been handled, so do not focus on these aspects.
- "improvement_proposal": Choose ONE high-impact improvement from the identified potential improvements and describe it in detail. This should be a focused and comprehensive plan to enhance the agent's overall coding ability.
- "implementation_suggestion": Referring to the coding agent's summary and implementation, think critically about what feature or tool could be added or improved to best implement the proposed improvement. If the proposed feature can be implemented by modifying the existing tools, describe the modifications needed, instead of suggesting a new tool.
- "problem_description": Phrase the improvement proposal and implementation suggestion as a GitHub issue description. It should clearly describe the feature so that a software engineer viewing the issue and the repository can implement it.

Your response will be automatically parsed, so ensure that the string response is precisely in the correct format. Do NOT include the `<JSON>` tag in your output.
\end{MyVerbatim}
\end{tcolorbox}
