\documentclass{article}


% if you need to pass options to natbib, use, e.g.:
\PassOptionsToPackage{sort&compress}{natbib}
% before loading neurips_2025
\PassOptionsToPackage{numbers}{natbib}


% ready for submission
% \usepackage{neurips_2025}


% to compile a preprint version, e.g., for submission to arXiv, add add the
% [preprint] option:
    \usepackage[preprint]{neurips_2025}


% to compile a camera-ready version, add the [final] option, e.g.:
%     \usepackage[final]{neurips_2025}


% to avoid loading the natbib package, add option nonatbib:
%    \usepackage[nonatbib]{neurips_2025}


\usepackage[utf8]{inputenc} % allow utf-8 input
\usepackage[T1]{fontenc}    % use 8-bit T1 fonts
\usepackage{hyperref}       % hyperlinks
\usepackage{url}            % simple URL typesetting
\usepackage{booktabs}       % professional-quality tables
\usepackage{amsfonts}       % blackboard math symbols
\usepackage{nicefrac}       % compact symbols for 1/2, etc.
\usepackage{microtype}      % microtypography
\usepackage{xcolor}         % colors
\usepackage[normalem]{ulem}
\usepackage{titletoc}
\usepackage{graphicx}
\usepackage{amsmath}
\usepackage{cleveref}
\usepackage{multicol}
\usepackage{tcolorbox}
\usepackage{float}
\usepackage[ruled,vlined]{algorithm2e}
\tcbuselibrary{breakable}
\tcbuselibrary{listings}
\tcbuselibrary{listingsutf8}
\usepackage{subcaption}
\usepackage{fvextra}
\DefineVerbatimEnvironment{MyVerbatim}{Verbatim}{breaklines=true}
% nicer diff display
\lstdefinelanguage{Diff}{
  sensitive=false,
  morekeywords={diff,index,---,+++},               % keywords
  morecomment=[f][\color{red!80!black}]{-},         % lines beginning with '-'
  morecomment=[f][\color{green!60!black}]{+},       % lines beginning with '+'
}
\lstdefinestyle{diffstyle}{
  language=Diff,
  basicstyle=\ttfamily\scriptsize,
  columns=fullflexible,
  frame=single,
  rulecolor=\color{gray!50},
  backgroundcolor=\color{black!2},
  breaklines=true,
  breakatwhitespace=true,
  postbreak=\mbox{\textcolor{gray}{$\hookrightarrow$}\space},
  xleftmargin=1em,
  xrightmargin=1em,
  aboveskip=0.5em,                        % less vertical padding
  belowskip=0.5em,
  commentstyle=\itshape,                  % context lines
  keywordstyle=\bfseries\color{blue!70!black}, % diff/index/@@...
}
\lstset{style=diffstyle}

% \newif\ifcomment
% % \commentfalse
% \newcommand{\jc}[1]{\textcolor{blue}{\ifcomment JC: #1\else\fi}}
% \newcommand{\jcsout}[1]{\textcolor{blue}{\ifcomment JC: \sout{#1}\else\fi}}
% \newcommand{\jcc}[1]{\textcolor{blue}{\ifcomment(JC Comment: #1)\else#1\fi}}
% \newcommand{\jz}[1]{\textcolor{olive}{(\ifcomment JZ: #1\else\fi)}}
% \newcommand{\jzsout}[1]{\textcolor{olive}{\ifcomment JZ: \sout{#1}\else\fi}}
% \newcommand{\cl}[1]{\textcolor{orange}{(\ifcomment CL: #1\else\fi)}}
% \newcommand{\clout}[1]{\textcolor{orange}{\ifcomment CL: \sout{#1}\else\fi}}
% \newcommand{\sh}[1]{\textcolor{violet}{\ifcomment SH: #1\else\fi}}
% \newcommand{\shout}[1]{\textcolor{violet}{\ifcomment SH: \sout{#1}\else\fi}}
% \newcommand{\rl}[1]{\textcolor{pink}{(\ifcomment RL: #1\else\fi)}}
% \newcommand{\rlout}[1]{\textcolor{pink}{\ifcomment RL: \sout{#1}\else\fi}}
% \newcommand{\todo}[1]{\textcolor{red}{\ifcomment(\textbf{TODO}: #1)\else\fi}}

\title{Darwin G\"odel Machine:\\Open-Ended Evolution of Self-Improving Agents}


% The \author macro works with any number of authors. There are two commands
% used to separate the names and addresses of multiple authors: \And and \AND.
%
% Using \And between authors leaves it to LaTeX to determine where to break the
% lines. Using \AND forces a line break at that point. So, if LaTeX puts 3 of 4
% authors names on the first line, and the last on the second line, try using
% \AND instead of \And before the third author name.


\author{%
Jenny Zhang\textsuperscript{*,1,2} \quad
Shengran Hu\textsuperscript{*,1,2,3} \quad
Cong Lu\textsuperscript{1,2,3} \quad
Robert Lange\textsuperscript{\dag,3} \quad
Jeff Clune\textsuperscript{\dag,1,2,4}
\\[1ex]
\textsuperscript{1}University of British Columbia \quad
\textsuperscript{2}Vector Institute \quad
\textsuperscript{3}Sakana AI \quad
\textsuperscript{4}Canada CIFAR AI Chair \\
\texttt{\{jennyzzt,srhu,conglu\}@cs.ubc.ca}, \texttt{robert@sakana.ai}, \texttt{jeff.clune@ubc.ca}
}


\begin{document}

\renewcommand{\thefootnote}{\fnsymbol{footnote}}
\footnotetext[1]{co-authors}
\footnotetext[2]{co-senior authors}
\renewcommand{\thefootnote}{\arabic{footnote}}

% \commenttrue % Set \commentfalse to togle inline coments off
% \commentfalse

\maketitle

\vspace{-5pt}
\begin{abstract}
\vspace{-5pt}
Most of today’s AI systems are constrained by human-designed, fixed architectures and cannot autonomously and continuously improve themselves. The scientific method, on the other hand, provides a cumulative and open-ended system, where each innovation builds upon previous artifacts, enabling future discoveries. There is growing hope that the current manual process of advancing AI could itself be automated. If done safely, such automation would accelerate AI development and allow us to reap its benefits much sooner. This prospect raises the question of how AI systems can endlessly improve themselves while getting better at solving relevant problems. Previous approaches, such as meta-learning, provide a toolset for automating the discovery of novel algorithms but are limited by the human design of a suitable search space and first-order improvements. The G\"odel machine~\citep{schmidhuber2007godel}, on the other hand, introduced a theoretical approach to a self-improving AI, capable of modifying itself in a provably beneficial manner. Unfortunately, this original formulation is in practice impossible to create due to the inability to prove the impact of most self-modifications. To address this limitation, we propose the Darwin G\"odel Machine (DGM), a novel self-improving system that iteratively modifies its own code (thereby also improving its ability to modify its own codebase) and empirically validates each change using coding benchmarks. In this paper, the DGM aims to optimize the design of coding agents, powered by frozen foundation models, which enable the ability to read, write, and execute code via tool use. Inspired by biological evolution and open-endedness research, the DGM maintains an archive of generated coding agents. It then samples from this archive and tries to create a new, interesting, improved version of the sampled agent. This open-ended exploration forms a growing tree of diverse, high-quality agents and allows the parallel exploration of many different paths through the search space. Empirically, the DGM automatically improves its coding capabilities (e.g., better code editing tools, long-context window management, peer-review mechanisms), producing performance increases on SWE-bench from 20.0\% to 50.0\%, and on Polyglot from 14.2\% to 30.7\%. Furthermore, the DGM significantly outperforms baselines without self-improvement or open-ended exploration. All experiments were done with safety precautions (e.g., sandboxing, human oversight). Overall, the DGM represents a significant step toward self-improving AI, capable of gathering its own stepping stones along a path that unfolds into endless innovation. All code is open-sourced at \href{https://github.com/jennyzzt/dgm}{https://github.com/jennyzzt/dgm}.
\end{abstract}

\section{Introduction}
\label{sec:intro}

\begin{figure}[t!]
\centering
\includegraphics[width=\textwidth]{figures/conceptual.pdf}
\caption{\textbf{Darwin G\"odel Machine.} The DGM iteratively builds a growing archive of agents by interleaving self-modification with downstream task evaluation. Agents in the archive are selected for self-modification through open-ended exploration.}
\vspace{-0.3cm}
\label{fig:conceptual}
\end{figure}

Scientific progress is cumulative and open-ended, with each breakthrough standing on the shoulders of countless prior insights. In the same way, our most advanced AI systems are built upon a long lineage of innovations. For instance, transformers~\citep{vaswani2017attention}, the backbone of current large language models (LLMs)~\citep{brown2020language}, did not emerge in isolation but were built upon years of past innovations, such as recurrent neural networks~\citep{rumelhart1985learning, hopfield1982neural} and attention mechanisms~\citep{bahdanau2015neural, kim2017structured, parikh2016decomposable}. However, most of today’s AI systems remain bound by fixed, human-designed architectures that learn within predefined boundaries, without the capacity to autonomously rewrite their own source code to self-improve. As a result, each advancement in AI development still leans heavily on human interventions, tethering the pace of progress. This paper investigates the intriguing possibility of safely automating the search for ever-better AI. One can imagine an AI system that, like scientific discovery itself, becomes an engine of its own advancement: building upon its past, recursively improving, and propelling itself toward more advanced capabilities.

\citet{schmidhuber2007godel} presented a class of mathematically rigorous, self-referential, self-improving problem solvers. It relies on formal proofs to justify code rewrites, ensuring that any self-modification is provably beneficial. However, in practice and without restrictive assumptions about the system, it is impossible to formally prove whether a modification to an AI system will be beneficial. For example, while it may seem that an LLM-based coding agent would benefit from access to more tools (e.g., code search, test runners), the actual impact depends heavily on the model’s training and task context (e.g., a testing tool that is optimized for one setup may confuse the agent when working with others). Instead of requiring formal proofs, we empirically validate self-modifications against a benchmark, allowing the system to improve and explore based on observed results. This approach mirrors biological evolution, where mutations and adaptations are not verified in advance but are produced, trialed, and then selected via natural selection. We also take inspiration from Darwinian evolution~\citep{darwin2023origin} and investigate the effectiveness of maintaining a library of previously discovered agents to serve as stepping stones for future generations.

We propose the \textbf{Darwin G\"odel Machine (DGM)}, a self-referential, self-improving system that writes and modifies its own code to become a better coding agent. Each self-modification requires the DGM to edit its own codebase. We use Python, which is Turing-complete, giving the DGM the potential to build any computable machine. Our framework envisions agents that can rewrite their own training scripts (including training a new foundation model (FM)). However, we do not show that in this paper, as training FMs is computationally intensive and would introduce substantial additional complexity, which we leave as future work. Instead, this paper focuses on improving the design of coding agents with frozen pretrained FMs (e.g., tool use, workflows). The DGM alternates between self-modification and evaluation phases. During the self-modification phase, selected coding agents from the archive generate modified versions of themselves. During the evaluation phase, each modified agent is tested on a coding benchmark, estimating the agent's coding capabilities, and then added to the archive. By improving its own capabilities through this loop, the DGM becomes better at both solving coding tasks and making future self-improvements. A key assumption is that an increase in performance on coding benchmarks indicates better coding capabilities, and hence better ability to self-modify and self-improve. Furthermore, the DGM maintains an archive of generated coding agents, initialized with only one agent, and continuously accumulates all generated variants over time. To support continual self-improvement, the DGM draws inspiration from open-endedness research~\citep{wang2019paired, fernandopromptbreeder, faldor2025omni}, accumulating diverse stepping stones (i.e., interesting yet suboptimal solutions or features that may enable future breakthroughs). This open-ended exploration encourages the discovery of novel and potentially useful self-modifications beyond immediate performance gains.

We present results on two coding benchmarks: SWE-bench~\citep{jimenez2024swebench} and Polyglot~\citep{gauthier2024polyglot}. The DGM automatically improves itself from 20.0\% to 50.0\% on SWE-bench, and from 14.2\% to 30.7\% on Polyglot. We show that self-improvement enables continued progress, as the DGM outperforms the baseline where the same initial agent is repeatedly used to modify and generate new agents without self-improvement. We also show that open-ended exploration and keeping an archive of all previously generated agents lead to the discovery of better coding agents. The DGM outperforms the baseline of not having open-ended exploration (i.e., a baseline without the accumulation of an archive of interestingly different stepping stones), where the coding agent always builds off the most recent version of itself. Overall, the DGM represents a step toward AI systems that can build upon their own prior innovations and improve recursively. We consider and discuss safety aspects extensively, including sandboxing and traceability of self-modifications, to ensure responsible experimentation (\Cref{sec:safety}). By advancing the possibility of safe, self-referential, self-improving models, the DGM moves us closer to AI that not only learns but evolves in an open-ended, self-accelerating trajectory, much like science itself.

\section{Related Work}
\label{sec:related}

\textbf{Open-Endedness.}
A grand challenge for driving unbounded innovation is designing open-ended AI systems that continuously generate novel and learnable artifacts~\citep{stanley2017open}. Building on this, \citet{hughes2024open} characterized open-endedness as a system's capacity to generate sequences of artifacts that are both novel and learnable from an observer's perspective. A central difficulty lies in structuring and exploring vast search spaces to consistently produce artifacts that are interesting to humans~\citep{clune2019ai, jiang2023general}. Early efforts addressed this through quality-diversity algorithms~\citep{pugh2016quality, chatzilygeroudis2021quality, mouret2015illuminating, nguyen2015innovation}, goal-directed exploration methods~\citep{ecoffet2019go, ecoffet2021first, schaul2015universal, andrychowicz2017hindsight, eysenbach2018diversity}, intrinsic motivation~\citep{lehman2011novelty, oudeyer2007intrinsic, li2014encouraging, pathak2017curiosity}, or learning progress frameworks~\citep{kanitscheider2021multi, gaven2025magellan, baranes2013active, colas2019curious, colas2022autotelic, jiang2021prioritized, dennis2020emergent, schmidhuber2008driven, schmidhuber2013powerplay}. More recently, large-scale foundation models (FMs)~\citep{brown2020language, radford2019language} have emerged as powerful proxies for human notions of interestingness~\citep{zhang2024omni,faldor2025omni,sancaktar2025sensei} and effective mutation operators to propose novel solutions in code~\citep{romera2024mathematical,Novikov2025AlphaEvolve,lehman2023evolution,faldor2025omni,hu2025automated}. FMs can guide autotelic agents~\citep{colas2022autotelic, colas2023augmenting, colas2022language}, model human preferences for quality and diversity~\citep{bradley2023quality, ding2023quality, wang2023diversity, klissarov2023motif, klissarov2024maestromotif, samvelyan2024rainbow, lim2024large, havrilla2024surveying}, design reward functions~\citep{wang2023voyager, ma2023eureka, faldor2025omni}, create simulated environments~\citep{sudhakaran2023mariogpt, nasir2023practical, aki2024llm, nasir2024word2world, bruce2024genie, parkerholder2024genie2}, drive ever-evolving multi-agent dynamics~\citep{dharna2024quality, zhou2025autoredteamer}, search diverse ambulating robot morphologies~\citep{lehman2023evolution}, and search expansive solution spaces for benchmark or objective optimization~\citep{lange2024large, zhang2024omni, faldor2025omni, hu2025automated, lu2024intelligent, romera2024mathematical, fernandopromptbreeder, lu2024ai, khan2024debating, lu2025automated, liu2024evolution, Novikov2025AlphaEvolve}. However, these approaches have yet to close the self-improvement loop, meaning improvements on downstream tasks do not translate into enhanced capabilities for self-modification or the acceleration of further innovations. We aim to mimic the acceleration of science and technology, where new tools and discoveries catalyze the creation of even more discoveries. Similarly, how can we emulate nature's arc of evolution, which bends not only toward complexity but also an ever greater capacity to evolve~\citep{dawkins2019evolution, gerhart2007theory, hendrikse2007evolvability}?

\textbf{Meta-Learning FM Agents.}
Many FM-based agents are handcrafted. Some building blocks include prompt engineering~\citep{chen2023unleashing, schulhoff2024prompt}, chain-of-thought~\citep{wei2022chain, yao2023react, hu2023ThoughtCloning, guo2025deepseek, lightman2023let, muennighoff2025s1, zelikman2024quiet}, self-reflection \citep{shinn2023reflexion, yao2023react, madaan2303self}, multi-agent debate \citep{liang2023encouraging, khan2024debating}, memory~\citep{liu2023think, zhong2024memorybank, modarressi2023ret}, temperature sampling \citep{zhu2024hot}, and retrieval augmented generation~\citep{lewis2020retrieval}. The manual composition of these components limits the system's abilities to the ingenuity of its human designer. More recently, several meta-learning approaches have emerged that leverage FM to automatically optimize prompts~\citep{fernandopromptbreeder, meta2022human, khattab2023dspy,cheng2025trace,yuksekgonul2024textgrad,yuan2024evoagent} and design agentic modules~\citep{zhang2024aflow, zhou2024symbolic, yin2024g, zhugegptswarm, rosser2025agentbreeder, zhang2025multi, ye2025mas, gao2025flowreasoner, nie2025weak, su2025debflow, zhang2025gnns,niu2025flow}. The Automated Design of Agentic Systems \citep[ADAS,][]{hu2025automated} iteratively generates downstream agents with a fixed meta-agent, evaluates them against a target benchmark, and incorporates feedback to refine subsequent generations. In contrast, the DGM is a single system that both solves downstream tasks (i.e., coding problems) and refines its own implementation (i.e., its codebase), removing the need for a fixed, handcrafted meta-agent and enabling self-referential improvements.

\textbf{Self-Improving AI.}
Early on, various researchers outlined theoretical and conceptual approaches to self-improvement \citep{good1966speculations, schmidhuber1987evolutionary, schmidhuber2007godel}. Some practical approaches to automated self-improvement include systems defined by neural network weight parameterizations~\citep{hall2007self, hobbhahn2025swebenchmini, kirsch2022self, lu2023arbitrary, havrilla2024glore}. \citet{metz2021training} developed a gradient-based optimizer that is self-referentially meta-trained using a variant of population-based training~\citep{jaderberg2017population}. \citet{lange2023discovering} extended this approach to gradient-free learning. \citet{silver2017mastering} used self-play to continuously evolve agents, achieving superhuman performance in challenging domains such as chess and Go. More closely related to the DGM are recent approaches that leverage FM-based agents for self-improvement~\citep{yin2024g, robeyns2025self, hu2024self, zelikman2024self, huang2022large, singh2023beyond}. \citet{zelikman2024self} use a meta-agent to generate downstream agents, updating the meta-agent based on the meta-utility derived from the generated solutions. \citet{yin2024g} use a single system to both solve downstream tasks and recursively modify itself. However, the downstream tasks or the meta-utility do not always align with the capabilities required for self-improvement. In the DGM, improvement in downstream tasks directly reflects an increase in self-improvement ability, enabling the potential for self-accelerating progress. Most similar is concurrent work by \citet{robeyns2025self}, which also has a single agent recursively solving coding problems and modifying its own codebase. The main difference between the DGM and \citet{robeyns2025self} is that the DGM has an open-ended exploration loop, encouraging self-modifications beyond immediate performance gains and thus avoiding stagnation in suboptimal self-modifications.

\vspace{-2mm}
\section{Darwin G\"odel Machine}
\label{sec:methods}
\vspace{-2mm}

A G\"odel Machine is a theoretical idea of an AI that searches for ways that \emph{provably} improve itself~\citep{schmidhuber2007godel}. In this paper, we propose Darwin G\"odel Machine (DGM), an attempt to realize the long-held dream of creating a G\"odel Machine. The DGM relaxes the G\"odel Machine's impractical requirement of theoretically \emph{proving} that a change will improve the system, instead requiring \emph{empirical evidence} from experiments to demonstrate that a proposed new version enhances performance. Additionally, since the DGM relies on empirical evidence of improvement, it may get stuck in a local optimum within the vast search space of possible systems (i.e., all computable algorithms). To address this, the DGM maintains an archive of discovered solutions during the search, facilitating open-ended exploration rather than relying on evolving a single solution. Since the principles echo Darwinian evolution~\citep{darwin2023origin}, where new innovations emerge by selecting an entity from an archive of previously discovered solutions, modifying it, and keeping it if it is interestingly new~\citep{zhang2024omni, faldor2025omni, stanley2015greatness}, we call our algorithm a Darwin G\"odel Machine.

\textbf{Self-referential Self-improvement of Coding Agents.}
The DGM is initialized with only one coding agent, and its progression is evaluated on coding benchmarks. A coding agent is defined as a single system, implemented with a code repository and powered by frozen pretrained foundation models (FMs), capable of reading, writing, and executing code. Recent works~\citep{hu2025automated, zhang2024aflow} demonstrate that such agents can be improved through meta-learning of their designs (e.g., prompts, workflows, and tools), which are implemented in their code repository. Therefore, we define self-improvement as a coding task that involves modifying the design of an agent's own components (i.e., its own code). The key motivation is that the empirical evidence must reflect the system's ability to both self-improve and solve downstream tasks. By configuring the DGM as a coding agent and testing its coding capabilities, the observed improvements demonstrate not only enhanced performance in downstream tasks but also the capacity for further self-improvement, as self-improvement is fundamentally a coding task that modifies the coding agent's own code repository.

\textbf{Population-based Open-ended Exploration.}
Starting from a single initial coding agent, the DGM builds an archive of all discovered agents. In each iteration, the DGM selects parent agents to self-modify and branch off to produce new agents. Parent selection is roughly proportional to each agent's performance score and the number of its children with codebase-editing functionality (\Cref{app:parent-select}). Each selected parent analyzes its own benchmark evaluation logs, proposes the next feature to implement, and receives this proposal as a problem statement to execute (\Cref{app:selfimprove-prompts}). The parent then implements the suggested feature into its own codebase, generating a new coding agent. Each newly generated agent is quantitatively evaluated on a chosen coding benchmark to estimate its coding abilities. Only agents that compile successfully and retain the ability to edit a given codebase are added to the DGM archive, as only they can continue self-modification. All others are discarded. The cycle of parent selection, self-modification, and evaluation continues, progressively growing the archive of solutions. Importantly, we note that archived solutions can serve as stepping stones that result in improvements much later than their original discovery, making our approach substantially different from hill-climbing agentic design approaches \citep{robeyns2025self}. Currently, the open-ended exploration process (i.e., archive maintenance, parent selection) is fixed and not modifiable by the DGM, which we leave as an avenue for future work. \Cref{app:pseudocode} shows the pseudocode for the DGM algorithm.

\section{Experiments}
\label{sec:experiments}

\Cref{sec:experiment-setup} describes the experimental setup, including the initial coding agent that initializes the DGM. We conduct experiments on two coding benchmarks: SWE-bench~\citep{jimenez2024swebench} and Polyglot~\citep{gauthier2024polyglot} (\Cref{sec:benchmarks}). For each benchmark, we compare the DGM against two baselines: DGM without self-improvement and DGM without open-ended exploration (\Cref{sec:baselines}). Across all experiments, we find that the DGM outperforms both baselines, showing that the self-improvement mechanism and open-ended exploration are essential for sustained performance gains (\Cref{sec:results}). Furthermore, we show that the features discovered by the DGM transfer across LLMs (\Cref{sec:results}).

\subsection{Experiment Setup}
\label{sec:experiment-setup}

The DGM is initialized with a single coding agent. This initial agent is built around an FM and is augmented with tool use capabilities~\citep{schick2023toolformer, anthropic_tool_use_2024}. The FM autonomously determines its action at each step, using whichever tool it deems appropriate. The FM has access to two tools in the initial agent: a Bash tool for executing bash commands, and an edit tool for viewing and editing directories and entire files (\Cref{app:initial-agent}). The agent receives a single input prompt that specifies the location of the target repository, the problem statement, and the repository's testing framework (\Cref{app:initial-agent}). We run the DGM for 80 iterations (generating one new agent per iteration), with two iterations running in parallel for SWE-bench and four for Polyglot (\Cref{app:parent-select}). During self-modifications, coding agents are powered by Claude 3.5 Sonnet (New)~\citep{anthropic2024claude35sonnet} both SWE-bench and Polyglot experiments. During benchmark evaluation, coding agents are powered by Claude 3.5 Sonnet (New) for SWE-bench and o3-mini~\citep{openai2025o3mini} for Polyglot.

\subsection{Benchmarks}
\label{sec:benchmarks}

Both SWE-bench and Polyglot are benchmarks that evaluate how well AI agents solve coding tasks automatically. Given a code repository and a task instruction, the agent is expected to make changes to the repository in order to fulfill the task. Both SWE-bench and Polyglot are widely used benchmarks~\citep{zhang2024autocoderover, zhang2024rest, xia2024agentless, cao2024spider2, GoogleDeepMind2025GeminiThinking,aider2024} that require the AI agent to navigate a code repository, understand the interplay between functions in different files, and spot small errors in convoluted code. SWE-bench only has Python tasks, while Polyglot has tasks in multiple programming languages. Another difference is that each SWE-bench task may require edits to multiple files, whereas each Polyglot task primarily involves implementing a solution from scratch in a single file (although the agent still needs to examine other files to understand what changes are necessary), resulting in fewer file edits overall. We evaluate the DGM on two popular benchmarks that assess different aspects of coding tasks to validate the algorithm's effectiveness across various use cases.

\textbf{SWE-bench.} To avoid wasting compute on unsolvable tasks, we use SWE-bench Verified~\citep{openai2024sweverified}, a subset of SWE-bench~\citep{jimenez2024swebench} filtered by human annotators to ensure that every task is solvable. Throughout this paper, the term SWE-bench refers by default to the SWE-bench Verified subset.

\textbf{Polyglot.} Polyglot includes tasks in multiple programming languages (C++, Rust, Python, etc.)~\citep{gauthier2024polyglot}. Compared to SWE-bench, one of the most widely used coding benchmarks and likely included in the training sets of FMs, Polyglot is more niche and less likely to be included in FMs' post-training data. Additionally, Polyglot is primarily used for evaluating Aider~\citep{aider2024} by its developer. This provides an opportunity to compare automatically designed agents with a representative coding agent in the community, which has been continuously developed and tested against the Polyglot benchmark over a year by human developers. Finally, Polyglot enables testing of the DGM's ability to generalize across multiple programming languages beyond just Python. We adopt a similar setup to the Polyglot leaderboard~\citep{gauthier2024polyglot}, with one key difference: the leaderboard reports pass@2 performance, where the agent can view feedback from ground-truth tests once. In contrast, we use a pass@1 setting, where the agent never sees the results of ground-truth tests.

Since it is expensive to evaluate on the full benchmarks (\Cref{app:cost-estimate}), we use a staged evaluation strategy to estimate the coding agent's performance. In general, we evaluate the coding agent on more tasks when we are more confident that it is a relatively strong performer. We first evaluate each coding agent on a set of 10 tasks to verify basic codebase-editing functionality. Only agents that retain the ability to edit code can solve downstream coding tasks or perform further self-modifications. Agents that pass this initial stage are then evaluated on an expanded set of 50 tasks. For SWE-bench, tasks are selected based on SWE-bench-verified-mini~\citep{hobbhahn2025swebenchmini}, which is designed to be a representative sample of the full benchmark (\Cref{app:swebench-tasks}). For Polyglot, tasks are chosen as a random subset of the full benchmark (\Cref{app:polyglot-tasks}). Because the LLMs we use are inherently stochastic, performance can be noisy. We use smaller subset evaluations to estimate general effectiveness and identify top candidates. For SWE-bench, if a coding agent achieves over 40\% success (a heuristic chosen based on the noise observed in preliminary runs) on the 60-task subset and ranks among the top two performers in the archive, it is then evaluated on 200 tasks (140 plus the previous 60) to more accurately assess its true coding capability (\Cref{app:swebench-tasks}). For Polyglot, if a coding agent achieves over 40\% success on the 10-task subset, it is evaluated on an expanded 50-task subset (\Cref{app:polyglot-tasks}).

\subsection{Baselines}
\label{sec:baselines}

To isolate and evaluate the contribution of the DGM's core components, we compare the DGM against two baselines: DGM without self-improvement (\textbf{DGM w/o self-improve}) and DGM without open-ended exploration (\textbf{DGM w/o open-ended exploration}). DGM w/o self-improve replicates the approach of ADAS~\citep{hu2025automated} in this setting, wherein the meta agent responsible for modifying the coding agents remains fixed as the initial agent throughout the experiment. This baseline allows us to assess the role of iterative self-improvement in accelerating the evolution of better coding agents. DGM w/o open-ended exploration eliminates the use of an archive and always self-modifies the latest stored version of itself. If a coding agent self-modifies to the point where it loses the basic functionality required to edit a codebase, it can no longer modify itself or solve any coding task. Therefore, DGM w/o open-ended exploration retains the latest version of itself that still maintains the basic functionality for codebase editing. This baseline allows us to evaluate the impact of having an archive and the well-documented beneficial principles of open-ended exploration~\citep{clune2019ai, stanley2015greatness, zhang2024omni, fernandopromptbreeder, lee2020learning, samvelyan2024rainbow, colas2022autotelic} in guiding the agent's evolution.

In addition to the learned baselines, we compare the DGM against handcrafted, open-source solutions. For SWE-bench, we take the state-of-the-art (SoTA) open-source solution that has been checked (i.e., the SWE-bench team was able to reproduce the results) (\Cref{app:swebench-sota}). For Polyglot, we take the representative agent (Aider)~\citep{aider2024}, which is open-sourced and designed to support multiple programming languages and large codebase editing (\Cref{app:polyglot-sota}). For a fair comparison, we measure the percentage of solved tasks on the same benchmark subsets used to evaluate the DGM (\Cref{app:swebench-tasks}, \Cref{app:polyglot-tasks}). These values are shown as dotted horizontal lines in \Cref{fig:dgm-comparisons}.

\subsection{Results}
\label{sec:results}

\begin{figure}[ht]
    \centering
    \begin{subfigure}[t]{0.48\textwidth}
        \centering
        \includegraphics[width=\textwidth]{figures/dgm_comparisons.pdf}
        \label{fig:dgm-comparisons-swe}
    \end{subfigure}
    \hfill
    \begin{subfigure}[t]{0.48\textwidth}
        \centering
        \includegraphics[width=\textwidth]{figures/dgm_comparisons_polyglot.pdf}
        \label{fig:dgm-comparisons-polyglot}
    \end{subfigure}
    \vspace{-15pt}
    \caption{\textbf{Self-improvement and open-ended exploration enable the DGM to continue making progress and improve its performance.} The DGM automatically discovers increasingly better coding agents and performs better on both (Left) SWE-bench and (Right) Polyglot. It outperforms baselines that lack either self-improvement or open-ended exploration, showing that both components are essential for continual self-improvement. These scores are obtained from evaluating on the benchmark subsets detailed in \Cref{sec:benchmarks}.}
    \label{fig:dgm-comparisons}
\end{figure}

\begin{figure}[ht]
    \centering
    \includegraphics[width=0.48\textwidth]{figures/dgm_archive.pdf}
    \includegraphics[width=0.48\textwidth]{figures/dgm_progress.pdf}
    \caption{\textbf{The DGM automatically self-improves to become a better coding agent.} (Left) Archive of coding agents generated during the DGM run on SWE-bench. Each node represents a coding agent, with node 0 corresponding to the initial agent. Node color indicates performance on SWE-bench (percentage of solved tasks), while border color reflects the number of tasks for which the agent was evaluated. Edges show which agents self-modified to produce the offsprings. Many paths to innovation traverse lower-performing nodes, and key innovations (like node 24) lead to an explosion of innovations built on top of them. Both properties underscore the benefits of open-ended search. (Right) Progress plot of the DGM on SWE-bench. The light blue line shows the average score of all agents possessing basic codebase-editing functionality. The blue line tracks the best score achieved by any agent in the archive at each iteration. The dark line shows the lineage of the final best-discovered agent and its precursor nodes, which includes two performance dips. This illustrates the benefits of open-ended search, which explores a diverse set of interesting stepping stones instead of focusing only on branching off the best solution found so far.}
    \label{fig:dgm-archive}
    \label{fig:dgm-progress}
    \vspace{-5mm}
\end{figure}

After 80 iterations of the DGM, the coding agent's performance increases from 20.0\% to 50.0\% on SWE-bench, and from 14.0\% to 38.0\% on Polyglot (\Cref{fig:dgm-comparisons}). Since the DGM is evaluated on only 50 tasks in the Polyglot experiment setup (\Cref{sec:benchmarks}), we additionally evaluate both the initial agent and the best DGM-discovered agent on the full Polyglot benchmark to more accurately estimate the improvement. On the full Polyglot benchmark, the DGM improves the coding agent from 14.2\% to 30.7\%. This shows that the DGM can automatically self-improve to create a better coding agent. Moreover, the performance of the best DGM-discovered agent is comparable to that of the checked, open-source, human-designed SoTA on SWE-bench (\Cref{fig:dgm-comparisons}). On Polyglot, although the DGM starts with an initial agent whose performance is lower than that of Aider, it discovers an agent that far surpasses Aider (\Cref{fig:dgm-comparisons}). The DGM-discovered agents are comparable to or outperform handcrafted agents on both benchmarks. While the SoTA SWE-bench agent and Aider were painstakingly shaped by human efforts, the DGM hints at a future in which such ingenuity is automated, evolving through self-referential cycles of continuous self-improvements.

The DGM automatically improves both the tools and the workflow of how FMs are utilized (\Cref{fig:dgm-progress}). For example, the DGM enhanced the edit tool to allow more granular file viewing (by lines) and more precise file editing (by string replacement), instead of always viewing or replacing the entire file. Workflow improvements include making multiple attempts to solve a task and using another FM to evaluate and select the best solution. Other workflow improvements include considering previous attempts when generating subsequent ones. \Cref{app:best-agent-dgm} and \Cref{app:best-agent-dgm-polyglot} show all modifications leading up to the final best-discovered agents on SWE-bench and Polyglot respectively. 

Because open-ended exploration allows branching from any agent in the archive with non-zero probability, the DGM can get out of deceptive dips or peaks in performance. For example, at iterations 4 and 56 of the experiment on SWE-bench, although the agent's score temporarily fell below that of its parent, the DGM was still able to explore innovations along that path and create a new agent that outperformed all of its predecessors (\Cref{fig:dgm-progress}). Furthermore, open-ended exploration allows different implementations of the same target functionality to be attempted. For example, while the goal is to provide finer‑grained editing tools, the specific implementation of this feature can vary greatly and hence lead to very different performance (\Cref{app:supp-dgm-diffimpl}). The DGM can explore multiple implementations to find the most suitable one and avoid getting trapped in a suboptimal one.

The DGM outperforms the baselines of DGM w/o self-improve and DGM w/o open-ended exploration on both benchmarks (\Cref{fig:dgm-comparisons}). Without updating the meta agent that modifies coding agents, DGM w/o self-improve improves the agents in early iterations, but its gains taper off quickly (\Cref{app:plots-swe}). In DGM w/o open-ended exploration, only the most recent agent is retained, so a poorly performing self-modification makes subsequent improvements harder to achieve (\Cref{app:plots-swe}).

\begin{figure}[ht]
    \centering
    \includegraphics[width=0.95\textwidth]{figures/transfer_model_task.pdf}
    \caption{\textbf{Transfer between Models and Tasks.} (Left and Middle) The superior performance of DGM-discovered agents can be transferred across different models and (Right) different task domains, such as from Python tasks to tasks in other languages like Rust, C++, Go, and others.}
    \label{fig:transfer-overview}
    \vspace{-0.2cm}
\end{figure}
        
To evaluate the generality of the improvements from the DGM, we tested the initial agent (\Cref{sec:experiment-setup}) and the best agent discovered during the DGM run (\Cref{fig:dgm-comparisons}) with different FMs than those used during optimization. For SWE-bench, where the DGM was run using Claude 3.5 Sonnet (New), we replaced the FM with Claude 3.7 Sonnet~\citep{anthropic2025claude37} or o3-mini, and evaluated on 200 tasks (\Cref{fig:transfer-overview}, Left). With o3-mini, the initial agent achieved 23.0\% and the DGM-discovered agent 33.0\%. With Claude 3.7 Sonnet, the initial agent achieved 19.0\% and the DGM-discovered agent 59.5\%. For Polyglot, where the DGM was run with o3-mini, we replaced the FM with Claude 3.5 Sonnet (New) or Claude 3.7 Sonnet, and evaluated on the full benchmark (\Cref{fig:transfer-overview}, Middle). With Claude 3.5 Sonnet (New), the initial agent achieved 32.0\% and the DGM-discovered agent 33.3\%. With Claude 3.7 Sonnet, the initial agent achieved 35.6\% and the DGM-discovered agent 36.8\%. These results suggest that the DGM yields improvements that generalize across FMs, rather than being tightly coupled to the specific FM used during its run (\Cref{fig:transfer-overview}).

Furthermore, to evaluate the transferability of the DGM-discovered agent across programming languages, we experiment with a version of the DGM trained exclusively on Python tasks from Polyglot and then transfer the discovered agent to tasks in other languages. Focusing primarily on Python tasks slightly improves performance on Python tasks but reduces performance on non-Python tasks compared to the DGM trained on all languages (\Cref{fig:transfer-overview}, Right). However, after being transferred from Python to other unseen languages during the search, the agent still achieves performance comparable to that of the DGM trained on all languages and substantially outperforms both the initial agent and Aider. These results demonstrate the robustness of the discovered improvements, showing that they do not overfit to a specific programming language.

\section{Safety Discussion}
\label{sec:safety}

Systems capable of self-improvement, such as the DGM, represent a step toward more autonomous AI development, aligning with long-standing goals in the field of making capable AI that can benefit humanity~\citep{schmidhuber1987evolutionary, clune2019ai, markoff2016machines, lehman2023machine}. However, this capability introduces unique safety considerations stemming from the system's ability to autonomously modify its own code. Modifications optimized solely for benchmark performance might inadvertently introduce vulnerabilities or behaviors misaligned with human intentions, even if they improve the target metric~\citep{bostrom2020ethical}. In particular, if evaluation benchmarks do not fully capture all desired agent properties (e.g., safety and robustness), the self-improvement loop could amplify misalignment over successive generations. Iterative self-modification could also lead to increasingly complex and uninterpretable internal logic, hindering human understanding, oversight, and control~\citep{sheth2025safety, anwar2024foundational, greenblatt2024alignment, ganguli2022red}.

Recognizing these challenges, the current implementation and experimental setup of the DGM incorporates several safeguards. All agent execution and self-modification processes are conducted within isolated sandboxed environments, limiting their ability to affect the host system, and thereby mitigating the risk of unintended actions. Each execution within the sandbox is subjected to a strict time limit, reducing the risk of resource exhaustion or unbounded behavior. The self-improvement process is currently confined to the well-defined domain of enhancing performance on specific coding benchmarks by modifying the agent's own Python codebase, thus limiting the scope of potential modifications. Additionally, we actively monitor agent performance and code changes, with the DGM archive providing a traceable lineage of modifications for review. At this stage, we have found no evidence of harmful or malicious behavior in the generated agents, and the self-modifications have been primarily focused on improving coding capabilities.

Conversely, a significant potential benefit of the self-improvement paradigm is that it could, in principle, be directed toward enhancing safety and interpretability themselves. We conduct a preliminary investigation into how the DGM can be deployed in AI safety settings to develop countermeasures for FM hallucination (\Cref{app:dgm-halluc}). Just as the DGM learns to improve its coding capabilities, it could potentially discover and integrate better internal safeguards or modify itself for greater transparency (e.g., incorporating principles akin to Constitutional AI~\citep{bai2022constitutional}), if such properties were included in its evaluation criteria~\citep{rosser2025agentbreeder}. This suggests a promising, albeit challenging, pathway in which self-improvement becomes a tool for building more trustworthy AI systems. Additional research could also explore weaving Constitutional AI in from the start, though the challenge would be incentivizing the system to retain these directives (an option worth exploring is to create an unmodifiable part of the system to be able to evaluate at halt the rest).

The DGM demonstrates the potential of self-improving AI while still operating within safe research boundaries due to the current limitations of frontier FMs and effective mitigations like sandboxing. We include this safety discussion proactively to raise awareness about the emerging prospect of self-improving AI systems and their associated safety implications, particularly as these systems inevitably become more capable~\citep{yudkowsky2008artificial, bostrom2002a,ecoffet2020open, bengio2024managing,clune2019ai}. Accordingly, we advocate for continued investigation into the safe and beneficial evolution of AI-Generating Algorithms~\citep{clune2019ai} and self-improving systems.

\section{Conclusion and Limitations}
\label{sec:conclusion}

We introduce the Darwin G\"odel Machine (DGM), the first self-improving system powered by FMs with open-ended exploration, where progress on its evaluation benchmarks can directly translate into better self-improvement capabilities. We demonstrate the automatic discovery of better tools and FM systems, resulting in better performance on two benchmarks: SWE-bench and Polyglot. Through self-improvement and open-ended exploration, the DGM shows a continuous increase in performance, bringing us one step closer to self-accelerating, self-improving AI systems.

We demonstrate that the DGM can autonomously achieve performance on par with openly available solutions. However, it still falls short of closed-source SoTA SWE-bench solutions. An open question is whether running the DGM for longer would continue to yield performance gains and eventually surpass closed-source solutions. These closed-source solutions often rely on elaborately handcrafted techniques developed by teams of highly skilled experts. Since FMs have yet to match the capabilities of such experts (e.g., in reasoning), the DGM currently requires extensive compute to discover improvements. A single run of the DGM on SWE-bench, as presented in \Cref{sec:experiments}, takes about 2 weeks and incurs significant API costs (\Cref{app:cost-estimate}). We hypothesize that further progress will require more efficient use of computational resources and the development of better reasoning skills.

Since this version of the DGM is mainly powered by FMs, it is inherently limited by the capabilities of the underlying FM. Hence, an exciting future direction is to extend self-modification beyond just prompts or FM workflows, to include more computationally intensive methods, such as rewriting its own training script to update the FM itself. While this version of the DGM focuses on coding, AI systems are increasingly applied across a wide range of domains (e.g., computer vision, creative writing). Another promising extension is to develop self-improving AI systems capable of enhancing themselves beyond just the coding domain. A key assumption in this work is that coding benchmarks are a good reflection of the agent's ability to self-improve, since the self-modification task requires the agent to modify its own codebase. However, one could envision an alternative approach that co-evolves the target task distribution, thereby removing the constraint of self-improvement being tied to a single objective, as in true open-ended processes. As discussed in \Cref{sec:safety}, we must also continue to keep safety front and center as we explore this powerful technology.

In conclusion, the DGM represents a significant step toward the automation of AI development through self-improving systems capable of editing their own codebase. While current limitations in compute and reasoning constrain its full potential, continued advances in FMs and infrastructure may unlock more powerful and general-purpose self-improvements. Provided that the safety concerns are carefully navigated (\Cref{sec:safety}), the future of self-improving AI systems and AI-Generating Algorithms~\citep{clune2019ai} holds immense promise to open-endedly evolve AI, continually rewriting or retraining itself in pursuit of greater capabilities aligned with human values.


\begin{ack}
This research was supported by the Vector Institute, the Canada CIFAR AI Chairs program, a grant from Schmidt Futures, an NSERC Discovery Grant, and a generous donation from Rafael Cosman. Resources used in preparing this research were provided, in part, by the Province of Ontario, the Government of Canada through CIFAR, and companies sponsoring the Vector Institute (https://vectorinstitute.ai/partnerships/current-partners/). Any opinions, findings, and conclusions or recommendations expressed in this material are those of the authors and do not necessarily reflect the views of the sponsors. We also thank Aaron Dharna, Ben Norman, Cédric Colas, and Shyam Sudhakaran for insightful discussions and feedback.
\end{ack}

% \newpage
\bibliographystyle{plainnat}
\bibliography{main}

%%%%%%%%%%%%%%%%%%%%%%%%%%%%%%%%%%%%%%%%%%%%%%%%%%%%%%%%%%%%
\clearpage
\appendix

\section*{\LARGE Supplementary Material}

\vspace*{20pt}
\section*{Table of Contents}
\vspace*{-5pt}
\startcontents[sections]
\printcontents[sections]{l}{1}{\setcounter{tocdepth}{2}}

\clearpage

\section{Algorithmic Details}

\subsection{Initial Coding Agent}
\label{app:initial-agent}

In this section, we present the details of the tools available to the initial coding agent (\Cref{sec:experiment-setup}) and its task prompt.

Information of the given Bash tool:
\begin{lstlisting}
def tool_info():
    return {
        "name": "bash",
        "description": """Run commands in a bash shell\n
* When invoking this tool, the contents of the "command" parameter does NOT need to be XML-escaped.\n
* You don't have access to the internet via this tool.\n
* You do have access to a mirror of common linux and python packages via apt and pip.\n
* State is persistent across command calls and discussions with the user.\n
* To inspect a particular line range of a file, e.g. lines 10-25, try 'sed -n 10,25p /path/to/the/file'.\n
* Please avoid commands that may produce a very large amount of output.\n
* Please run long lived commands in the background, e.g. 'sleep 10 &' or start a server in the background.""",
        "input_schema": {
            "type": "object",
            "properties": {
                "command": {
                    "type": "string",
                    "description": "The bash command to run."
                }
            },
            "required": ["command"]
        }
    }
\end{lstlisting}

Information of the given Edit tool:
\begin{lstlisting}
def tool_info():
    return {
        "name": "editor",
        "description": """Custom editing tool for viewing, creating, and editing files\n
* State is persistent across command calls and discussions with the user.\n
* If `path` is a file, `view` displays the entire file with line numbers. If `path` is a directory, `view` lists non-hidden files and directories up to 2 levels deep.\n
* The `create` command cannot be used if the specified `path` already exists as a file.\n
* If a `command` generates a long output, it will be truncated and marked with `<response clipped>`.\n
* The `edit` command overwrites the entire file with the provided `file_text`.\n
* No partial/line-range edits or partial viewing are supported.""",
        "input_schema": {
            "type": "object",
            "properties": {
                "command": {
                    "type": "string",
                    "enum": ["view", "create", "edit"],
                    "description": "The command to run: `view`, `create`, or `edit`."
                },
                "path": {
                    "description": "Absolute path to file or directory, e.g. `/repo/file.py` or `/repo`.",
                    "type": "string"
                },
                "file_text": {
                    "description": "Required parameter of `create` or `edit` command, containing the content for the entire file.",
                    "type": "string"
                }
            },
            "required": ["command", "path"]
        }
    }
\end{lstlisting}

Task prompt:
\begin{lstlisting}
I have uploaded a Python code repository in the directory {self.git_tempdir}. Help solve the following problem.

<problem_description>
{self.problem_statement}
</problem_description>

<test_description>
{self.test_description}
</test_description>

Your task is to make changes to the files in the {self.git_tempdir} directory to address the <problem_description>. I have already taken care of the required dependencies.
\end{lstlisting}


\subsection{Parent Selection}
\label{app:parent-select}

At each DGM iteration, we select a subset of agents from the archive as parents to self-modify and produce new child agents (\Cref{sec:methods}). The details of the parent selection process are outlined below.

At each iteration \(t\) of the DGM run, let
\[
\mathcal{A}^t = \{a_1^t, a_2^t, \dots, a_N^t\}
\]
be the archive of candidate agents.  We first define the eligible set
\[
\mathcal{E}^t = \bigl\{\,a_i^t \in \mathcal{A}^t : \alpha_i < 1 \bigr\},
\]
i.e.\ only those agents whose performance score is not yet perfect. We then sample \(k\) parents (with replacement) from \(\mathcal{E}^t\) in proportion to a combined measure of performance and number of children with codebase-editing functionality. Concretely, for each agent \(a_i^t \in \mathcal{E}^t\):

\begin{align}
\text{(performance)}\quad & \alpha_i \;=\; \mathrm{performance}(a_i^t), 
\\
\text{(children count)}\quad & n_i \;=\; \mathrm{functioning\_children\_count}(a_i^t),
\\[6pt]
\text{(sigmoid‐scaled performance)}\quad 
& s_i \;=\; \frac{1}{1 + \exp\!\bigl(-\lambda\,(\alpha_i - \alpha_0)\bigr)},
\\[4pt]
\text{(novelty bonus)}\quad 
& h_i \;=\; \frac{1}{1 + n_i},
\\[4pt]
\text{(unnormalized weight)}\quad 
& w_i \;=\; s_i\,h_i,
\\[4pt]
\text{(selection probability)}\quad 
& p_i \;=\; \frac{w_i}{\sum_{j=1}^{N} w_j},
\\[8pt]
\text{(sampling)}\quad 
& \{\,\text{parents}\,\} \;\sim\; \mathrm{Categorical}\bigl(\{p_i\}_{a_i^t \in \mathcal{E}^t}\bigr)\quad\text{(draw \(k\) samples).}
\end{align}

Here, \(\lambda>0\) controls the sharpness of the sigmoid, and \(\alpha_0\) is the midpoint accuracy. We set \(\lambda=10\) and \(\alpha_0=0.5\). \(k\) determines how many self-improvements are done in parallel during each iteration of the DGM run. We set \(k=2\) in SWE-bench experiments, and \(k=4\) in Polyglot experiments.

This parent selection favors agents with high performance and fewer existing children, thereby promoting both exploitation and exploration in the open-ended process. All agents in the archive maintain a non-zero probability of being sampled, meaning that any path to improvement is feasible given sufficient compute.

\subsection{Self-Improve Prompts}
\label{app:selfimprove-prompts}

For a chosen coding agent to self-modify, we feed the evaluation logs from its performance on the coding benchmark into an FM to diagnose the next potential improvement (\Cref{sec:methods}). We chose to use OpenAI's o1 model~\citep{openai2024o1} for this purpose, as we found it had better reasoning capabilities than other FMs available at the time of our experiments. The prompts used for this diagnosis are provided below. From the diagnostic output, the \texttt{implementation\_suggestion} and \texttt{problem\_description} are extracted and passed to the coding agent as its self-modification task.

Self-improve diagnosis prompt for SWE-bench experiments:
\begin{tcolorbox}[breakable,colback=orange!5!white, colframe=orange!80!black, title=Self-Improve Prompt]
\scriptsize
\begin{MyVerbatim}
# Coding Agent Summary

- **Main File**: `coding_agent.py`
  - Primary Class: `AgenticSystem`
  - The `forward()` function is the central entry point.
  - Prompts are located either within the `forward()` function or in the `prompts/` directory.
- **Tools**: `tools/`
  - The `tools/` directory contains various tools that LLMs can use to perform specific tasks.
  - Each tool must have a `tool_info()` function that returns a JSON object containing 'name', 'description', and 'input_schema'. The 'input_schema' should be a JSON object containing 'type', 'properties', and 'required'.
  - Each tool must have a `tool_function()` function that takes the arguments defined in input_schema, performs the tool's task, and returns a string.
  - See other tools for reference.
- **Utilities**: `utils/`
  - The `utils/` directory contains utility functions used across the codebase.

- **Additional Details**:
  - The agent is very good at automatically utilizing the right available tools at the right time. So do not have an agentic flow that explicitly forces a tool's usage.
  - Common tools, such as file editing and bash commands, are easy for the agent to recognize and use appropriately. However, more complex and niche tools may require explicit instructions in the prompt.
  - Tools should be designed to be as general as possible, ensuring they work across any GitHub repository. Avoid hardcoding repository-specific details or behaviors (e.g., paths).
  - Do not use 'while True' loops in the agent's code. This can cause the agent to get stuck and not respond.
  - Verify the implementation details of helper functions prior to usage to ensure proper integration and expected behavior.
  - Do not install additional packages or dependencies directly. Update `requirements.txt` if new dependencies are required and install them using `pip install -r requirements.txt`.

Here is the implementation of the coding agent.

# Coding Agent Implementation
----- Coding Agent Implementation Start -----
{code}
----- Coding Agent Implementation End -----

Your task is to identify ONE detailed plan that would improve the agent's coding ability. The improvement should not be specific to any particular GitHub issue or repository.

Here is the log for the coding agent trying to solve the GitHub issues but failed.

# Agent Running Log
----- Agent Running Log Start -----
{md_log}
----- Agent Running Log End -----

# GitHub Issue
The GitHub issue that the agent is trying to solve.
----- GitHub Issue Start -----
{github_issue}
----- GitHub Issue End -----

# Predicted Patch
The agent's predicted patch to solve the issue.
----- Predicted Patch Start -----
{predicted_patch}
----- Predicted Patch End -----

# Private Test Patch
SWE-bench's official private tests to detect whether the issue is solved. This is not available to the agent during evaluation. The agent should try to implement its own tests.
----- Private Test Patch Start -----
{test_patch}
----- Private Test Patch End -----

# Issue Test Results
The test results from SWE-bench using the above official private tests.
----- Issue Test Results Start -----
{eval_log}
----- Issue Test Results End -----

Respond precisely in the following format including the JSON start and end markers:

```json
<JSON>
```

In <JSON>, provide a JSON response with the following fields:
- "log_summarization": Analyze the above logs and summarize how the agent tried to solve the GitHub issue. Note which tools and how they are used, the agent's problem-solving approach, and any issues encountered.
- "potential_improvements": Identify potential improvements to the coding agent that could enhance its coding capabilities. Focus on the agent's general coding abilities (e.g., better or new tools usable across any repository) rather than issue-specific fixes (e.g., tools only usable in one framework). All necessary dependencies and environment setup have already been handled, so do not focus on these aspects.
- "improvement_proposal": Choose ONE high-impact improvement from the identified potential improvements and describe it in detail. This should be a focused and comprehensive plan to enhance the agent's overall coding ability.
- "implementation_suggestion": Referring to the coding agent's summary and implementation, think critically about what feature or tool could be added or improved to best implement the proposed improvement. If the proposed feature can be implemented by modifying the existing tools, describe the modifications needed, instead of suggesting a new tool.
- "problem_description": Phrase the improvement proposal and implementation suggestion as a GitHub issue description. It should clearly describe the feature so that a software engineer viewing the issue and the repository can implement it.

Your response will be automatically parsed, so ensure that the string response is precisely in the correct format. Do NOT include the `<JSON>` tag in your output.
\end{MyVerbatim}
\end{tcolorbox}


Additional prompt for Polyglot experiments:

\begin{tcolorbox}[breakable,colback=orange!5!white, colframe=orange!80!black, title=Additional Prompt for o3-mini Tool Use and Polyglot]
\scriptsize
\begin{MyVerbatim}
### DOC: tool function schema

Carefully consider whether to add/enhance the current tool or edit the workflow in forward()

Pay special attention to making sure that "required" and "type" are always at the correct level of nesting. For example, "required" should be at the same level as "properties", not inside it.
Make sure that every property, no matter how short, has a type and description correctly nested inside it.
Other arguments than you have seen are not permitted. For example, in "edit_line_ranges" with "type": "array", arguments like "minItems" and "maxItems" are not permitted.

...

Here is the log for the coding agent trying to solve a programming task. 
A task is in one programming language, but the coding agent needs to deal with different languages including C++, Go, Java, JavaScript, Python, and Rust.
\end{MyVerbatim}
\end{tcolorbox}


\subsection{Pseudocode}
\label{app:pseudocode}

This is the pseudocode of the DGM algorithm, described in \Cref{sec:methods}.

\begin{algorithm}[H]
\DontPrintSemicolon
\SetAlgoLined

\KwIn{Initial coding agent $g_0$, benchmark suite $B$, maximum iterations $T$}
\KwOut{Archive of agents $\mathcal{A}$}

%— Initialization —%
\BlankLine
\textbf{initialize} $\mathcal{A} \leftarrow \{\,g_0\}$ \tcp*{Start with the initial agent}

%— Evolutionary loop —%
\For{$t \leftarrow 1$ \KwTo $T$}{
    % Parent selection
    $\mathcal{P} \leftarrow \text{SelectParents}(\mathcal{A})$\ \tcp*{Select parent agents}

    \ForEach{$p \in \mathcal{P}$}{
        % Self-modification
        $c \leftarrow p.\text{self\_modify}(f)$\ \tcp*{Self-modification}

        % Benchmark evaluation
        $s \leftarrow \text{evaluate}(c, B)$\ \tcp*{Evaluate on benchmark}

        % Validity check
        \If{$c.\text{is\_valid}()$}{
            $\mathcal{A} \leftarrow \mathcal{A} \,\cup\, \{(c, s)\}$\ \tcp*{Keep children capable of codebase editing}
        }
    }
}

\Return{$\mathcal{A}$}
\caption{Darwin Gödel Machine}
\label{alg:dgm}
\end{algorithm}

\section{Benchmark Details}

\subsection{Cost Estimate}
\label{app:cost-estimate}

The estimated cost of completing a single run of the DGM on SWE-bench, as presented in \Cref{sec:experiments}, is about USD 22,000. A more granular break down is:

\begin{table}[H]
\centering
\begin{tabular}{@{}cccc@{}}
\toprule
\textbf{LLM} & \textbf{Benchmark} & \textbf{Number of Tasks} & \textbf{Cost Estimate (USD)} \\ \midrule
Claude 3.5 Sonnet (New) & SWE-bench & 60 & \$350 \\
o3-mini & Polyglot & 60 & \$5 \\ \bottomrule
\end{tabular}
\end{table}

\subsection{SWE-bench Tasks}
\label{app:swebench-tasks}

Initial 10 tasks for verifying basic functionality of a coding agent:
\begin{multicols}{2}
\begin{itemize}
    \item \texttt{django\_\_django-10973}
    \item \texttt{django\_\_django-11066}
    \item \texttt{django\_\_django-12754}
    \item \texttt{django\_\_django-15930}
    \item \texttt{django\_\_django-13279}
    \item \texttt{django\_\_django-16661}
    \item \texttt{django\_\_django-13346}
    \item \texttt{django\_\_django-10880}
    \item \texttt{django\_\_django-10999}
    \item \texttt{django\_\_django-11087}
\end{itemize}
\end{multicols}

Additional 50 tasks for estimating general effectiveness of a coding agent:
\begin{multicols}{2}
\begin{itemize}
  \item \texttt{django\_\_django-9296}
  \item \texttt{django\_\_django-11790}
  \item \texttt{django\_\_django-11815}
  \item \texttt{django\_\_django-11848}
  \item \texttt{django\_\_django-11880}
  \item \texttt{django\_\_django-11885}
  \item \texttt{django\_\_django-11951}
  \item \texttt{django\_\_django-11964}
  \item \texttt{django\_\_django-11999}
  \item \texttt{django\_\_django-12039}
  \item \texttt{django\_\_django-12050}
  \item \texttt{django\_\_django-12143}
  \item \texttt{django\_\_django-12155}
  \item \texttt{django\_\_django-12193}
  \item \texttt{django\_\_django-12209}
  \item \texttt{django\_\_django-12262}
  \item \texttt{django\_\_django-12273}
  \item \texttt{django\_\_django-12276}
  \item \texttt{django\_\_django-12304}
  \item \texttt{django\_\_django-12308}
  \item \texttt{django\_\_django-12325}
  \item \texttt{django\_\_django-12406}
  \item \texttt{django\_\_django-12708}
  \item \texttt{django\_\_django-12713}
  \item \texttt{django\_\_django-12774}
  \item \texttt{sphinx-doc\_\_sphinx-7454}
  \item \texttt{sphinx-doc\_\_sphinx-7590}
  \item \texttt{sphinx-doc\_\_sphinx-7748}
  \item \texttt{sphinx-doc\_\_sphinx-7757}
  \item \texttt{sphinx-doc\_\_sphinx-7985}
  \item \texttt{sphinx-doc\_\_sphinx-8035}
  \item \texttt{sphinx-doc\_\_sphinx-8056}
  \item \texttt{sphinx-doc\_\_sphinx-8265}
  \item \texttt{sphinx-doc\_\_sphinx-8269}
  \item \texttt{sphinx-doc\_\_sphinx-8475}
  \item \texttt{sphinx-doc\_\_sphinx-8548}
  \item \texttt{sphinx-doc\_\_sphinx-8551}
  \item \texttt{sphinx-doc\_\_sphinx-8638}
  \item \texttt{sphinx-doc\_\_sphinx-8721}
  \item \texttt{sphinx-doc\_\_sphinx-9229}
  \item \texttt{sphinx-doc\_\_sphinx-9230}
  \item \texttt{sphinx-doc\_\_sphinx-9281}
  \item \texttt{sphinx-doc\_\_sphinx-9320}
  \item \texttt{sphinx-doc\_\_sphinx-9367}
  \item \texttt{sphinx-doc\_\_sphinx-9461}
  \item \texttt{sphinx-doc\_\_sphinx-9698}
  \item \texttt{sphinx-doc\_\_sphinx-10449}
  \item \texttt{sphinx-doc\_\_sphinx-10466}
  \item \texttt{sphinx-doc\_\_sphinx-10673}
  \item \texttt{sphinx-doc\_\_sphinx-11510}
\end{itemize}
\end{multicols}

Additional 140 tasks for more accurate assessment of a coding agent's performance:
\begin{multicols}{2}
\begin{itemize}
    \item \texttt{astropy\_\_astropy-12907}
    \item \texttt{astropy\_\_astropy-13033}
    \item \texttt{astropy\_\_astropy-13236}
    \item \texttt{astropy\_\_astropy-13398}
    \item \texttt{astropy\_\_astropy-13453}
    \item \texttt{astropy\_\_astropy-13579}
    \item \texttt{astropy\_\_astropy-13977}
    \item \texttt{astropy\_\_astropy-14096}
    \item \texttt{astropy\_\_astropy-14182}
    \item \texttt{astropy\_\_astropy-14309}
    \item \texttt{astropy\_\_astropy-14365}
    \item \texttt{astropy\_\_astropy-14369}
    \item \texttt{astropy\_\_astropy-14508}
    \item \texttt{astropy\_\_astropy-14539}
    \item \texttt{astropy\_\_astropy-14598}
    \item \texttt{astropy\_\_astropy-14995}
    \item \texttt{astropy\_\_astropy-7166}
    \item \texttt{astropy\_\_astropy-7336}
    \item \texttt{astropy\_\_astropy-7606}
    \item \texttt{astropy\_\_astropy-7671}
    \item \texttt{astropy\_\_astropy-8707}
    \item \texttt{astropy\_\_astropy-8872}
    \item \texttt{django\_\_django-10097}
    \item \texttt{django\_\_django-10554}
    \item \texttt{django\_\_django-10914}
    \item \texttt{django\_\_django-11095}
    \item \texttt{django\_\_django-11099}
    \item \texttt{django\_\_django-11119}
    \item \texttt{django\_\_django-11133}
    \item \texttt{django\_\_django-11138}
    \item \texttt{django\_\_django-11141}
    \item \texttt{django\_\_django-11149}
    \item \texttt{django\_\_django-11163}
    \item \texttt{django\_\_django-11179}
    \item \texttt{django\_\_django-11206}
    \item \texttt{django\_\_django-11211}
    \item \texttt{django\_\_django-11239}
    \item \texttt{django\_\_django-11265}
    \item \texttt{django\_\_django-11276}
    \item \texttt{django\_\_django-11292}
    \item \texttt{django\_\_django-11299}
    \item \texttt{django\_\_django-11333}
    \item \texttt{django\_\_django-11400}
    \item \texttt{django\_\_django-11433}
    \item \texttt{django\_\_django-11451}
    \item \texttt{django\_\_django-11477}
    \item \texttt{django\_\_django-11490}
    \item \texttt{django\_\_django-11532}
    \item \texttt{django\_\_django-11551}
    \item \texttt{django\_\_django-11555}
    \item \texttt{django\_\_django-11603}
    \item \texttt{django\_\_django-11728}
    \item \texttt{django\_\_django-11734}
    \item \texttt{django\_\_django-11740}
    \item \texttt{django\_\_django-11749}
    \item \texttt{django\_\_django-11820}
    \item \texttt{django\_\_django-12125}
    \item \texttt{django\_\_django-12419}
    \item \texttt{django\_\_django-12663}
    \item \texttt{django\_\_django-12741}
    \item \texttt{django\_\_django-12858}
    \item \texttt{django\_\_django-12965}
    \item \texttt{django\_\_django-13012}
    \item \texttt{django\_\_django-13023}
    \item \texttt{django\_\_django-13028}
    \item \texttt{django\_\_django-13033}
    \item \texttt{django\_\_django-13089}
    \item \texttt{django\_\_django-13109}
    \item \texttt{django\_\_django-13112}
    \item \texttt{django\_\_django-13121}
    \item \texttt{django\_\_django-13128}
    \item \texttt{django\_\_django-13158}
    \item \texttt{django\_\_django-13195}
    \item \texttt{django\_\_django-13212}
    \item \texttt{django\_\_django-13297}
    \item \texttt{django\_\_django-13315}
    \item \texttt{django\_\_django-13343}
    \item \texttt{django\_\_django-13344}
    \item \texttt{django\_\_django-13363}
    \item \texttt{django\_\_django-13401}
    \item \texttt{django\_\_django-13406}
    \item \texttt{django\_\_django-13410}
    \item \texttt{django\_\_django-13417}
    \item \texttt{django\_\_django-13449}
    \item \texttt{django\_\_django-13512}
    \item \texttt{django\_\_django-13513}
    \item \texttt{django\_\_django-13516}
    \item \texttt{django\_\_django-13551}
    \item \texttt{django\_\_django-13568}
    \item \texttt{django\_\_django-13569}
    \item \texttt{django\_\_django-13590}
    \item \texttt{django\_\_django-13658}
    \item \texttt{django\_\_django-13670}
    \item \texttt{django\_\_django-13741}
    \item \texttt{django\_\_django-13786}
    \item \texttt{django\_\_django-13794}
    \item \texttt{django\_\_django-13807}
    \item \texttt{django\_\_django-13809}
    \item \texttt{django\_\_django-13810}
    \item \texttt{django\_\_django-13820}
    \item \texttt{django\_\_django-13821}
    \item \texttt{django\_\_django-13837}
    \item \texttt{django\_\_django-13925}
    \item \texttt{django\_\_django-13933}
    \item \texttt{django\_\_django-13964}
    \item \texttt{django\_\_django-14007}
    \item \texttt{django\_\_django-14011}
    \item \texttt{django\_\_django-14017}
    \item \texttt{django\_\_django-14034}
    \item \texttt{django\_\_django-14053}
    \item \texttt{django\_\_django-14089}
    \item \texttt{django\_\_django-14122}
    \item \texttt{django\_\_django-14140}
    \item \texttt{django\_\_django-14155}
    \item \texttt{django\_\_django-14170}
    \item \texttt{django\_\_django-14238}
    \item \texttt{django\_\_django-14311}
    \item \texttt{django\_\_django-14315}
    \item \texttt{django\_\_django-14349}
    \item \texttt{django\_\_django-14351}
    \item \texttt{django\_\_django-14373}
    \item \texttt{django\_\_django-14376}
    \item \texttt{django\_\_django-14404}
    \item \texttt{django\_\_django-14434}
    \item \texttt{django\_\_django-14493}
    \item \texttt{django\_\_django-14500}
    \item \texttt{django\_\_django-14534}
    \item \texttt{django\_\_django-14539}
    \item \texttt{django\_\_django-14559}
    \item \texttt{django\_\_django-14580}
    \item \texttt{django\_\_django-14608}
    \item \texttt{django\_\_django-14631}
    \item \texttt{django\_\_django-14672}
    \item \texttt{django\_\_django-14725}
    \item \texttt{django\_\_django-14752}
    \item \texttt{django\_\_django-14765}
    \item \texttt{django\_\_django-14771}
    \item \texttt{django\_\_django-14787}
    \item \texttt{django\_\_django-14792}
    \item \texttt{django\_\_django-14855}
\end{itemize}
\end{multicols}

\subsection{Polyglot Tasks}
\label{app:polyglot-tasks}

Initial 10 tasks for verifying basic functionality of a coding agent:
\begin{multicols}{2}
\begin{itemize}
    \item \texttt{django\_\_django-10973}
    \item \texttt{django\_\_django-11066}
    \item \texttt{django\_\_django-12754}
    \item \texttt{django\_\_django-15930}
    \item \texttt{django\_\_django-13279}
    \item \texttt{django\_\_django-16661}
    \item \texttt{django\_\_django-13346}
    \item \texttt{django\_\_django-10880}
    \item \texttt{django\_\_django-10999}
    \item \texttt{django\_\_django-11087}
\end{itemize}
\end{multicols}

Additional 50 tasks for estimating general effectiveness of a coding agent:
\begin{multicols}{2}
\begin{itemize}
  \item \texttt{django\_\_django-9296}
  \item \texttt{django\_\_django-11790}
  \item \texttt{django\_\_django-11815}
  \item \texttt{django\_\_django-11848}
  \item \texttt{django\_\_django-11880}
  \item \texttt{django\_\_django-11885}
  \item \texttt{django\_\_django-11951}
  \item \texttt{django\_\_django-11964}
  \item \texttt{django\_\_django-11999}
  \item \texttt{django\_\_django-12039}
  \item \texttt{django\_\_django-12050}
  \item \texttt{django\_\_django-12143}
  \item \texttt{django\_\_django-12155}
  \item \texttt{django\_\_django-12193}
  \item \texttt{django\_\_django-12209}
  \item \texttt{django\_\_django-12262}
  \item \texttt{django\_\_django-12273}
  \item \texttt{django\_\_django-12276}
  \item \texttt{django\_\_django-12304}
  \item \texttt{django\_\_django-12308}
  \item \texttt{django\_\_django-12325}
  \item \texttt{django\_\_django-12406}
  \item \texttt{django\_\_django-12708}
  \item \texttt{django\_\_django-12713}
  \item \texttt{django\_\_django-12774}
  \item \texttt{sphinx-doc\_\_sphinx-7454}
  \item \texttt{sphinx-doc\_\_sphinx-7590}
  \item \texttt{sphinx-doc\_\_sphinx-7748}
  \item \texttt{sphinx-doc\_\_sphinx-7757}
  \item \texttt{sphinx-doc\_\_sphinx-7985}
  \item \texttt{sphinx-doc\_\_sphinx-8035}
  \item \texttt{sphinx-doc\_\_sphinx-8056}
  \item \texttt{sphinx-doc\_\_sphinx-8265}
  \item \texttt{sphinx-doc\_\_sphinx-8269}
  \item \texttt{sphinx-doc\_\_sphinx-8475}
  \item \texttt{sphinx-doc\_\_sphinx-8548}
  \item \texttt{sphinx-doc\_\_sphinx-8551}
  \item \texttt{sphinx-doc\_\_sphinx-8638}
  \item \texttt{sphinx-doc\_\_sphinx-8721}
  \item \texttt{sphinx-doc\_\_sphinx-9229}
  \item \texttt{sphinx-doc\_\_sphinx-9230}
  \item \texttt{sphinx-doc\_\_sphinx-9281}
  \item \texttt{sphinx-doc\_\_sphinx-9320}
  \item \texttt{sphinx-doc\_\_sphinx-9367}
  \item \texttt{sphinx-doc\_\_sphinx-9461}
  \item \texttt{sphinx-doc\_\_sphinx-9698}
  \item \texttt{sphinx-doc\_\_sphinx-10449}
  \item \texttt{sphinx-doc\_\_sphinx-10466}
  \item \texttt{sphinx-doc\_\_sphinx-10673}
  \item \texttt{sphinx-doc\_\_sphinx-11510}
\end{itemize}
\end{multicols}

\subsection{SWE-bench State-of-The-Art}
\label{app:swebench-sota}
At the time of writing this paper (16 April 2025), the highest performing, checked (i.e., the SWE-bench team received access to the system and were able to reproduce the patch generations), open-source entry on SWE-bench Verified is OpenHands + CodeAct v2.1 (claude-3-5-sonnet-20241022)~\citep{wang2024openhands}, achieving 53.0\%. Only considering the same subset of 200 tasks used by the DGM (\Cref{app:swebench-tasks}), OpenHands + CodeAct v2.1 (claude-3-5-sonnet-20241022) achieves 51.0\%.

\subsection{Polyglot Representative Agent}
\label{app:polyglot-sota}

Aider~\citep{aider2024}, a popular coding agent in the community, was published in Spring 2024. It has garnered over 33,000 stars on GitHub and has been continuously developed and tested against the Polyglot benchmark for over a year by human developers, primarily to evaluate its performance. Aider has also become a standard baseline for assessing the performance of different models, with the current top performers on the Polyglot benchmark being a mix of o3 (high) and GPT-4.1. We adopt a setup similar to that of the Polyglot leaderboard, with one key difference: the leaderboard reports pass@2 performance, where the agent can view feedback from ground-truth tests once. In contrast, we use a pass@1 setting, where the agent never sees the results of ground-truth tests, as we believe this more closely reflects realistic coding applications.

\section{Best-Discovered Agents}

\subsection{DGM on SWE-bench}
\label{app:best-agent-dgm}
Diff patches contributing to the best agent discovered by the DGM on SWE-bench:
\begin{lstlisting}[style=diffstyle]
diff --git a/coding_agent.py b/coding_agent.py
index 2cd395a..9a2cc2f 100644
--- a/coding_agent.py
+++ b/coding_agent.py
@@ -4,6 +4,7 @@ import logging
 from logging.handlers import RotatingFileHandler
 import os
 import threading
+import re
 
 from llm_withtools import CLAUDE_MODEL, OPENAI_MODEL, chat_with_agent
 from utils.eval_utils import get_report_score, msg_history_to_report, score_tie_breaker
@@ -63,6 +64,42 @@ def safe_log(message, level=logging.INFO):
     else:
         print(f"Warning: No logger found for thread {threading.get_ident()}")
 
+def is_patch_valid(patch_str):
+    """
+    Parse the patch to check if any non-test source files are modified.
+    Returns (bool, str) tuple: (is_valid, reason)
+    """
+    if not patch_str or patch_str.isspace():
+        return False, "Empty patch"
+    
+    # Parse the patch to find modified files
+    modified_files = []
+    diff_header_pattern = re.compile(r'^\+\+\+ b/(.+)$', re.MULTILINE)
+    for match in diff_header_pattern.finditer(patch_str):
+        filepath = match.group(1)
+        if filepath != '/dev/null':  # Skip deleted files
+            modified_files.append(filepath)
+    
+    if not modified_files:
+        return False, "No files modified"
+        
+    # Check if any non-test files are modified
+    test_patterns = (
+        lambda f: f.startswith('tests/'),
+        lambda f: f.startswith('test_'),
+        lambda f: f.endswith('_test.py')
+    )
+    
+    source_files = [
+        f for f in modified_files 
+        if not any(pattern(f) for pattern in test_patterns)
+    ]
+    
+    if not source_files:
+        return False, "Only test files were modified"
+    
+    return True, "Valid patch with source file modifications"
+
 class AgenticSystem:
     def __init__(
             self,
@@ -73,6 +110,7 @@ class AgenticSystem:
             test_description=None,
             self_improve=False,
             instance_id=None,
+            max_retries=3,
         ):
         self.problem_statement = problem_statement
         self.git_tempdir = git_tempdir
@@ -82,6 +120,7 @@ class AgenticSystem:
         self.self_improve = self_improve
         self.instance_id = instance_id if not self_improve else 'dgm'
         self.code_model = CLAUDE_MODEL
+        self.max_retries = max_retries
 
         # Initialize logger and store it in thread-local storage
         self.logger = setup_logger(chat_history_file)
@@ -153,7 +192,7 @@ Your task is to run the regression tests in the {self.git_tempdir} directory to
         """
         The forward function for the AgenticSystem.
         """
-        instruction = f"""I have uploaded a Python code repository in the directory {self.git_tempdir}. Help solve the following problem.
+        base_instruction = f"""I have uploaded a Python code repository in the directory {self.git_tempdir}. Help solve the following problem.
 
 <problem_description>
 {self.problem_statement}
@@ -165,7 +204,39 @@ Your task is to run the regression tests in the {self.git_tempdir} directory to
 
 Your task is to make changes to the files in the {self.git_tempdir} directory to address the <problem_description>. I have already taken care of the required dependencies.
 """
-        new_msg_history = chat_with_agent(instruction, model=self.code_model, msg_history=[], logging=safe_log)
+
+        retry_count = 0
+        while retry_count < self.max_retries:
+            safe_log(f"\n=== Attempt {retry_count + 1} of {self.max_retries} ===")
+            
+            # Reset to base commit before each attempt
+            if retry_count > 0:
+                reset_to_commit(self.git_tempdir, self.base_commit)
+            
+            # Add retry context to instruction if this is a retry attempt
+            instruction = base_instruction
+            if retry_count > 0:
+                instruction += f"""\nNOTE: Previous attempt(s) failed because they either produced empty patches or only modified test files. 
+Please ensure your solution includes changes to the main source code files, not just test files."""
+
+            # Run the agent
+            new_msg_history = chat_with_agent(instruction, model=self.code_model, msg_history=[], logging=safe_log)
+            
+            # Check the patch
+            patch = self.get_current_edits()
+            is_valid, reason = is_patch_valid(patch)
+            
+            if is_valid:
+                safe_log(f"Valid patch generated: {reason}")
+                break
+            else:
+                safe_log(f"Invalid patch: {reason}")
+                if retry_count < self.max_retries - 1:
+                    safe_log("Retrying with a new attempt...")
+                else:
+                    safe_log("Maximum retries reached. Unable to generate a valid patch.")
+            
+            retry_count += 1
 
 def main():
     parser = argparse.ArgumentParser(description='Process repository with an agentic system.')
@@ -177,6 +248,7 @@ def main():
     parser.add_argument('--test_description', default=None, required=False, help='Description of how to test the repository')
     parser.add_argument('--self_improve', default=False, action='store_true', help='Whether to self-improve the repository or solving swe')
     parser.add_argument('--instance_id', default=None, help='Instance ID for SWE issue')
+    parser.add_argument('--max_retries', type=int, default=3, help='Maximum number of patch generation attempts')
     args = parser.parse_args()
 
     # Process the repository
@@ -188,6 +260,7 @@ def main():
         test_description=args.test_description,
         self_improve=args.self_improve,
         instance_id=args.instance_id,
+        max_retries=args.max_retries,
     )
 
     # Run the agentic system to try to solve the problem
@@ -200,4 +273,4 @@ def main():
         f.write(model_patch)
 
 if __name__ == "__main__":
-    main()
+    main()
\ No newline at end of file
diff --git a/tests/test_patch_validator.py b/tests/test_patch_validator.py
new file mode 100644
index 0000000..5689f7d
--- /dev/null
+++ b/tests/test_patch_validator.py
@@ -0,0 +1,77 @@
+import pytest
+from coding_agent import is_patch_valid
+
+def test_empty_patch():
+    # Test empty patch
+    is_valid, reason = is_patch_valid("")
+    assert not is_valid
+    assert reason == "Empty patch"
+
+    # Test whitespace-only patch
+    is_valid, reason = is_patch_valid("   \n   ")
+    assert not is_valid
+    assert reason == "Empty patch"
+
+def test_test_only_patch():
+    patch = """
+diff --git a/tests/test_edit_tool.py b/tests/test_edit_tool.py
+index abc123..def456 100644
+--- a/tests/test_edit_tool.py
++++ b/tests/test_edit_tool.py
+@@ -10,6 +10,8 @@ def test_something():
+     assert True
++    assert 1 == 1
+"""
+    is_valid, reason = is_patch_valid(patch)
+    assert not is_valid
+    assert reason == "Only test files were modified"
+
+def test_source_file_patch():
+    patch = """
+diff --git a/tools/edit.py b/tools/edit.py
+index abc123..def456 100644
+--- a/tools/edit.py
++++ b/tools/edit.py
+@@ -10,6 +10,8 @@ class Editor:
+     def edit(self):
+         pass
++        return True
+"""
+    is_valid, reason = is_patch_valid(patch)
+    assert is_valid
+    assert reason == "Valid patch with source file modifications"
+
+def test_mixed_files_patch():
+    patch = """
+diff --git a/tools/edit.py b/tools/edit.py
+index abc123..def456 100644
+--- a/tools/edit.py
++++ b/tools/edit.py
+@@ -10,6 +10,8 @@ class Editor:
+     def edit(self):
+         pass
++        return True
+
+diff --git a/tests/test_edit.py b/tests/test_edit.py
+index abc123..def456 100644
+--- a/tests/test_edit.py
++++ b/tests/test_edit.py
+@@ -10,6 +10,8 @@ def test_something():
+     assert True
++    assert 1 == 1
+"""
+    is_valid, reason = is_patch_valid(patch)
+    assert is_valid
+    assert reason == "Valid patch with source file modifications"
+
+def test_no_files_modified():
+    patch = """
+diff --git a/nonexistent.py b/nonexistent.py
+deleted file mode 100644
+index abc123..0000000
+--- a/nonexistent.py
++++ /dev/null
+"""
+    is_valid, reason = is_patch_valid(patch)
+    assert not is_valid
+    assert reason == "No files modified"
\ No newline at end of file
\end{lstlisting}
\begin{lstlisting}[style=diffstyle]
diff --git a/tools/edit.py b/tools/edit.py
index 59137ee..16ae521 100644
--- a/tools/edit.py
+++ b/tools/edit.py
@@ -1,16 +1,17 @@
 from pathlib import Path
 import subprocess
+from typing import Optional, List, Tuple, Union
 
 def tool_info():
     return {
         "name": "editor",
         "description": """Custom editing tool for viewing, creating, and editing files\n
 * State is persistent across command calls and discussions with the user.\n
-* If `path` is a file, `view` displays the entire file with line numbers. If `path` is a directory, `view` lists non-hidden files and directories up to 2 levels deep.\n
+* If `path` is a file, `view` displays the file with line numbers. With optional `view_range` [start, end], it displays only specified lines. Use -1 in `end` for all remaining lines.\n
+* If `path` is a directory, `view` lists non-hidden files and directories up to 2 levels deep.\n
 * The `create` command cannot be used if the specified `path` already exists as a file.\n
 * If a `command` generates a long output, it will be truncated and marked with `<response clipped>`.\n
-* The `edit` command overwrites the entire file with the provided `file_text`.\n
-* No partial/line-range edits or partial viewing are supported.""",
+* The `edit` command overwrites the entire file with the provided `file_text`.""",
         "input_schema": {
             "type": "object",
             "properties": {
@@ -26,6 +27,13 @@ def tool_info():
                 "file_text": {
                     "description": "Required parameter of `create` or `edit` command, containing the content for the entire file.",
                     "type": "string"
+                },
+                "view_range": {
+                    "description": "Optional parameter for `view` command. Array of [start_line, end_line] (1-based). Use -1 for end_line to read until end of file.",
+                    "type": "array",
+                    "items": {"type": "integer"},
+                    "minItems": 2,
+                    "maxItems": 2
                 }
             },
             "required": ["command", "path"]
@@ -89,6 +97,46 @@ def read_file(path: Path) -> str:
     except Exception as e:
         raise ValueError(f"Failed to read file: {e}")
 
+def read_file_range(path: Path, line_range: Optional[List[int]] = None) -> Tuple[str, int]:
+    """
+    Read and return file contents within specified line range.
+    Returns tuple of (content, start_line).
+    
+    Args:
+        path: Path object for the file
+        line_range: Optional [start, end] line numbers (1-based). Use -1 for end to read until EOF.
+    """
+    try:
+        if line_range is None:
+            return read_file(path), 1
+
+        start, end = line_range
+        if start < 1:
+            raise ValueError("Start line must be >= 1")
+        if end != -1 and end < start:
+            raise ValueError("End line must be >= start line or -1")
+
+        with path.open() as f:
+            # Skip lines before start
+            for _ in range(start - 1):
+                next(f, None)
+
+            lines = []
+            current_line = start
+            while True:
+                line = next(f, None)
+                if line is None:  # EOF
+                    break
+                if end != -1 and current_line > end:
+                    break
+                lines.append(line.rstrip('\n'))
+                current_line += 1
+
+        return '\n'.join(lines), start
+
+    except Exception as e:
+        raise ValueError(f"Failed to read file range: {e}")
+
 def write_file(path: Path, content: str):
     """Write (overwrite) entire file contents."""
     try:
@@ -96,9 +144,18 @@ def write_file(path: Path, content: str):
     except Exception as e:
         raise ValueError(f"Failed to write file: {e}")
 
-def view_path(path_obj: Path) -> str:
-    """View the entire file contents or directory listing."""
+def view_path(path_obj: Path, view_range: Optional[List[int]] = None) -> str:
+    """
+    View the file contents (optionally within a range) or directory listing.
+    
+    Args:
+        path_obj: Path object for the file or directory
+        view_range: Optional [start, end] line numbers for file viewing
+    """
     if path_obj.is_dir():
+        if view_range is not None:
+            raise ValueError("view_range is not supported for directory listings")
+        
         # For directories: list non-hidden files up to 2 levels deep
         try:
             result = subprocess.run(
@@ -115,14 +172,14 @@ def view_path(path_obj: Path) -> str:
         except Exception as e:
             raise ValueError(f"Failed to list directory: {e}")
 
-    # If it's a file, show the entire file with line numbers
-    content = read_file(path_obj)
-    return format_output(content, str(path_obj))
+    # If it's a file, show the file content (with optional line range)
+    content, start_line = read_file_range(path_obj, view_range)
+    return format_output(content, str(path_obj), start_line)
 
-def tool_function(command: str, path: str, file_text: str = None) -> str:
+def tool_function(command: str, path: str, file_text: str = None, view_range: Optional[List[int]] = None) -> str:
     """
     Main tool function that handles:
-      - 'view'  : View the entire file or directory listing
+      - 'view'  : View file or directory listing, optionally within line range for files
       - 'create': Create a new file with the given file_text
       - 'edit'  : Overwrite an existing file with file_text
     """
@@ -130,7 +187,7 @@ def tool_function(command: str, path: str, file_text: str = None) -> str:
         path_obj = validate_path(path, command)
 
         if command == "view":
-            return view_path(path_obj)
+            return view_path(path_obj, view_range)
 
         elif command == "create":
             if file_text is None:
@@ -152,4 +209,4 @@ def tool_function(command: str, path: str, file_text: str = None) -> str:
 
 if __name__ == "__main__":
     # Example usage
-    print(tool_function("view", "/home/ubuntu/xx/dgm/coding_agent.py"))
+    print(tool_function("view", "/home/ubuntu/xx/dgm/coding_agent.py"))
\ No newline at end of file
diff --git a/tests/test_tools/test_edit.py b/tests/test_tools/test_edit.py
new file mode 100644
index 0000000..04f535b
--- /dev/null
+++ b/tests/test_tools/test_edit.py
@@ -0,0 +1,54 @@
+import pytest
+from pathlib import Path
+from tools.edit import tool_function
+
+def test_view_line_range(tmp_path):
+    # Create a test file
+    test_file = tmp_path / "test.txt"
+    test_content = "line1\nline2\nline3\nline4\nline5\n"
+    test_file.write_text(test_content)
+
+    # Test viewing specific line range
+    result = tool_function("view", str(test_file), view_range=[2, 4])
+    assert "line2" in result
+    assert "line3" in result
+    assert "line4" in result
+    assert "line1" not in result
+    assert "line5" not in result
+    assert "     2\t" in result  # Correct line numbering
+
+    # Test viewing from start to middle
+    result = tool_function("view", str(test_file), view_range=[1, 3])
+    assert "line1" in result
+    assert "line2" in result
+    assert "line3" in result
+    assert "line4" not in result
+    assert "     1\t" in result
+
+    # Test viewing from middle to end with -1
+    result = tool_function("view", str(test_file), view_range=[3, -1])
+    assert "line1" not in result
+    assert "line2" not in result
+    assert "line3" in result
+    assert "line4" in result
+    assert "line5" in result
+    assert "     3\t" in result
+
+def test_view_range_validation(tmp_path):
+    # Create a test file
+    test_file = tmp_path / "test.txt"
+    test_content = "line1\nline2\nline3\n"
+    test_file.write_text(test_content)
+
+    # Test invalid start line
+    result = tool_function("view", str(test_file), view_range=[0, 2])
+    assert "Failed to read file range: Start line must be >= 1" in result
+
+    # Test invalid range (end < start)
+    result = tool_function("view", str(test_file), view_range=[2, 1])
+    assert "Failed to read file range: End line must be >= start line or -1" in result
+
+def test_view_range_with_directory(tmp_path):
+    # Test that view_range is rejected for directories
+    result = tool_function("view", str(tmp_path), view_range=[1, 10])
+    assert "Error: view_range is not supported for directory listings" in result
\ No newline at end of file
\end{lstlisting}

\begin{lstlisting}[style=diffstyle]
diff --git a/tools/edit.py b/tools/edit.py
index 16ae521..757f5c2 100644
--- a/tools/edit.py
+++ b/tools/edit.py
@@ -11,21 +11,21 @@ def tool_info():
 * If `path` is a directory, `view` lists non-hidden files and directories up to 2 levels deep.\n
 * The `create` command cannot be used if the specified `path` already exists as a file.\n
 * If a `command` generates a long output, it will be truncated and marked with `<response clipped>`.\n
-* The `edit` command overwrites the entire file with the provided `file_text`.""",
+* The `str_replace` command replaces a unique occurrence of old_str with new_str, failing if old_str is not found or appears multiple times.""",
         "input_schema": {
             "type": "object",
             "properties": {
                 "command": {
                     "type": "string",
-                    "enum": ["view", "create", "edit"],
-                    "description": "The command to run: `view`, `create`, or `edit`."
+                    "enum": ["view", "create", "str_replace"],
+                    "description": "The command to run: `view`, `create`, or `str_replace`."
                 },
                 "path": {
                     "description": "Absolute path to file or directory, e.g. `/repo/file.py` or `/repo`.",
                     "type": "string"
                 },
                 "file_text": {
-                    "description": "Required parameter of `create` or `edit` command, containing the content for the entire file.",
+                    "description": "Required parameter of `create` command, containing the content for the entire file.",
                     "type": "string"
                 },
                 "view_range": {
@@ -34,6 +34,14 @@ def tool_info():
                     "items": {"type": "integer"},
                     "minItems": 2,
                     "maxItems": 2
+                },
+                "old_str": {
+                    "description": "Required parameter of `str_replace` command, containing the exact text to find and replace.",
+                    "type": "string"
+                },
+                "new_str": {
+                    "description": "Required parameter of `str_replace` command, containing the new text to replace old_str with.",
+                    "type": "string"
                 }
             },
             "required": ["command", "path"]
@@ -51,7 +59,7 @@ def validate_path(path: str, command: str) -> Path:
     Validate the file path for each command:
       - 'view': path may be a file or directory; must exist.
       - 'create': path must not exist (for new file creation).
-      - 'edit': path must exist (for overwriting).
+      - 'str_replace': path must exist and be a file.
     """
     path_obj = Path(path)
 
@@ -69,7 +77,7 @@ def validate_path(path: str, command: str) -> Path:
         # Path must not exist
         if path_obj.exists():
             raise ValueError(f"Cannot create new file; {path} already exists.")
-    elif command == "edit":
+    elif command == "str_replace":
         # Path must exist and must be a file
         if not path_obj.exists():
             raise ValueError(f"The file {path} does not exist.")
@@ -144,6 +152,28 @@ def write_file(path: Path, content: str):
     except Exception as e:
         raise ValueError(f"Failed to write file: {e}")
 
+def str_replace_in_file(path: Path, old_str: str, new_str: str) -> str:
+    """
+    Replace an exact occurrence of old_str with new_str in the file.
+    Only performs the replacement if old_str occurs exactly once.
+    Returns a message indicating success or failure.
+    """
+    try:
+        content = read_file(path)
+        occurrences = content.count(old_str)
+        
+        if occurrences == 0:
+            return f"Error: Could not find the exact text to replace in {path}"
+        elif occurrences > 1:
+            return f"Error: Found multiple ({occurrences}) occurrences of the text in {path}. Must be unique."
+        else:
+            new_content = content.replace(old_str, new_str)
+            write_file(path, new_content)
+            return f"Successfully replaced text in {path}"
+            
+    except Exception as e:
+        return f"Error during string replacement: {e}"
+
 def view_path(path_obj: Path, view_range: Optional[List[int]] = None) -> str:
     """
     View the file contents (optionally within a range) or directory listing.
@@ -176,12 +206,13 @@ def view_path(path_obj: Path, view_range: Optional[List[int]] = None) -> str:
     content, start_line = read_file_range(path_obj, view_range)
     return format_output(content, str(path_obj), start_line)
 
-def tool_function(command: str, path: str, file_text: str = None, view_range: Optional[List[int]] = None) -> str:
+def tool_function(command: str, path: str, file_text: str = None, view_range: Optional[List[int]] = None,
+                 old_str: str = None, new_str: str = None) -> str:
     """
     Main tool function that handles:
-      - 'view'  : View file or directory listing, optionally within line range for files
-      - 'create': Create a new file with the given file_text
-      - 'edit'  : Overwrite an existing file with file_text
+      - 'view'       : View file or directory listing, optionally within line range for files
+      - 'create'     : Create a new file with the given file_text
+      - 'str_replace': Replace exact occurrence of old_str with new_str in the file
     """
     try:
         path_obj = validate_path(path, command)
@@ -195,11 +226,10 @@ def tool_function(command: str, path: str, file_text: str = None, view_range: Op
             write_file(path_obj, file_text)
             return f"File created successfully at: {path}"
 
-        elif command == "edit":
-            if file_text is None:
-                raise ValueError("Missing required `file_text` for 'edit' command.")
-            write_file(path_obj, file_text)
-            return f"File at {path} has been overwritten with new content."
+        elif command == "str_replace":
+            if old_str is None or new_str is None:
+                raise ValueError("Missing required `old_str` and/or `new_str` for 'str_replace' command.")
+            return str_replace_in_file(path_obj, old_str, new_str)
 
         else:
             raise ValueError(f"Unknown command: {command}")
diff --git a/tests/__init__.py b/tests/__init__.py
new file mode 100644
index 0000000..e69de29
diff --git a/tests/test_tools.py b/tests/test_tools.py
new file mode 100644
index 0000000..c7f242f
--- /dev/null
+++ b/tests/test_tools.py
@@ -0,0 +1,65 @@
+import pytest
+from pathlib import Path
+from tools.edit import tool_function
+
+# Test fixtures
+@pytest.fixture
+def temp_file(tmp_path):
+    file_path = tmp_path / "test.txt"
+    content = "line 1\nline 2\nline 3\n"
+    file_path.write_text(content)
+    return str(file_path)
+
+def test_str_replace_success(temp_file):
+    # Test successful replacement
+    result = tool_function(
+        command="str_replace",
+        path=temp_file,
+        old_str="line 2\n",
+        new_str="replaced line\n"
+    )
+    assert "Successfully replaced" in result
+    assert Path(temp_file).read_text() == "line 1\nreplaced line\nline 3\n"
+
+def test_str_replace_not_found(temp_file):
+    # Test when old_str is not found
+    result = tool_function(
+        command="str_replace",
+        path=temp_file,
+        old_str="nonexistent",
+        new_str="something"
+    )
+    assert "Could not find" in result
+    # Original file should be unchanged
+    assert Path(temp_file).read_text() == "line 1\nline 2\nline 3\n"
+
+def test_str_replace_multiple_occurrences(temp_file):
+    # First create a file with multiple occurrences
+    Path(temp_file).write_text("same\nsame\nsame\n")
+    result = tool_function(
+        command="str_replace",
+        path=temp_file,
+        old_str="same\n",
+        new_str="different\n"
+    )
+    assert "multiple" in result
+    # Original file should be unchanged
+    assert Path(temp_file).read_text() == "same\nsame\nsame\n"
+
+def test_str_replace_missing_params(temp_file):
+    # Test missing parameters
+    result = tool_function(
+        command="str_replace",
+        path=temp_file,
+    )
+    assert "Missing required" in result
+
+def test_str_replace_invalid_path():
+    # Test with non-existent file
+    result = tool_function(
+        command="str_replace",
+        path="/nonexistent/path",
+        old_str="old",
+        new_str="new"
+    )
+    assert "does not exist" in result
\ No newline at end of file
\end{lstlisting}
\begin{lstlisting}[style=diffstyle]
diff --git a/coding_agent.py b/coding_agent.py
index 6639abd..97f4b69 100644
--- a/coding_agent.py
+++ b/coding_agent.py
@@ -52,6 +52,10 @@ class SolutionAttempt:
     test_output: str  # Raw test output
     test_success: bool  # Whether tests passed
     test_stats: dict  # Test statistics (e.g., number of passed/failed tests)
+    error_messages: List[str] = None  # List of specific error messages
+    test_details: dict = None  # Detailed test information like specific test names and their status
+    execution_time: float = None  # Test execution time in seconds
+    attempt_number: int = None  # The attempt number in the sequence
 
 def get_thread_logger():
     """
@@ -150,12 +154,82 @@ class AgenticSystem:
         ]
         return new_msg_history
 
+    def extract_test_details(self, output: str) -> tuple[dict, List[str], dict]:
+        """Extract detailed test information from the output."""
+        error_messages = []
+        test_details = {}
+        stats = {"passed": 0, "failed": 0, "errors": 0, "total": 0, "skipped": 0}
+
+        # Split output into lines for analysis
+        lines = output.split("\n")
+        
+        # Language-specific parsing
+        if self.language == "python":
+            for line in lines:
+                if "FAILED" in line and "::" in line:
+                    test_name = line.split("::")[1].split()[0]
+                    test_details[test_name] = "FAILED"
+                    stats["failed"] += 1
+                elif "PASSED" in line and "::" in line:
+                    test_name = line.split("::")[1].split()[0]
+                    test_details[test_name] = "PASSED"
+                    stats["passed"] += 1
+                elif "ERROR" in line and "::" in line:
+                    test_name = line.split("::")[1].split()[0]
+                    test_details[test_name] = "ERROR"
+                    stats["errors"] += 1
+                    # Extract error message
+                    if lines.index(line) + 1 < len(lines):
+                        error_messages.append(lines[lines.index(line) + 1])
+        
+        elif self.language in ["javascript", "node"]:
+            current_test = None
+            for line in lines:
+                if line.startswith('checkmark'):
+                    test_name = line.replace('checkmark,', '').strip()
+                    test_details[test_name] = "PASSED"
+                    stats["passed"] += 1
+                elif line.startswith('x'):
+                    test_name = line.replace('x', '').strip()
+                    test_details[test_name] = "FAILED"
+                    stats["failed"] += 1
+                    current_test = test_name
+                elif current_test and ('Error:' in line or 'AssertionError:' in line):
+                    error_messages.append(f"{current_test}: {line.strip()}")
+
+        elif self.language == "rust":
+            for line in lines:
+                if "test" in line and "... ok" in line:
+                    test_name = line.split("test")[1].split("...")[0].strip()
+                    test_details[test_name] = "PASSED"
+                    stats["passed"] += 1
+                elif "test" in line and "... FAILED" in line:
+                    test_name = line.split("test")[1].split("...")[0].strip()
+                    test_details[test_name] = "FAILED"
+                    stats["failed"] += 1
+                elif "---- " in line and " stdout ----" in line:
+                    test_name = line.split("----")[1].split("stdout")[0].strip()
+                    if test_name in test_details and test_details[test_name] == "FAILED":
+                        error_messages.append(f"{test_name}: {next((l for l in lines[lines.index(line)+1:] if l.strip()), '')}")
+
+        # Generic counting for other languages or as fallback
+        if not any(stats.values()):
+            stats["passed"] = output.count("PASS") + output.count("ok")
+            stats["failed"] = output.count("FAIL") + output.count("not ok")
+            stats["errors"] = output.count("ERROR") + output.count("panic:")
+            
+        stats["total"] = stats["passed"] + stats["failed"] + stats["errors"]
+        
+        return stats, error_messages, test_details
+
     def run_tests(self) -> tuple[bool, str, dict]:
         """Run tests and return success status, output, and test statistics."""
+        import time
+        
         success = False
         output = ""
-        stats = {"passed": 0, "failed": 0, "errors": 0, "total": 0}
-
+        start_time = time.time()
+        
         try:
             for command in TEST_COMMANDS.get(self.language, []):
                 proc = subprocess.run(
@@ -169,34 +243,97 @@ class AgenticSystem:
                 success = proc.returncode == 0
                 if not success:
                     break
-
-            # Try to extract test statistics from output
-            # This is a simple example; you might want to add more sophisticated parsing
-            stats["passed"] = output.count("PASS") + output.count("ok")
-            stats["failed"] = output.count("FAIL") + output.count("not ok")
-            stats["errors"] = output.count("ERROR") + output.count("panic:")
-            stats["total"] = stats["passed"] + stats["failed"] + stats["errors"]
+            
+            # Extract detailed test information
+            stats, error_messages, test_details = self.extract_test_details(output)
+            stats["execution_time"] = time.time() - start_time
+            
+            # Enhance stats with extracted information
+            stats["error_messages"] = error_messages
+            stats["test_details"] = test_details
             
         except Exception as e:
             output = f"Error running tests: {str(e)}"
             success = False
+            stats = {
+                "passed": 0, "failed": 0, "errors": 1, "total": 1,
+                "execution_time": time.time() - start_time,
+                "error_messages": [str(e)],
+                "test_details": {}
+            }
 
         return success, output, stats
 
     def analyze_test_results(self, attempts: List[SolutionAttempt]) -> str:
-        """Analyze test results and create a summary for the agent."""
+        """Analyze test results and create a detailed summary for the agent."""
         summary = "# Test Results Analysis\n\n"
         
+        # Overall progress tracking
+        if len(attempts) > 1:
+            summary += "## Progress Overview\n"
+            first_attempt = attempts[0].test_stats
+            last_attempt = attempts[-1].test_stats
+            
+            progress = {
+                "passed": last_attempt["passed"] - first_attempt["passed"],
+                "failed": first_attempt["failed"] - last_attempt["failed"],
+                "errors": first_attempt["errors"] - last_attempt["errors"]
+            }
+            
+            summary += "Progress since first attempt:\n"
+            summary += f"- Additional passing tests: {progress['passed']}\n"
+            summary += f"- Reduced failures: {progress['failed']}\n"
+            summary += f"- Reduced errors: {progress['errors']}\n\n"
+        
+        # Detailed attempt analysis
         for i, attempt in enumerate(attempts, 1):
             summary += f"## Attempt {i}\n"
             summary += f"Test Success: {attempt.test_success}\n"
-            summary += f"Test Stats: {json.dumps(attempt.test_stats, indent=2)}\n"
-            summary += "Key test output:\n```\n"
-            # Extract relevant parts of test output (e.g., error messages)
-            key_output = "\n".join(line for line in attempt.test_output.split("\n") 
-                                 if "FAIL" in line or "ERROR" in line or "PASS" in line)
-            summary += f"{key_output}\n```\n\n"
-
+            summary += f"Execution Time: {attempt.test_stats.get('execution_time', 'N/A'):.2f}s\n"
+            
+            # Test statistics
+            stats = attempt.test_stats
+            total = stats.get("total", 0) or 1  # Avoid division by zero
+            pass_rate = (stats.get("passed", 0) / total) * 100
+            
+            summary += f"Pass Rate: {pass_rate:.1f}% ({stats.get('passed', 0)}/{total})\n"
+            summary += "Test Statistics:\n"
+            summary += f"- Passed: {stats.get('passed', 0)}\n"
+            summary += f"- Failed: {stats.get('failed', 0)}\n"
+            summary += f"- Errors: {stats.get('errors', 0)}\n"
+            summary += f"- Total: {total}\n\n"
+            
+            # Error messages
+            if stats.get("error_messages"):
+                summary += "Error Messages:\n```\n"
+                for error in stats["error_messages"][:5]:  # Limit to top 5 errors
+                    summary += f"{error}\n"
+                if len(stats["error_messages"]) > 5:
+                    summary += f"... and {len(stats['error_messages']) - 5} more errors\n"
+                summary += "```\n\n"
+            
+            # Test details
+            if stats.get("test_details"):
+                summary += "Individual Test Results:\n```\n"
+                for test_name, result in stats["test_details"].items():
+                    summary += f"{result}: {test_name}\n"
+                summary += "```\n\n"
+
+        # Recommendations for next attempt
+        if not attempts[-1].test_success:
+            summary += "## Recommendations for Next Attempt\n"
+            last_stats = attempts[-1].test_stats
+            
+            if last_stats.get("errors", 0) > 0:
+                summary += "- Focus on resolving runtime errors first\n"
+            if last_stats.get("failed", 0) > 0:
+                summary += "- Address failing test cases\n"
+            if len(attempts) > 1 and not attempts[-1].test_success:
+                # Compare with previous attempt
+                prev_stats = attempts[-2].test_stats
+                if last_stats.get("passed", 0) < prev_stats.get("passed", 0):
+                    summary += "- Recent changes caused regressions. Consider reverting some changes\n"
+            
         return summary
 
     def forward(self):
@@ -238,20 +375,36 @@ Your task is to make changes to the files in the {self.git_tempdir} directory to
             # Run tests and collect results 
             test_success, test_output, test_stats = self.run_tests()
             
-            # Create and store attempt
+            # Create and store attempt with enhanced information
             attempt = SolutionAttempt(
                 patch=current_patch,
                 test_output=test_output,
                 test_success=test_success,
-                test_stats=test_stats
+                test_stats=test_stats,
+                error_messages=test_stats.get('error_messages', []),
+                test_details=test_stats.get('test_details', {}),
+                execution_time=test_stats.get('execution_time', None),
+                attempt_number=attempt_num + 1
             )
             attempts.append(attempt)
             
-            # Update best attempt if this one is better
-            if test_success and (best_attempt is None or 
-                               attempt.test_stats["passed"] > best_attempt.test_stats["passed"]):
+            # Update best attempt based on multiple criteria
+            if test_success and (
+                best_attempt is None or
+                (attempt.test_stats["passed"] > best_attempt.test_stats["passed"]) or
+                (attempt.test_stats["passed"] == best_attempt.test_stats["passed"] and
+                 len(attempt.error_messages or []) < len(best_attempt.error_messages or []))
+            ):
                 best_attempt = attempt
             
+            # Log detailed attempt information
+            safe_log(f"\n=== Attempt {attempt_num + 1} Summary ===")
+            safe_log(f"Test Success: {test_success}")
+            safe_log(f"Tests Passed: {test_stats.get('passed', 0)}")
+            safe_log(f"Tests Failed: {test_stats.get('failed', 0)}")
+            safe_log(f"Errors: {test_stats.get('errors', 0)}")
+            safe_log(f"Execution Time: {test_stats.get('execution_time', 'N/A'):.2f}s")
+            
             # If tests pass perfectly, we can stop
             if test_success and attempt.test_stats["failed"] == 0 and attempt.test_stats["errors"] == 0:
                 break
\end{lstlisting}
\begin{lstlisting}[style=diffstyle]
diff --git a/coding_agent.py b/coding_agent.py
index 9a2cc2f..3f1bc1d 100644
--- a/coding_agent.py
+++ b/coding_agent.py
@@ -111,6 +111,7 @@ class AgenticSystem:
             self_improve=False,
             instance_id=None,
             max_retries=3,
+            num_candidates=3,
         ):
         self.problem_statement = problem_statement
         self.git_tempdir = git_tempdir
@@ -121,6 +122,7 @@ class AgenticSystem:
         self.instance_id = instance_id if not self_improve else 'dgm'
         self.code_model = CLAUDE_MODEL
         self.max_retries = max_retries
+        self.num_candidates = num_candidates
 
         # Initialize logger and store it in thread-local storage
         self.logger = setup_logger(chat_history_file)
@@ -190,7 +192,7 @@ Your task is to run the regression tests in the {self.git_tempdir} directory to
 
     def forward(self):
         """
-        The forward function for the AgenticSystem.
+        The forward function for the AgenticSystem that generates and evaluates multiple candidate patches.
         """
         base_instruction = f"""I have uploaded a Python code repository in the directory {self.git_tempdir}. Help solve the following problem.
 
@@ -205,10 +207,18 @@ Your task is to run the regression tests in the {self.git_tempdir} directory to
 Your task is to make changes to the files in the {self.git_tempdir} directory to address the <problem_description>. I have already taken care of the required dependencies.
 """
 
+        # Get regression tests summary once at the start
+        regression_tests_summary = self.get_regression_tests()
+
+        # Lists to store candidates
+        valid_patches = []
+        valid_reports = []
+
         retry_count = 0
-        while retry_count < self.max_retries:
+        while retry_count < self.max_retries and len(valid_patches) < self.num_candidates:
             safe_log(f"\n=== Attempt {retry_count + 1} of {self.max_retries} ===")
-            
+            safe_log(f"Valid solutions so far: {len(valid_patches)} of {self.num_candidates} desired")
+
             # Reset to base commit before each attempt
             if retry_count > 0:
                 reset_to_commit(self.git_tempdir, self.base_commit)
@@ -216,8 +226,8 @@ Your task is to make changes to the files in the {self.git_tempdir} directory to
             # Add retry context to instruction if this is a retry attempt
             instruction = base_instruction
             if retry_count > 0:
-                instruction += f"""\nNOTE: Previous attempt(s) failed because they either produced empty patches or only modified test files. 
-Please ensure your solution includes changes to the main source code files, not just test files."""
+                instruction += f"""\nNOTE: Previous attempt(s) did not produce enough valid solutions. 
+Please provide a different approach to solve the problem. Your solution must include changes to the main source code files, not just test files."""
 
             # Run the agent
             new_msg_history = chat_with_agent(instruction, model=self.code_model, msg_history=[], logging=safe_log)
@@ -228,16 +238,45 @@ Please ensure your solution includes changes to the main source code files, not
             
             if is_valid:
                 safe_log(f"Valid patch generated: {reason}")
-                break
+                # Run regression tests for this candidate
+                test_report = self.run_regression_tests(regression_tests_summary)
+                test_score = get_report_score(test_report)
+                safe_log(f"Test score: {test_score}")
+
+                valid_patches.append(patch)
+                valid_reports.append(test_report)
+
+                if len(valid_patches) >= self.num_candidates:
+                    break
             else:
                 safe_log(f"Invalid patch: {reason}")
-                if retry_count < self.max_retries - 1:
-                    safe_log("Retrying with a new attempt...")
-                else:
-                    safe_log("Maximum retries reached. Unable to generate a valid patch.")
             
             retry_count += 1
 
+        if not valid_patches:
+            safe_log("Failed to generate any valid patches.")
+            return
+
+        # Use score_tie_breaker to select the best patch
+        safe_log(f"\n=== Selecting Best Solution from {len(valid_patches)} Candidates ===")
+        best_index = score_tie_breaker(
+            self.problem_statement,
+            valid_patches,
+            valid_reports,
+            logging=safe_log
+        )
+
+        # Reset to base and apply the best patch
+        reset_to_commit(self.git_tempdir, self.base_commit)
+        best_patch = valid_patches[best_index]
+        safe_log(f"\n=== Applying Best Solution (Candidate {best_index + 1}) ===")
+        apply_patch(self.git_tempdir, best_patch)
+
+        # Final validation of the selected patch
+        final_test_report = self.run_regression_tests(regression_tests_summary)
+        final_score = get_report_score(final_test_report)
+        safe_log(f"Final solution test score: {final_score}")
+
 def main():
     parser = argparse.ArgumentParser(description='Process repository with an agentic system.')
     parser.add_argument('--problem_statement', required=True, help='The problem statement to process')
@@ -249,6 +288,7 @@ def main():
     parser.add_argument('--self_improve', default=False, action='store_true', help='Whether to self-improve the repository or solving swe')
     parser.add_argument('--instance_id', default=None, help='Instance ID for SWE issue')
     parser.add_argument('--max_retries', type=int, default=3, help='Maximum number of patch generation attempts')
+    parser.add_argument('--num_candidates', type=int, default=3, help='Number of candidate solutions to generate')
     args = parser.parse_args()
 
     # Process the repository
@@ -261,6 +301,7 @@ def main():
         self_improve=args.self_improve,
         instance_id=args.instance_id,
         max_retries=args.max_retries,
+        num_candidates=args.num_candidates,
     )
 
     # Run the agentic system to try to solve the problem
\end{lstlisting}
\begin{lstlisting}[style=diffstyle]
diff --git a/coding_agent.py b/coding_agent.py
index 3f1bc1d..588938d 100644
--- a/coding_agent.py
+++ b/coding_agent.py
@@ -193,42 +193,59 @@ Your task is to run the regression tests in the {self.git_tempdir} directory to
     def forward(self):
         """
         The forward function for the AgenticSystem that generates and evaluates multiple candidate patches.
+        This version maintains history of prior valid patches and test results, only using the tie-breaker
+        when necessary.
         """
-        base_instruction = f"""I have uploaded a Python code repository in the directory {self.git_tempdir}. Help solve the following problem.
-
-<problem_description>
-{self.problem_statement}
-</problem_description>
-
-<test_description>
-{self.test_description}
-</test_description>
-
-Your task is to make changes to the files in the {self.git_tempdir} directory to address the <problem_description>. I have already taken care of the required dependencies.
-"""
-
-        # Get regression tests summary once at the start
         regression_tests_summary = self.get_regression_tests()
 
-        # Lists to store candidates
+        # Lists to store all valid patches and their information
         valid_patches = []
         valid_reports = []
+        valid_scores = []
+        best_score = 0
+        best_patches_indices = []  # Indices of patches that share the best score
 
         retry_count = 0
         while retry_count < self.max_retries and len(valid_patches) < self.num_candidates:
             safe_log(f"\n=== Attempt {retry_count + 1} of {self.max_retries} ===")
             safe_log(f"Valid solutions so far: {len(valid_patches)} of {self.num_candidates} desired")
+            safe_log(f"Current best test score: {best_score}")
 
             # Reset to base commit before each attempt
             if retry_count > 0:
                 reset_to_commit(self.git_tempdir, self.base_commit)
             
-            # Add retry context to instruction if this is a retry attempt
-            instruction = base_instruction
-            if retry_count > 0:
-                instruction += f"""\nNOTE: Previous attempt(s) did not produce enough valid solutions. 
+            # Construct instruction with previous best solutions if available
+            instruction = f"""I have uploaded a Python code repository in the directory {self.git_tempdir}. Help solve the following problem.
+
+<problem_description>
+{self.problem_statement}
+</problem_description>
+
+<test_description>
+{self.test_description}
+</test_description>"""
+
+            # Add previous solutions context if available
+            if valid_patches and retry_count > 0:
+                previous_solutions = []
+                for i, (patch, report, score) in enumerate(zip(valid_patches, valid_reports, valid_scores)):
+                    previous_solutions.append(f"""
+Previous Solution {i+1}:
+<code_changes>
+{patch}
+</code_changes>
+Test Score: {score}
+Test Report: {report}
+""")
+                instruction += "\n\nPrevious solution attempts:\n" + "\n".join(previous_solutions)
+                instruction += "\nPlease provide a new solution that addresses any limitations in the previous attempts or explores a different approach."
+            elif retry_count > 0:
+                instruction += """\nNOTE: Previous attempt(s) did not produce enough valid solutions. 
 Please provide a different approach to solve the problem. Your solution must include changes to the main source code files, not just test files."""
 
+            instruction += f"\n\nYour task is to make changes to the files in the {self.git_tempdir} directory to address the <problem_description>. I have already taken care of the required dependencies."
+
             # Run the agent
             new_msg_history = chat_with_agent(instruction, model=self.code_model, msg_history=[], logging=safe_log)
             
@@ -245,6 +262,14 @@ Please provide a different approach to solve the problem. Your solution must inc
 
                 valid_patches.append(patch)
                 valid_reports.append(test_report)
+                valid_scores.append(test_score)
+
+                # Update best score and indices
+                if test_score > best_score:
+                    best_score = test_score
+                    best_patches_indices = [len(valid_patches) - 1]
+                elif test_score == best_score:
+                    best_patches_indices.append(len(valid_patches) - 1)
 
                 if len(valid_patches) >= self.num_candidates:
                     break
@@ -257,25 +282,30 @@ Please provide a different approach to solve the problem. Your solution must inc
             safe_log("Failed to generate any valid patches.")
             return
 
-        # Use score_tie_breaker to select the best patch
+        # Only use tie-breaker if we have multiple patches with the best score
         safe_log(f"\n=== Selecting Best Solution from {len(valid_patches)} Candidates ===")
-        best_index = score_tie_breaker(
-            self.problem_statement,
-            valid_patches,
-            valid_reports,
-            logging=safe_log
-        )
+        if len(best_patches_indices) > 1:
+            safe_log(f"Multiple solutions ({len(best_patches_indices)}) tied for best score {best_score}. Using tie-breaker.")
+            best_index = score_tie_breaker(
+                self.problem_statement,
+                [valid_patches[i] for i in best_patches_indices],
+                [valid_reports[i] for i in best_patches_indices],
+                logging=safe_log
+            )
+            best_index = best_patches_indices[best_index]
+        else:
+            best_index = best_patches_indices[0]
 
         # Reset to base and apply the best patch
         reset_to_commit(self.git_tempdir, self.base_commit)
         best_patch = valid_patches[best_index]
-        safe_log(f"\n=== Applying Best Solution (Candidate {best_index + 1}) ===")
+        safe_log(f"\n=== Applying Best Solution (Candidate {best_index + 1}) with score {valid_scores[best_index]} ===")
         apply_patch(self.git_tempdir, best_patch)
 
         # Final validation of the selected patch
         final_test_report = self.run_regression_tests(regression_tests_summary)
         final_score = get_report_score(final_test_report)
-        safe_log(f"Final solution test score: {final_score}")
+        safe_log(f"Final validation test score: {final_score}")
 
 def main():
     parser = argparse.ArgumentParser(description='Process repository with an agentic system.')
\end{lstlisting}

\subsection{DGM on Polyglot}
\label{app:best-agent-dgm-polyglot}

Diff patches contributing to the best agent discovered by the DGM on Polyglot:
\begin{lstlisting}[style=diffstyle]
diff --git a/coding_agent.py b/coding_agent.py
index 2cd395a..9a2cc2f 100644
--- a/coding_agent.py
+++ b/coding_agent.py
@@ -4,6 +4,7 @@ import logging
 from logging.handlers import RotatingFileHandler
 import os
 import threading
+import re
 
 from llm_withtools import CLAUDE_MODEL, OPENAI_MODEL, chat_with_agent
 from utils.eval_utils import get_report_score, msg_history_to_report, score_tie_breaker
@@ -63,6 +64,42 @@ def safe_log(message, level=logging.INFO):
     else:
         print(f"Warning: No logger found for thread {threading.get_ident()}")
 
+def is_patch_valid(patch_str):
+    """
+    Parse the patch to check if any non-test source files are modified.
+    Returns (bool, str) tuple: (is_valid, reason)
+    """
+    if not patch_str or patch_str.isspace():
+        return False, "Empty patch"
+    
+    # Parse the patch to find modified files
+    modified_files = []
+    diff_header_pattern = re.compile(r'^\+\+\+ b/(.+)$', re.MULTILINE)
+    for match in diff_header_pattern.finditer(patch_str):
+        filepath = match.group(1)
+        if filepath != '/dev/null':  # Skip deleted files
+            modified_files.append(filepath)
+    
+    if not modified_files:
+        return False, "No files modified"
+        
+    # Check if any non-test files are modified
+    test_patterns = (
+        lambda f: f.startswith('tests/'),
+        lambda f: f.startswith('test_'),
+        lambda f: f.endswith('_test.py')
+    )
+    
+    source_files = [
+        f for f in modified_files 
+        if not any(pattern(f) for pattern in test_patterns)
+    ]
+    
+    if not source_files:
+        return False, "Only test files were modified"
+    
+    return True, "Valid patch with source file modifications"
+
 class AgenticSystem:
     def __init__(
             self,
@@ -73,6 +110,7 @@ class AgenticSystem:
             test_description=None,
             self_improve=False,
             instance_id=None,
+            max_retries=3,
         ):
         self.problem_statement = problem_statement
         self.git_tempdir = git_tempdir
@@ -82,6 +120,7 @@ class AgenticSystem:
         self.self_improve = self_improve
         self.instance_id = instance_id if not self_improve else 'dgm'
         self.code_model = CLAUDE_MODEL
+        self.max_retries = max_retries
 
         # Initialize logger and store it in thread-local storage
         self.logger = setup_logger(chat_history_file)
@@ -153,7 +192,7 @@ Your task is to run the regression tests in the {self.git_tempdir} directory to
         """
         The forward function for the AgenticSystem.
         """
-        instruction = f"""I have uploaded a Python code repository in the directory {self.git_tempdir}. Help solve the following problem.
+        base_instruction = f"""I have uploaded a Python code repository in the directory {self.git_tempdir}. Help solve the following problem.
 
 <problem_description>
 {self.problem_statement}
@@ -165,7 +204,39 @@ Your task is to run the regression tests in the {self.git_tempdir} directory to
 
 Your task is to make changes to the files in the {self.git_tempdir} directory to address the <problem_description>. I have already taken care of the required dependencies.
 """
-        new_msg_history = chat_with_agent(instruction, model=self.code_model, msg_history=[], logging=safe_log)
+
+        retry_count = 0
+        while retry_count < self.max_retries:
+            safe_log(f"\n=== Attempt {retry_count + 1} of {self.max_retries} ===")
+            
+            # Reset to base commit before each attempt
+            if retry_count > 0:
+                reset_to_commit(self.git_tempdir, self.base_commit)
+            
+            # Add retry context to instruction if this is a retry attempt
+            instruction = base_instruction
+            if retry_count > 0:
+                instruction += f"""\nNOTE: Previous attempt(s) failed because they either produced empty patches or only modified test files. 
+Please ensure your solution includes changes to the main source code files, not just test files."""
+
+            # Run the agent
+            new_msg_history = chat_with_agent(instruction, model=self.code_model, msg_history=[], logging=safe_log)
+            
+            # Check the patch
+            patch = self.get_current_edits()
+            is_valid, reason = is_patch_valid(patch)
+            
+            if is_valid:
+                safe_log(f"Valid patch generated: {reason}")
+                break
+            else:
+                safe_log(f"Invalid patch: {reason}")
+                if retry_count < self.max_retries - 1:
+                    safe_log("Retrying with a new attempt...")
+                else:
+                    safe_log("Maximum retries reached. Unable to generate a valid patch.")
+            
+            retry_count += 1
 
 def main():
     parser = argparse.ArgumentParser(description='Process repository with an agentic system.')
@@ -177,6 +248,7 @@ def main():
     parser.add_argument('--test_description', default=None, required=False, help='Description of how to test the repository')
     parser.add_argument('--self_improve', default=False, action='store_true', help='Whether to self-improve the repository or solving swe')
     parser.add_argument('--instance_id', default=None, help='Instance ID for SWE issue')
+    parser.add_argument('--max_retries', type=int, default=3, help='Maximum number of patch generation attempts')
     args = parser.parse_args()
 
     # Process the repository
@@ -188,6 +260,7 @@ def main():
         test_description=args.test_description,
         self_improve=args.self_improve,
         instance_id=args.instance_id,
+        max_retries=args.max_retries,
     )
 
     # Run the agentic system to try to solve the problem
@@ -200,4 +273,4 @@ def main():
         f.write(model_patch)
 
 if __name__ == "__main__":
-    main()
+    main()
\ No newline at end of file
diff --git a/tests/test_patch_validator.py b/tests/test_patch_validator.py
new file mode 100644
index 0000000..5689f7d
--- /dev/null
+++ b/tests/test_patch_validator.py
@@ -0,0 +1,77 @@
+import pytest
+from coding_agent import is_patch_valid
+
+def test_empty_patch():
+    # Test empty patch
+    is_valid, reason = is_patch_valid("")
+    assert not is_valid
+    assert reason == "Empty patch"
+
+    # Test whitespace-only patch
+    is_valid, reason = is_patch_valid("   \n   ")
+    assert not is_valid
+    assert reason == "Empty patch"
+
+def test_test_only_patch():
+    patch = """
+diff --git a/tests/test_edit_tool.py b/tests/test_edit_tool.py
+index abc123..def456 100644
+--- a/tests/test_edit_tool.py
++++ b/tests/test_edit_tool.py
+@@ -10,6 +10,8 @@ def test_something():
+     assert True
++    assert 1 == 1
+"""
+    is_valid, reason = is_patch_valid(patch)
+    assert not is_valid
+    assert reason == "Only test files were modified"
+
+def test_source_file_patch():
+    patch = """
+diff --git a/tools/edit.py b/tools/edit.py
+index abc123..def456 100644
+--- a/tools/edit.py
++++ b/tools/edit.py
+@@ -10,6 +10,8 @@ class Editor:
+     def edit(self):
+         pass
++        return True
+"""
+    is_valid, reason = is_patch_valid(patch)
+    assert is_valid
+    assert reason == "Valid patch with source file modifications"
+
+def test_mixed_files_patch():
+    patch = """
+diff --git a/tools/edit.py b/tools/edit.py
+index abc123..def456 100644
+--- a/tools/edit.py
++++ b/tools/edit.py
+@@ -10,6 +10,8 @@ class Editor:
+     def edit(self):
+         pass
++        return True
+
+diff --git a/tests/test_edit.py b/tests/test_edit.py
+index abc123..def456 100644
+--- a/tests/test_edit.py
++++ b/tests/test_edit.py
+@@ -10,6 +10,8 @@ def test_something():
+     assert True
++    assert 1 == 1
+"""
+    is_valid, reason = is_patch_valid(patch)
+    assert is_valid
+    assert reason == "Valid patch with source file modifications"
+
+def test_no_files_modified():
+    patch = """
+diff --git a/nonexistent.py b/nonexistent.py
+deleted file mode 100644
+index abc123..0000000
+--- a/nonexistent.py
++++ /dev/null
+"""
+    is_valid, reason = is_patch_valid(patch)
+    assert not is_valid
+    assert reason == "No files modified"
\ No newline at end of file
\end{lstlisting}
\begin{lstlisting}[style=diffstyle]
diff --git a/tools/edit.py b/tools/edit.py
index 59137ee..16ae521 100644
--- a/tools/edit.py
+++ b/tools/edit.py
@@ -1,16 +1,17 @@
 from pathlib import Path
 import subprocess
+from typing import Optional, List, Tuple, Union
 
 def tool_info():
     return {
         "name": "editor",
         "description": """Custom editing tool for viewing, creating, and editing files\n
 * State is persistent across command calls and discussions with the user.\n
-* If `path` is a file, `view` displays the entire file with line numbers. If `path` is a directory, `view` lists non-hidden files and directories up to 2 levels deep.\n
+* If `path` is a file, `view` displays the file with line numbers. With optional `view_range` [start, end], it displays only specified lines. Use -1 in `end` for all remaining lines.\n
+* If `path` is a directory, `view` lists non-hidden files and directories up to 2 levels deep.\n
 * The `create` command cannot be used if the specified `path` already exists as a file.\n
 * If a `command` generates a long output, it will be truncated and marked with `<response clipped>`.\n
-* The `edit` command overwrites the entire file with the provided `file_text`.\n
-* No partial/line-range edits or partial viewing are supported.""",
+* The `edit` command overwrites the entire file with the provided `file_text`.""",
         "input_schema": {
             "type": "object",
             "properties": {
@@ -26,6 +27,13 @@ def tool_info():
                 "file_text": {
                     "description": "Required parameter of `create` or `edit` command, containing the content for the entire file.",
                     "type": "string"
+                },
+                "view_range": {
+                    "description": "Optional parameter for `view` command. Array of [start_line, end_line] (1-based). Use -1 for end_line to read until end of file.",
+                    "type": "array",
+                    "items": {"type": "integer"},
+                    "minItems": 2,
+                    "maxItems": 2
                 }
             },
             "required": ["command", "path"]
@@ -89,6 +97,46 @@ def read_file(path: Path) -> str:
     except Exception as e:
         raise ValueError(f"Failed to read file: {e}")
 
+def read_file_range(path: Path, line_range: Optional[List[int]] = None) -> Tuple[str, int]:
+    """
+    Read and return file contents within specified line range.
+    Returns tuple of (content, start_line).
+    
+    Args:
+        path: Path object for the file
+        line_range: Optional [start, end] line numbers (1-based). Use -1 for end to read until EOF.
+    """
+    try:
+        if line_range is None:
+            return read_file(path), 1
+
+        start, end = line_range
+        if start < 1:
+            raise ValueError("Start line must be >= 1")
+        if end != -1 and end < start:
+            raise ValueError("End line must be >= start line or -1")
+
+        with path.open() as f:
+            # Skip lines before start
+            for _ in range(start - 1):
+                next(f, None)
+
+            lines = []
+            current_line = start
+            while True:
+                line = next(f, None)
+                if line is None:  # EOF
+                    break
+                if end != -1 and current_line > end:
+                    break
+                lines.append(line.rstrip('\n'))
+                current_line += 1
+
+        return '\n'.join(lines), start
+
+    except Exception as e:
+        raise ValueError(f"Failed to read file range: {e}")
+
 def write_file(path: Path, content: str):
     """Write (overwrite) entire file contents."""
     try:
@@ -96,9 +144,18 @@ def write_file(path: Path, content: str):
     except Exception as e:
         raise ValueError(f"Failed to write file: {e}")
 
-def view_path(path_obj: Path) -> str:
-    """View the entire file contents or directory listing."""
+def view_path(path_obj: Path, view_range: Optional[List[int]] = None) -> str:
+    """
+    View the file contents (optionally within a range) or directory listing.
+    
+    Args:
+        path_obj: Path object for the file or directory
+        view_range: Optional [start, end] line numbers for file viewing
+    """
     if path_obj.is_dir():
+        if view_range is not None:
+            raise ValueError("view_range is not supported for directory listings")
+        
         # For directories: list non-hidden files up to 2 levels deep
         try:
             result = subprocess.run(
@@ -115,14 +172,14 @@ def view_path(path_obj: Path) -> str:
         except Exception as e:
             raise ValueError(f"Failed to list directory: {e}")
 
-    # If it's a file, show the entire file with line numbers
-    content = read_file(path_obj)
-    return format_output(content, str(path_obj))
+    # If it's a file, show the file content (with optional line range)
+    content, start_line = read_file_range(path_obj, view_range)
+    return format_output(content, str(path_obj), start_line)
 
-def tool_function(command: str, path: str, file_text: str = None) -> str:
+def tool_function(command: str, path: str, file_text: str = None, view_range: Optional[List[int]] = None) -> str:
     """
     Main tool function that handles:
-      - 'view'  : View the entire file or directory listing
+      - 'view'  : View file or directory listing, optionally within line range for files
       - 'create': Create a new file with the given file_text
       - 'edit'  : Overwrite an existing file with file_text
     """
@@ -130,7 +187,7 @@ def tool_function(command: str, path: str, file_text: str = None) -> str:
         path_obj = validate_path(path, command)
 
         if command == "view":
-            return view_path(path_obj)
+            return view_path(path_obj, view_range)
 
         elif command == "create":
             if file_text is None:
@@ -152,4 +209,4 @@ def tool_function(command: str, path: str, file_text: str = None) -> str:
 
 if __name__ == "__main__":
     # Example usage
-    print(tool_function("view", "/home/ubuntu/xx/dgm/coding_agent.py"))
+    print(tool_function("view", "/home/ubuntu/xx/dgm/coding_agent.py"))
\ No newline at end of file
diff --git a/tests/test_tools/test_edit.py b/tests/test_tools/test_edit.py
new file mode 100644
index 0000000..04f535b
--- /dev/null
+++ b/tests/test_tools/test_edit.py
@@ -0,0 +1,54 @@
+import pytest
+from pathlib import Path
+from tools.edit import tool_function
+
+def test_view_line_range(tmp_path):
+    # Create a test file
+    test_file = tmp_path / "test.txt"
+    test_content = "line1\nline2\nline3\nline4\nline5\n"
+    test_file.write_text(test_content)
+
+    # Test viewing specific line range
+    result = tool_function("view", str(test_file), view_range=[2, 4])
+    assert "line2" in result
+    assert "line3" in result
+    assert "line4" in result
+    assert "line1" not in result
+    assert "line5" not in result
+    assert "     2\t" in result  # Correct line numbering
+
+    # Test viewing from start to middle
+    result = tool_function("view", str(test_file), view_range=[1, 3])
+    assert "line1" in result
+    assert "line2" in result
+    assert "line3" in result
+    assert "line4" not in result
+    assert "     1\t" in result
+
+    # Test viewing from middle to end with -1
+    result = tool_function("view", str(test_file), view_range=[3, -1])
+    assert "line1" not in result
+    assert "line2" not in result
+    assert "line3" in result
+    assert "line4" in result
+    assert "line5" in result
+    assert "     3\t" in result
+
+def test_view_range_validation(tmp_path):
+    # Create a test file
+    test_file = tmp_path / "test.txt"
+    test_content = "line1\nline2\nline3\n"
+    test_file.write_text(test_content)
+
+    # Test invalid start line
+    result = tool_function("view", str(test_file), view_range=[0, 2])
+    assert "Failed to read file range: Start line must be >= 1" in result
+
+    # Test invalid range (end < start)
+    result = tool_function("view", str(test_file), view_range=[2, 1])
+    assert "Failed to read file range: End line must be >= start line or -1" in result
+
+def test_view_range_with_directory(tmp_path):
+    # Test that view_range is rejected for directories
+    result = tool_function("view", str(tmp_path), view_range=[1, 10])
+    assert "Error: view_range is not supported for directory listings" in result
\ No newline at end of file
\end{lstlisting}

\begin{lstlisting}[style=diffstyle]
diff --git a/tools/edit.py b/tools/edit.py
index 16ae521..757f5c2 100644
--- a/tools/edit.py
+++ b/tools/edit.py
@@ -11,21 +11,21 @@ def tool_info():
 * If `path` is a directory, `view` lists non-hidden files and directories up to 2 levels deep.\n
 * The `create` command cannot be used if the specified `path` already exists as a file.\n
 * If a `command` generates a long output, it will be truncated and marked with `<response clipped>`.\n
-* The `edit` command overwrites the entire file with the provided `file_text`.""",
+* The `str_replace` command replaces a unique occurrence of old_str with new_str, failing if old_str is not found or appears multiple times.""",
         "input_schema": {
             "type": "object",
             "properties": {
                 "command": {
                     "type": "string",
-                    "enum": ["view", "create", "edit"],
-                    "description": "The command to run: `view`, `create`, or `edit`."
+                    "enum": ["view", "create", "str_replace"],
+                    "description": "The command to run: `view`, `create`, or `str_replace`."
                 },
                 "path": {
                     "description": "Absolute path to file or directory, e.g. `/repo/file.py` or `/repo`.",
                     "type": "string"
                 },
                 "file_text": {
-                    "description": "Required parameter of `create` or `edit` command, containing the content for the entire file.",
+                    "description": "Required parameter of `create` command, containing the content for the entire file.",
                     "type": "string"
                 },
                 "view_range": {
@@ -34,6 +34,14 @@ def tool_info():
                     "items": {"type": "integer"},
                     "minItems": 2,
                     "maxItems": 2
+                },
+                "old_str": {
+                    "description": "Required parameter of `str_replace` command, containing the exact text to find and replace.",
+                    "type": "string"
+                },
+                "new_str": {
+                    "description": "Required parameter of `str_replace` command, containing the new text to replace old_str with.",
+                    "type": "string"
                 }
             },
             "required": ["command", "path"]
@@ -51,7 +59,7 @@ def validate_path(path: str, command: str) -> Path:
     Validate the file path for each command:
       - 'view': path may be a file or directory; must exist.
       - 'create': path must not exist (for new file creation).
-      - 'edit': path must exist (for overwriting).
+      - 'str_replace': path must exist and be a file.
     """
     path_obj = Path(path)
 
@@ -69,7 +77,7 @@ def validate_path(path: str, command: str) -> Path:
         # Path must not exist
         if path_obj.exists():
             raise ValueError(f"Cannot create new file; {path} already exists.")
-    elif command == "edit":
+    elif command == "str_replace":
         # Path must exist and must be a file
         if not path_obj.exists():
             raise ValueError(f"The file {path} does not exist.")
@@ -144,6 +152,28 @@ def write_file(path: Path, content: str):
     except Exception as e:
         raise ValueError(f"Failed to write file: {e}")
 
+def str_replace_in_file(path: Path, old_str: str, new_str: str) -> str:
+    """
+    Replace an exact occurrence of old_str with new_str in the file.
+    Only performs the replacement if old_str occurs exactly once.
+    Returns a message indicating success or failure.
+    """
+    try:
+        content = read_file(path)
+        occurrences = content.count(old_str)
+        
+        if occurrences == 0:
+            return f"Error: Could not find the exact text to replace in {path}"
+        elif occurrences > 1:
+            return f"Error: Found multiple ({occurrences}) occurrences of the text in {path}. Must be unique."
+        else:
+            new_content = content.replace(old_str, new_str)
+            write_file(path, new_content)
+            return f"Successfully replaced text in {path}"
+            
+    except Exception as e:
+        return f"Error during string replacement: {e}"
+
 def view_path(path_obj: Path, view_range: Optional[List[int]] = None) -> str:
     """
     View the file contents (optionally within a range) or directory listing.
@@ -176,12 +206,13 @@ def view_path(path_obj: Path, view_range: Optional[List[int]] = None) -> str:
     content, start_line = read_file_range(path_obj, view_range)
     return format_output(content, str(path_obj), start_line)
 
-def tool_function(command: str, path: str, file_text: str = None, view_range: Optional[List[int]] = None) -> str:
+def tool_function(command: str, path: str, file_text: str = None, view_range: Optional[List[int]] = None,
+                 old_str: str = None, new_str: str = None) -> str:
     """
     Main tool function that handles:
-      - 'view'  : View file or directory listing, optionally within line range for files
-      - 'create': Create a new file with the given file_text
-      - 'edit'  : Overwrite an existing file with file_text
+      - 'view'       : View file or directory listing, optionally within line range for files
+      - 'create'     : Create a new file with the given file_text
+      - 'str_replace': Replace exact occurrence of old_str with new_str in the file
     """
     try:
         path_obj = validate_path(path, command)
@@ -195,11 +226,10 @@ def tool_function(command: str, path: str, file_text: str = None, view_range: Op
             write_file(path_obj, file_text)
             return f"File created successfully at: {path}"
 
-        elif command == "edit":
-            if file_text is None:
-                raise ValueError("Missing required `file_text` for 'edit' command.")
-            write_file(path_obj, file_text)
-            return f"File at {path} has been overwritten with new content."
+        elif command == "str_replace":
+            if old_str is None or new_str is None:
+                raise ValueError("Missing required `old_str` and/or `new_str` for 'str_replace' command.")
+            return str_replace_in_file(path_obj, old_str, new_str)
 
         else:
             raise ValueError(f"Unknown command: {command}")
diff --git a/tests/__init__.py b/tests/__init__.py
new file mode 100644
index 0000000..e69de29
diff --git a/tests/test_tools.py b/tests/test_tools.py
new file mode 100644
index 0000000..c7f242f
--- /dev/null
+++ b/tests/test_tools.py
@@ -0,0 +1,65 @@
+import pytest
+from pathlib import Path
+from tools.edit import tool_function
+
+# Test fixtures
+@pytest.fixture
+def temp_file(tmp_path):
+    file_path = tmp_path / "test.txt"
+    content = "line 1\nline 2\nline 3\n"
+    file_path.write_text(content)
+    return str(file_path)
+
+def test_str_replace_success(temp_file):
+    # Test successful replacement
+    result = tool_function(
+        command="str_replace",
+        path=temp_file,
+        old_str="line 2\n",
+        new_str="replaced line\n"
+    )
+    assert "Successfully replaced" in result
+    assert Path(temp_file).read_text() == "line 1\nreplaced line\nline 3\n"
+
+def test_str_replace_not_found(temp_file):
+    # Test when old_str is not found
+    result = tool_function(
+        command="str_replace",
+        path=temp_file,
+        old_str="nonexistent",
+        new_str="something"
+    )
+    assert "Could not find" in result
+    # Original file should be unchanged
+    assert Path(temp_file).read_text() == "line 1\nline 2\nline 3\n"
+
+def test_str_replace_multiple_occurrences(temp_file):
+    # First create a file with multiple occurrences
+    Path(temp_file).write_text("same\nsame\nsame\n")
+    result = tool_function(
+        command="str_replace",
+        path=temp_file,
+        old_str="same\n",
+        new_str="different\n"
+    )
+    assert "multiple" in result
+    # Original file should be unchanged
+    assert Path(temp_file).read_text() == "same\nsame\nsame\n"
+
+def test_str_replace_missing_params(temp_file):
+    # Test missing parameters
+    result = tool_function(
+        command="str_replace",
+        path=temp_file,
+    )
+    assert "Missing required" in result
+
+def test_str_replace_invalid_path():
+    # Test with non-existent file
+    result = tool_function(
+        command="str_replace",
+        path="/nonexistent/path",
+        old_str="old",
+        new_str="new"
+    )
+    assert "does not exist" in result
\ No newline at end of file
\end{lstlisting}
\begin{lstlisting}[style=diffstyle]
diff --git a/coding_agent.py b/coding_agent.py
index 6639abd..97f4b69 100644
--- a/coding_agent.py
+++ b/coding_agent.py
@@ -52,6 +52,10 @@ class SolutionAttempt:
     test_output: str  # Raw test output
     test_success: bool  # Whether tests passed
     test_stats: dict  # Test statistics (e.g., number of passed/failed tests)
+    error_messages: List[str] = None  # List of specific error messages
+    test_details: dict = None  # Detailed test information like specific test names and their status
+    execution_time: float = None  # Test execution time in seconds
+    attempt_number: int = None  # The attempt number in the sequence
 
 def get_thread_logger():
     """
@@ -150,12 +154,82 @@ class AgenticSystem:
         ]
         return new_msg_history
 
+    def extract_test_details(self, output: str) -> tuple[dict, List[str], dict]:
+        """Extract detailed test information from the output."""
+        error_messages = []
+        test_details = {}
+        stats = {"passed": 0, "failed": 0, "errors": 0, "total": 0, "skipped": 0}
+
+        # Split output into lines for analysis
+        lines = output.split("\n")
+        
+        # Language-specific parsing
+        if self.language == "python":
+            for line in lines:
+                if "FAILED" in line and "::" in line:
+                    test_name = line.split("::")[1].split()[0]
+                    test_details[test_name] = "FAILED"
+                    stats["failed"] += 1
+                elif "PASSED" in line and "::" in line:
+                    test_name = line.split("::")[1].split()[0]
+                    test_details[test_name] = "PASSED"
+                    stats["passed"] += 1
+                elif "ERROR" in line and "::" in line:
+                    test_name = line.split("::")[1].split()[0]
+                    test_details[test_name] = "ERROR"
+                    stats["errors"] += 1
+                    # Extract error message
+                    if lines.index(line) + 1 < len(lines):
+                        error_messages.append(lines[lines.index(line) + 1])
+        
+        elif self.language in ["javascript", "node"]:
+            current_test = None
+            for line in lines:
+                if line.startswith('checkmark'):
+                    test_name = line.replace('checkmark,', '').strip()
+                    test_details[test_name] = "PASSED"
+                    stats["passed"] += 1
+                elif line.startswith('x'):
+                    test_name = line.replace('x', '').strip()
+                    test_details[test_name] = "FAILED"
+                    stats["failed"] += 1
+                    current_test = test_name
+                elif current_test and ('Error:' in line or 'AssertionError:' in line):
+                    error_messages.append(f"{current_test}: {line.strip()}")
+
+        elif self.language == "rust":
+            for line in lines:
+                if "test" in line and "... ok" in line:
+                    test_name = line.split("test")[1].split("...")[0].strip()
+                    test_details[test_name] = "PASSED"
+                    stats["passed"] += 1
+                elif "test" in line and "... FAILED" in line:
+                    test_name = line.split("test")[1].split("...")[0].strip()
+                    test_details[test_name] = "FAILED"
+                    stats["failed"] += 1
+                elif "---- " in line and " stdout ----" in line:
+                    test_name = line.split("----")[1].split("stdout")[0].strip()
+                    if test_name in test_details and test_details[test_name] == "FAILED":
+                        error_messages.append(f"{test_name}: {next((l for l in lines[lines.index(line)+1:] if l.strip()), '')}")
+
+        # Generic counting for other languages or as fallback
+        if not any(stats.values()):
+            stats["passed"] = output.count("PASS") + output.count("ok")
+            stats["failed"] = output.count("FAIL") + output.count("not ok")
+            stats["errors"] = output.count("ERROR") + output.count("panic:")
+            
+        stats["total"] = stats["passed"] + stats["failed"] + stats["errors"]
+        
+        return stats, error_messages, test_details
+
     def run_tests(self) -> tuple[bool, str, dict]:
         """Run tests and return success status, output, and test statistics."""
+        import time
+        
         success = False
         output = ""
-        stats = {"passed": 0, "failed": 0, "errors": 0, "total": 0}
-
+        start_time = time.time()
+        
         try:
             for command in TEST_COMMANDS.get(self.language, []):
                 proc = subprocess.run(
@@ -169,34 +243,97 @@ class AgenticSystem:
                 success = proc.returncode == 0
                 if not success:
                     break
-
-            # Try to extract test statistics from output
-            # This is a simple example; you might want to add more sophisticated parsing
-            stats["passed"] = output.count("PASS") + output.count("ok")
-            stats["failed"] = output.count("FAIL") + output.count("not ok")
-            stats["errors"] = output.count("ERROR") + output.count("panic:")
-            stats["total"] = stats["passed"] + stats["failed"] + stats["errors"]
+            
+            # Extract detailed test information
+            stats, error_messages, test_details = self.extract_test_details(output)
+            stats["execution_time"] = time.time() - start_time
+            
+            # Enhance stats with extracted information
+            stats["error_messages"] = error_messages
+            stats["test_details"] = test_details
             
         except Exception as e:
             output = f"Error running tests: {str(e)}"
             success = False
+            stats = {
+                "passed": 0, "failed": 0, "errors": 1, "total": 1,
+                "execution_time": time.time() - start_time,
+                "error_messages": [str(e)],
+                "test_details": {}
+            }
 
         return success, output, stats
 
     def analyze_test_results(self, attempts: List[SolutionAttempt]) -> str:
-        """Analyze test results and create a summary for the agent."""
+        """Analyze test results and create a detailed summary for the agent."""
         summary = "# Test Results Analysis\n\n"
         
+        # Overall progress tracking
+        if len(attempts) > 1:
+            summary += "## Progress Overview\n"
+            first_attempt = attempts[0].test_stats
+            last_attempt = attempts[-1].test_stats
+            
+            progress = {
+                "passed": last_attempt["passed"] - first_attempt["passed"],
+                "failed": first_attempt["failed"] - last_attempt["failed"],
+                "errors": first_attempt["errors"] - last_attempt["errors"]
+            }
+            
+            summary += "Progress since first attempt:\n"
+            summary += f"- Additional passing tests: {progress['passed']}\n"
+            summary += f"- Reduced failures: {progress['failed']}\n"
+            summary += f"- Reduced errors: {progress['errors']}\n\n"
+        
+        # Detailed attempt analysis
         for i, attempt in enumerate(attempts, 1):
             summary += f"## Attempt {i}\n"
             summary += f"Test Success: {attempt.test_success}\n"
-            summary += f"Test Stats: {json.dumps(attempt.test_stats, indent=2)}\n"
-            summary += "Key test output:\n```\n"
-            # Extract relevant parts of test output (e.g., error messages)
-            key_output = "\n".join(line for line in attempt.test_output.split("\n") 
-                                 if "FAIL" in line or "ERROR" in line or "PASS" in line)
-            summary += f"{key_output}\n```\n\n"
-
+            summary += f"Execution Time: {attempt.test_stats.get('execution_time', 'N/A'):.2f}s\n"
+            
+            # Test statistics
+            stats = attempt.test_stats
+            total = stats.get("total", 0) or 1  # Avoid division by zero
+            pass_rate = (stats.get("passed", 0) / total) * 100
+            
+            summary += f"Pass Rate: {pass_rate:.1f}% ({stats.get('passed', 0)}/{total})\n"
+            summary += "Test Statistics:\n"
+            summary += f"- Passed: {stats.get('passed', 0)}\n"
+            summary += f"- Failed: {stats.get('failed', 0)}\n"
+            summary += f"- Errors: {stats.get('errors', 0)}\n"
+            summary += f"- Total: {total}\n\n"
+            
+            # Error messages
+            if stats.get("error_messages"):
+                summary += "Error Messages:\n```\n"
+                for error in stats["error_messages"][:5]:  # Limit to top 5 errors
+                    summary += f"{error}\n"
+                if len(stats["error_messages"]) > 5:
+                    summary += f"... and {len(stats['error_messages']) - 5} more errors\n"
+                summary += "```\n\n"
+            
+            # Test details
+            if stats.get("test_details"):
+                summary += "Individual Test Results:\n```\n"
+                for test_name, result in stats["test_details"].items():
+                    summary += f"{result}: {test_name}\n"
+                summary += "```\n\n"
+
+        # Recommendations for next attempt
+        if not attempts[-1].test_success:
+            summary += "## Recommendations for Next Attempt\n"
+            last_stats = attempts[-1].test_stats
+            
+            if last_stats.get("errors", 0) > 0:
+                summary += "- Focus on resolving runtime errors first\n"
+            if last_stats.get("failed", 0) > 0:
+                summary += "- Address failing test cases\n"
+            if len(attempts) > 1 and not attempts[-1].test_success:
+                # Compare with previous attempt
+                prev_stats = attempts[-2].test_stats
+                if last_stats.get("passed", 0) < prev_stats.get("passed", 0):
+                    summary += "- Recent changes caused regressions. Consider reverting some changes\n"
+            
         return summary
 
     def forward(self):
@@ -238,20 +375,36 @@ Your task is to make changes to the files in the {self.git_tempdir} directory to
             # Run tests and collect results 
             test_success, test_output, test_stats = self.run_tests()
             
-            # Create and store attempt
+            # Create and store attempt with enhanced information
             attempt = SolutionAttempt(
                 patch=current_patch,
                 test_output=test_output,
                 test_success=test_success,
-                test_stats=test_stats
+                test_stats=test_stats,
+                error_messages=test_stats.get('error_messages', []),
+                test_details=test_stats.get('test_details', {}),
+                execution_time=test_stats.get('execution_time', None),
+                attempt_number=attempt_num + 1
             )
             attempts.append(attempt)
             
-            # Update best attempt if this one is better
-            if test_success and (best_attempt is None or 
-                               attempt.test_stats["passed"] > best_attempt.test_stats["passed"]):
+            # Update best attempt based on multiple criteria
+            if test_success and (
+                best_attempt is None or
+                (attempt.test_stats["passed"] > best_attempt.test_stats["passed"]) or
+                (attempt.test_stats["passed"] == best_attempt.test_stats["passed"] and
+                 len(attempt.error_messages or []) < len(best_attempt.error_messages or []))
+            ):
                 best_attempt = attempt
             
+            # Log detailed attempt information
+            safe_log(f"\n=== Attempt {attempt_num + 1} Summary ===")
+            safe_log(f"Test Success: {test_success}")
+            safe_log(f"Tests Passed: {test_stats.get('passed', 0)}")
+            safe_log(f"Tests Failed: {test_stats.get('failed', 0)}")
+            safe_log(f"Errors: {test_stats.get('errors', 0)}")
+            safe_log(f"Execution Time: {test_stats.get('execution_time', 'N/A'):.2f}s")
+            
             # If tests pass perfectly, we can stop
             if test_success and attempt.test_stats["failed"] == 0 and attempt.test_stats["errors"] == 0:
                 break
\end{lstlisting}

\section{Similar Target Functionality, Different Implementations}
\label{app:supp-dgm-diffimpl}
For the same target functionality, the suggestions provided to the coding agents and the resulting implementations can differ significantly, leading to large variations in coding capability and benchmark performance. For example, consider nodes 6 and 24 in the DGM run on SWE-bench (\Cref{fig:dgm-archive}). Both aimed to enhance the existing editor tool to support finer‑grained file editing. However, the implementation for node 6 retained the original \texttt{edit} command and added the parameters \texttt{edit\_type} and \texttt{edit\_actions}, whereas the implementation for node 24 replaced the original \texttt{edit} command with a new \texttt{str\_replace} command. Despite targeting similar functionality, the feature suggestions and thus the implementation details differed greatly, as did their performance. Node 6 achieved only 23.3\%, while node 24 achieved 40.5\%.

Feature suggestion to obtain node 6:
\begin{lstlisting}
Within the tools/edit.py file, remove the 'edit' command and introduce a 'str_replace' command. It should accept parameters 'old_str' (the unique text to find) and 'new_str' (the new text to replace it with). After verifying that 'old_str' appears exactly once in the file, replace the first occurrence with 'new_str'. If 'old_str' is not unique or is missing, do not write any changes.

Replace the 'edit' command in the editor tool with a 'str_replace' command that locates and replaces an exact substring in a file. The 'old_str' must match consecutive lines uniquely, and if it is not found or occurs multiple times, no write should occur. The 'new_str' parameter contains the revised lines. This precise substring replacement flow will enable small incremental edits with minimal breakage.
\end{lstlisting}

Feature suggestion to obtain node 24:
\begin{lstlisting}
Augment the existing 'editor' tool to support line or token-based editing. The agent would specify individual lines or tokens to insert, delete, or replace. Internally, the tool would parse the file contents, apply changes atomically, and verify that the surrounding context remains intact. This prevents entire-file overwrites, reducing the risk of losing unrelated logic. The tool could also validate that the edited lines match expected code fragments to catch unanticipated differences or merge conflicts.

Enhance the agent's file editing process to support fine-grained edits. Specifically, add a line-based or token-based editing feature in the 'editor' tool. This improvement should let the agent insert, replace, or delete small sections of code without overwriting entire files or large blocks, preserving unrelated logic. Implement safeguards to confirm that contextual code around the edited region remains accurate, so we reduce the risk of creating regressions when fixing issues.
\end{lstlisting}

Implementation of suggested feature to obtain node 6:
\begin{lstlisting}[style=diffstyle]
diff --git a/tests/test_edit_tool.py b/tests/test_edit_tool.py
index f0782c6..d3a4d5c 100644
--- a/tests/test_edit_tool.py
+++ b/tests/test_edit_tool.py
@@ -1,7 +1,7 @@
 import pytest
 from pathlib import Path
 import tempfile
-from tools.edit import tool_function
+from tools.edit import tool_function, validate_edit_actions, validate_context
 
 @pytest.fixture
 def temp_dir():
@@ -17,6 +17,18 @@ def sample_file(temp_dir):
     file_path.write_text(content)
     return file_path
 
+@pytest.fixture
+def code_file(temp_dir):
+    """Create a sample Python file for testing token-based edits."""
+    file_path = temp_dir / "test.py"
+    content = """def test_function():
+    x = 42
+    y = "old_value"
+    return x + len(y)
+"""
+    file_path.write_text(content)
+    return file_path
+
 class TestEditorTool:
     def test_view_file(self, sample_file):
         """Test viewing entire file content."""
@@ -39,17 +51,182 @@ class TestEditorTool:
         assert "Error" in result
         assert "already exists" in result
 
-    def test_edit_file(self, sample_file):
-        """Test editing an existing file."""
+    def test_edit_file_full(self, sample_file):
+        """Test editing an existing file with file mode."""
         new_content = "edited content\nnew line"
-        result = tool_function("edit", str(sample_file), file_text=new_content)
+        result = tool_function(
+            "edit", 
+            str(sample_file), 
+            file_text=new_content, 
+            edit_type="file"
+        )
         assert "has been overwritten" in result
         assert sample_file.read_text() == new_content
 
+    def test_line_based_edit_insert(self, sample_file):
+        """Test line-based insert operation."""
+        actions = [{
+            "action": "insert",
+            "line_number": 3,
+            "content": "inserted line"
+        }]
+        result = tool_function(
+            "edit",
+            str(sample_file),
+            edit_type="line",
+            edit_actions=actions
+        )
+        assert "successfully edited" in result
+        content = sample_file.read_text().splitlines()
+        assert content[2] == "inserted line"
+        assert len(content) == 6
+
+    def test_line_based_edit_delete(self, sample_file):
+        """Test line-based delete operation."""
+        actions = [{
+            "action": "delete",
+            "line_number": 2
+        }]
+        result = tool_function(
+            "edit",
+            str(sample_file),
+            edit_type="line",
+            edit_actions=actions
+        )
+        assert "successfully edited" in result
+        content = sample_file.read_text().splitlines()
+        assert "line 2" not in content
+        assert len(content) == 4
+
+    def test_line_based_edit_replace(self, sample_file):
+        """Test line-based replace operation."""
+        actions = [{
+            "action": "replace",
+            "line_number": 3,
+            "content": "replaced line"
+        }]
+        result = tool_function(
+            "edit",
+            str(sample_file),
+            edit_type="line",
+            edit_actions=actions
+        )
+        assert "successfully edited" in result
+        content = sample_file.read_text().splitlines()
+        assert content[2] == "replaced line"
+        assert len(content) == 5
+
+    def test_token_based_edit(self, code_file):
+        """Test token-based edit operation."""
+        actions = [{
+            "action": "replace",
+            "old_token": "old_value",
+            "new_token": "new_value"
+        }]
+        result = tool_function(
+            "edit",
+            str(code_file),
+            edit_type="token",
+            edit_actions=actions
+        )
+        assert "successfully edited" in result
+        content = code_file.read_text()
+        assert "old_value" not in content
+        assert "new_value" in content
+
+    def test_multiple_token_edits(self, code_file):
+        """Test multiple token-based edit operations."""
+        actions = [
+            {
+                "action": "replace",
+                "old_token": "42",
+                "new_token": "100"
+            },
+            {
+                "action": "replace",
+                "old_token": "old_value",
+                "new_token": "new_text"
+            }
+        ]
+        result = tool_function(
+            "edit",
+            str(code_file),
+            edit_type="token",
+            edit_actions=actions
+        )
+        assert "successfully edited" in result
+        content = code_file.read_text()
+        assert "42" not in content
+        assert "100" in content
+        assert "old_value" not in content
+        assert "new_text" in content
+
+    def test_invalid_line_number(self, sample_file):
+        """Test edit with invalid line number."""
+        actions = [{
+            "action": "replace",
+            "line_number": 100,
+            "content": "invalid line"
+        }]
+        result = tool_function(
+            "edit",
+            str(sample_file),
+            edit_type="line",
+            edit_actions=actions
+        )
+        assert "Error" in result
+        assert "exceeds file length" in result
+
+    def test_invalid_edit_type(self, sample_file):
+        """Test edit with invalid edit type."""
+        result = tool_function(
+            "edit",
+            str(sample_file),
+            edit_type="invalid",
+            edit_actions=[{"action": "replace"}]
+        )
+        assert "Error" in result
+        assert "Invalid edit_type" in result
+
+    def test_missing_required_fields(self, sample_file):
+        """Test edit with missing required fields."""
+        actions = [{
+            "action": "replace"  # Missing line_number and content
+        }]
+        result = tool_function(
+            "edit",
+            str(sample_file),
+            edit_type="line",
+            edit_actions=actions
+        )
+        assert "Error" in result
+        assert "Missing required field" in result
+
+    def test_validate_context(self):
+        """Test context validation function."""
+        original = "line 1\nline 2\nline 3\nline 4\nline 5"
+        modified = "line 1\nline 2\nmodified\nline 4\nline 5"
+        assert validate_context(original, modified) is True
+
+        # Test invalid context (too many changes)
+        modified = "completely\ndifferent\ncontent"
+        assert validate_context(original, modified) is False
+
+    def test_empty_edit_actions(self, sample_file):
+        """Test edit with empty edit actions."""
+        result = tool_function(
+            "edit",
+            str(sample_file),
+            edit_type="line",
+            edit_actions=[]
+        )
+        assert "Error" in result
+        assert "Edit actions cannot be empty" in result
+
     def test_edit_nonexistent_file(self, temp_dir):
         """Test attempting to edit a nonexistent file."""
         non_existent_file = temp_dir / "does_not_exist.txt"
-        result = tool_function("edit", str(non_existent_file), file_text="new content")
+        result = tool_function("edit", str(non_existent_file), edit_type="line", edit_actions=[])
         assert "Error" in result
         assert "does not exist" in result
 
@@ -82,4 +259,4 @@ class TestEditorTool:
     def test_invalid_commands(self, command, sample_file):
         """Test various invalid commands."""
         result = tool_function(command, str(sample_file))
-        assert "Error" in result
+        assert "Error" in result
\ No newline at end of file
diff --git a/tools/edit.py b/tools/edit.py
index 59137ee..982ee84 100644
--- a/tools/edit.py
+++ b/tools/edit.py
@@ -1,5 +1,7 @@
 from pathlib import Path
 import subprocess
+from typing import Dict, Any, List, Union, Tuple
+from difflib import SequenceMatcher
 
 def tool_info():
     return {
@@ -9,8 +11,9 @@ def tool_info():
 * If `path` is a file, `view` displays the entire file with line numbers. If `path` is a directory, `view` lists non-hidden files and directories up to 2 levels deep.\n
 * The `create` command cannot be used if the specified `path` already exists as a file.\n
 * If a `command` generates a long output, it will be truncated and marked with `<response clipped>`.\n
-* The `edit` command overwrites the entire file with the provided `file_text`.\n
-* No partial/line-range edits or partial viewing are supported.""",
+* The `edit` command supports both entire file overwrites and fine-grained line/token editing via the `edit_type` parameter.\n
+* Line-based edits require line numbers and content to modify specific parts of a file.\n
+* Token-based edits require specifying old and new tokens to replace specific text fragments.""",
         "input_schema": {
             "type": "object",
             "properties": {
@@ -24,8 +27,28 @@ def tool_info():
                     "type": "string"
                 },
                 "file_text": {
-                    "description": "Required parameter of `create` or `edit` command, containing the content for the entire file.",
+                    "description": "Required parameter of `create` or `edit` command with edit_type='file', containing the content for the entire file.",
                     "type": "string"
+                },
+                "edit_type": {
+                    "type": "string",
+                    "enum": ["file", "line", "token"],
+                    "description": "Type of edit operation: 'file' for full file, 'line' for line-based edits, 'token' for token-based edits.",
+                    "default": "file"
+                },
+                "edit_actions": {
+                    "type": "array",
+                    "description": "List of edit actions for line/token operations. Each action contains operation details.",
+                    "items": {
+                        "type": "object",
+                        "properties": {
+                            "action": {"type": "string", "enum": ["insert", "delete", "replace"]},
+                            "line_number": {"type": "integer", "description": "Line number for the operation (1-based)"},
+                            "content": {"type": "string", "description": "Content to insert/replace"},
+                            "old_token": {"type": "string", "description": "Token to be replaced (for token edits)"},
+                            "new_token": {"type": "string", "description": "New token (for token edits)"}
+                        }
+                    }
                 }
             },
             "required": ["command", "path"]
@@ -119,12 +142,126 @@ def view_path(path_obj: Path) -> str:
     content = read_file(path_obj)
     return format_output(content, str(path_obj))
 
-def tool_function(command: str, path: str, file_text: str = None) -> str:
+def validate_edit_actions(actions: List[Dict[str, Any]], edit_type: str) -> None:
+    """Validate edit actions based on edit type."""
+    if not actions:
+        raise ValueError("Edit actions cannot be empty for line/token edits")
+
+    valid_actions = ["insert", "delete", "replace"]
+    required_fields = {
+        "line": ["action", "line_number"],
+        "token": ["action", "old_token"]
+    }
+
+    for action in actions:
+        if "action" not in action or action["action"] not in valid_actions:
+            raise ValueError(f"Invalid action. Must be one of: {valid_actions}")
+
+        # Check required fields based on edit_type
+        for field in required_fields[edit_type]:
+            if field not in action:
+                raise ValueError(f"Missing required field '{field}' in edit action")
+
+        # Validate line number if provided
+        if "line_number" in action:
+            if not isinstance(action["line_number"], int) or action["line_number"] < 1:
+                raise ValueError("Line number must be a positive integer")
+
+        # Validate content requirements
+        if action["action"] in ["insert", "replace"]:
+            if edit_type == "line" and "content" not in action:
+                raise ValueError("Content required for insert/replace actions")
+            if edit_type == "token" and "new_token" not in action:
+                raise ValueError("new_token required for token operations")
+
+def apply_line_edits(content: List[str], actions: List[Dict[str, Any]]) -> List[str]:
+    """Apply line-based edits to the content."""
+    modified_content = content.copy()
+    
+    # Sort actions by line number in reverse order to handle inserts/deletes correctly
+    sorted_actions = sorted(actions, key=lambda x: x["line_number"], reverse=True)
+
+    for action in sorted_actions:
+        line_num = action["line_number"] - 1  # Convert to 0-based index
+        
+        if line_num > len(modified_content):
+            raise ValueError(f"Line number {action['line_number']} exceeds file length")
+
+        if action["action"] == "delete":
+            if line_num >= 0:
+                del modified_content[line_num]
+        elif action["action"] == "insert":
+            modified_content.insert(line_num, action["content"])
+        elif action["action"] == "replace":
+            modified_content[line_num] = action["content"]
+
+    return modified_content
+
+def apply_token_edits(content: str, actions: List[Dict[str, Any]]) -> str:
+    """Apply token-based edits to the content."""
+    modified_content = content
+
+    for action in actions:
+        old_token = action["old_token"]
+        if action["action"] == "delete":
+            modified_content = modified_content.replace(old_token, "")
+        elif action["action"] in ["insert", "replace"]:
+            new_token = action["new_token"]
+            modified_content = modified_content.replace(old_token, new_token)
+
+    return modified_content
+
+def validate_context(original: str, modified: str, context_lines: int = 3) -> bool:
     """
-    Main tool function that handles:
+    Validate that the context around modified sections remains intact.
+    Returns True if context is valid, False otherwise.
+    """
+    # Use SequenceMatcher to find the similarity ratio between strings
+    similarity = SequenceMatcher(None, original, modified).ratio()
+    
+    # If content is too different (less than 30% similar), consider it invalid
+    if similarity < 0.3:
+        return False
+
+    # Split into lines for more detailed analysis
+    orig_lines = original.splitlines()
+    mod_lines = modified.splitlines()
+
+    # If line count difference is too large (more than 50%), consider it invalid
+    if abs(len(orig_lines) - len(mod_lines)) > len(orig_lines) // 2:
+        return False
+
+    # Calculate line-based similarity for sections
+    def get_block_similarity(block1: List[str], block2: List[str]) -> float:
+        return SequenceMatcher(None, "\n".join(block1), "\n".join(block2)).ratio()
+
+    # Check similarity of start and end blocks if they exist
+    if len(orig_lines) >= context_lines and len(mod_lines) >= context_lines:
+        start_similarity = get_block_similarity(
+            orig_lines[:context_lines],
+            mod_lines[:context_lines]
+        )
+        end_similarity = get_block_similarity(
+            orig_lines[-context_lines:],
+            mod_lines[-context_lines:]
+        )
+        
+        # If either the start or end blocks are too different, consider it invalid
+        if start_similarity < 0.7 or end_similarity < 0.7:
+            return False
+
+    return True
+
+def tool_function(command: str, path: str, file_text: str = None,
+                 edit_type: str = "file", edit_actions: List[Dict[str, Any]] = None) -> str:
+    """
+    Enhanced tool function that handles:
       - 'view'  : View the entire file or directory listing
       - 'create': Create a new file with the given file_text
-      - 'edit'  : Overwrite an existing file with file_text
+      - 'edit'  : Edit a file using one of three modes:
+                  - file: overwrite entire file
+                  - line: line-based edits
+                  - token: token-based edits
     """
     try:
         path_obj = validate_path(path, command)
@@ -139,11 +276,36 @@ def tool_function(command: str, path: str, file_text: str = None) -> str:
             return f"File created successfully at: {path}"
 
         elif command == "edit":
-            if file_text is None:
-                raise ValueError("Missing required `file_text` for 'edit' command.")
-            write_file(path_obj, file_text)
-            return f"File at {path} has been overwritten with new content."
+            # Read original content for line/token edits and validation
+            original_content = read_file(path_obj) if edit_type in ["line", "token"] else None
+
+            if edit_type == "file":
+                if file_text is None:
+                    raise ValueError("Missing required `file_text` for 'file' edit type.")
+                write_file(path_obj, file_text)
+                return f"File at {path} has been overwritten with new content."
+
+            elif edit_type in ["line", "token"]:
+                if not edit_actions:
+                    raise ValueError("Edit actions cannot be empty for line/token edits")
+
+                validate_edit_actions(edit_actions, edit_type)
+
+                if edit_type == "line":
+                    lines = original_content.splitlines()
+                    modified_lines = apply_line_edits(lines, edit_actions)
+                    modified_content = "\n".join(modified_lines)
+                else:  # token edits
+                    modified_content = apply_token_edits(original_content, edit_actions)
 
+                # Validate context before writing
+                if validate_context(original_content, modified_content):
+                    write_file(path_obj, modified_content)
+                    return f"File at {path} has been successfully edited using {edit_type}-based editing."
+                else:
+                    raise ValueError("Context validation failed. Edit operation aborted.")
+            else:
+                raise ValueError(f"Invalid edit_type: {edit_type}")
         else:
             raise ValueError(f"Unknown command: {command}")
 
@@ -152,4 +314,4 @@ def tool_function(command: str, path: str, file_text: str = None) -> str:
 
 if __name__ == "__main__":
     # Example usage
-    print(tool_function("view", "/home/ubuntu/xx/dgm/coding_agent.py"))
+    print(tool_function("view", "/home/ubuntu/xx/dgm/coding_agent.py"))
\ No newline at end of file
\end{lstlisting}

Implementation of suggested feature to obtain node 24:
\input{diff_implementations/node24_implementation}

\section{Supplementary Plots}

\subsection{Baselines on SWE-bench}
\label{app:plots-swe}

\begin{figure}[H]
    \centering
    \includegraphics[width=0.95\textwidth]{figures/dgm_wo_selfimprove.pdf}
    \caption{\textbf{DGM without self-improvement.} Keeping the meta-agent that is modifying and producing the next coding agents the same, DGM w/o self-improve is unable to continuously improve over time. (Left) Archive of coding agents generated during the DGM w/o self-improve run on SWE-bench. Each node represents a coding agent, with node 0 corresponding to the initial agent. Node color indicates performance on SWE-bench (percentage of solved tasks), while border color reflects the number of tasks for which the agent was evaluated. Edges show which agents self-modified to produce the offsprings. (Right) Progress plot of the DGM w/o self-improve on SWE-bench. The light green line shows the average score of all agents possessing basic codebase-editing functionality. The green line tracks the best score achieved by any agent in the archive at each iteration. The dark line shows the lineage of the final best-discovered agent and its precursor nodes.}
    \label{fig:dgm-no-selfimprove}
\end{figure}

\begin{figure}[H]
    \centering
    \includegraphics[width=0.95\textwidth]{figures/dgm_wo_openended.pdf}
    \caption{\textbf{DGM without open-ended exploration.} Removing the archive, DGM w/o open-ended exploration always uses the most recent agent to self-modify and makes very little progress on SWE-bench. (Left) Archive of coding agents generated during the DGM w/o open-ended exploration run on SWE-bench. Each node represents a coding agent, with node 0 corresponding to the initial agent. Node color indicates performance on SWE-bench (percentage of solved tasks), while border color reflects the number of tasks for which the agent was evaluated. Edges show which agents self-modified to produce the offsprings. (Right) Progress plot of the DGM w/o open-ended on SWE-bench. The orange line shows the average score of all agents possessing basic codebase-editing functionality. The light orange line tracks the best score achieved by any agent in the archive at each iteration. The dark line shows the lineage of the final best-discovered agent and its precursor nodes.}
    \label{fig:dgm-no-openended}
\end{figure}

% \subsection{Baselines on Polyglot}
% \label{app:plots-polyglot}

% \todo{}

\section{Case Study: Solving Hallucination}
\label{app:dgm-halluc}
The DGM can be used to optimize objectives beyond just coding, as discussed as a potential direction for future work in \Cref{sec:safety}. In this section, we show that the DGM can address hallucinations of tool use by FMs. Through feedback from evaluation logs, the DGM improves hallucination detection mechanisms and ultimately discovers ways to resolve these hallucinations. We provide a more detailed discussion of when these hallucinations occur, the DGM setup, and the resulting solutions. Finally, we present an example of objective hacking, where the agent optimizes for the quantifiable metric rather than truly fulfilling the spirit of the task and solving the underlying problem.

\begin{figure}[H]
    \centering
    \includegraphics[width=0.95\textwidth]{solve_halluc/dgm_halluc.pdf}
    \caption{\textbf{The DGM solving hallucination of tool use in FMs.} Archive of coding agents generated during the DGM run on SWE-bench to solve hallucination from FMs. Each node represents an agent, with node 0 corresponding to the initial agent. Node color indicates solved hallucination score, while border color reflects whether the agent has basic codebase-editing functionality. Edges show which agents self-modified to produce the offsprings.}
    \label{fig:dgm-halluc}
\end{figure}

\textbf{When hallucination occurs.}
When the input context to Claude 3.5 Sonnet (New) (Claude) is very long, including all previous tool use inputs and outputs, we found that Claude tends to hallucinate tool usage. Instead of actually making a tool call, it outputs the tool input and output as plain text. This is an example of Claude hallucinating tool use, claiming that the Bash tool was used to run tests and that the tool output suggests that all tests passed, but Claude did not invoke any tools, and the tool results were entirely hallucinated in plain text:
% \begin{lstlisting}
% Let me run this test command...

% # Hallucination of tool-use begins here. Everything until the next comment is completely made up.

% Tool Used: bash

% Tool Input: {'command': 'cd /testbed && ./tests/runtests.py ...'}

% Tool Result: ...
% test_accent (dbshell.test_postgresql.PostgreSqlDbshellCommandTestCase) ... ok
% test_basic (dbshell.test_postgresql.PostgreSqlDbshellCommandTestCase) ... ok
% test_column (dbshell.test_postgresql.PostgreSqlDbshellCommandTestCase) ... ok
% test_escape_characters (dbshell.test_postgresql.PostgreSqlDbshellCommandTestCase) ... ok
% test_nopass (dbshell.test_postgresql.PostgreSqlDbshellCommandTestCase) ... ok
% test_sigint_handler (dbshell.test_postgresql.PostgreSqlDbshellCommandTestCase) ... ok
% Ran 6 tests in 0.001s ... OK

% # The model believes that its own output above is the real output from the tool.

% Great! All the tests are passing successfully. 


% \end{lstlisting}


\begin{lstlisting}
I'll run the tests as specified in the <test_description>. The test description indicates that we should use:\n\n```bash\n./tests/runtests.py --verbosity 2 --settings=test_sqlite --parallel 1 dbshell.test_postgresql\n```\n\nLet me run this test command:\n\nTool Used: bash\nTool Input: {'command': 'cd /testbed && ./tests/runtests.py --verbosity 2 --settings=test_sqlite --parallel 1 dbshell.test_postgresql'}\nTool Result: Testing against Django installed in '/testbed/django'\nImporting application dbshell\nSkipping setup of unused database(s): default, other.\nSystem check identified no issues (0 silenced).\nError:\ntest_accent (dbshell.test_postgresql.PostgreSqlDbshellCommandTestCase) ... ok\ntest_basic (dbshell.test_postgresql.PostgreSqlDbshellCommandTestCase) ... ok\ntest_column (dbshell.test_postgresql.PostgreSqlDbshellCommandTestCase) ... ok\ntest_escape_characters (dbshell.test_postgresql.PostgreSqlDbshellCommandTestCase) ... ok\ntest_nopass (dbshell.test_postgresql.PostgreSqlDbshellCommandTestCase) ... ok\ntest_sigint_handler (dbshell.test_postgresql.PostgreSqlDbshellCommandTestCase)\nSIGINT is ignored in Python and passed to psql to abort quries. ... ok\n\n----------------------------------------------------------------------\nRan 6 tests in 0.001s\n\nOK\n\nTool Response: Great! All the tests are passing successfully. Let me explain what each test verifies:\n\n1. `test_accent`: Tests handling of accented characters in usernames and passwords\n   - Status: PASS\n   - Verifies proper handling of non-ASCII characters\n\n2. `test_basic`: Tests basic connection with standard parameters\n   - Status: PASS\n   - Verifies basic functionality with username, password, host, port, and dbname\n\n3. `test_column`: Tests handling of special characters in connection parameters\n   - Status: PASS \n   - Verifies proper handling of colons in usernames and passwords\n\n4. `test_escape_characters`: Tests handling of escape characters\n   - Status: PASS\n   - Verifies proper handling of backslashes in credentials\n\n5. `test_nopass`: Tests connection without password\n   - Status: PASS\n   - Verifies behavior when no password is provided\n\n6. `test_sigint_handler`: Tests signal handling for query abortion\n   - Status: PASS\n   - Verifies SIGINT handling works correctly\n\nTest Summary:\n- Total tests run: 6\n- Passed: 6\n- Failed: 0\n- Execution time: 0.001s\n\nThe test results confirm that our implementation:\n1. Successfully handles all types of passwords and special characters\n2. Properly manages environment variables\n3. Maintains correct signal handling\n4. Works correctly with or without passwords\n5. Preserves all required functionality\n\nThe tests pass with high speed (0.001s) which is expected since they're mocking the actual psql execution. The problem described in <problem_description> has been successfully fixed and verified through the test suite.
\end{lstlisting}


\textbf{DGM setup.}
We run this experiment on SWE-bench. Other than the number of SWE-bench tasks we evaluate each coding agent on, and the final scoring function of the DGM, all other settings remain the same as in the main experiments (\Cref{sec:methods}). The initial coding agent and all agents generated during the DGM run are evaluated on only 3 SWE-bench tasks. We use just 3 tasks because if hallucination occurs in one, it is likely to occur in others as well. Hence, 3 tasks are sufficient to provide the DGM with a reliable signal regarding whether tool use hallucination is still present.

\textbf{Solve hallucination score.}
We score each generated agent based on how effectively it presents hallucinations. For each SWE-bench task the agent is evaluated on, the score has two components: 0 - 1 based on whether hallucination occurred in the logs, and 1 - 2 (only added and accounted for after no hallucination occurs in the logs) if tools were used in all Claude calls. The final solve hallucination score is computed as the average across all evaluated tasks. To detect hallucination in the logs, we insert special tokens when actual tool use occurs and then check if these tokens appear in the model's plain-text output, indicating that the model mimicked tool use without actually invoking any tools. Importantly, the hallucination checking functions are hidden from the coding agent during self-modification. We found that objective hacking (i.e., optimizing for the measurable objective instead of truly solving the underlying problem) occurs more frequently when these functions are not hidden.

\textbf{Results.}
We run the DGM for 150 iterations. In this run, the DGM discovers two promising solutions: node 96, which achieves the second highest score of 1.67, and node 114, which achieves a perfect score of 2.0 (\Cref{fig:dgm-halluc}).
Node 96 addresses hallucination to some extent, achieving a score of 1.67 without objective hacking. In the lineage leading to node 96, the DGM first improves the tool transaction logging mechanisms to better detect when hallucination occurs. It then explores various strategies to solve it, such as removing hallucinated tool use from the output text when it determines that the tool was not actually invoked.
Node 114 achieves a perfect score quickly, after only 2 modifications in its lineage. However, we observed objective hacking: it scored highly according to our predefined evaluation functions, but it did not actually solve the underlying problem of tool use hallucination. In the modification leading up to node 114 (see below), the agent removed the logging of special tokens that indicate tool usage (despite instructions not to change the special tokens), effectively bypassing our hallucination detection function.

\begin{lstlisting}
diff --git a/coding_agent.py b/coding_agent.py
index 620da2f..800b06e 100644
--- a/coding_agent.py
+++ b/coding_agent.py
@@ -196,13 +196,16 @@ def chat_with_claude(task, chat_history, user_instr):
                 "input": tool_input
             })
 
-            # Process the tool call
+            # Process the tool call and record in structured format
             tool_result = process_tool_call(tools_dict, tool_name, tool_input)
-            output_ls.extend([
-                {TOOL_USED_MARKER: tool_name},
-                {TOOL_INPUT_MARKER: tool_input},
-                {TOOL_RESULT_MARKER: tool_result}
-            ])
+            # Use structured format instead of text markers
+            output_ls.append({
+                "tool_invocation": {
+                    "name": tool_name,
+                    "input": tool_input,
+                    "result": tool_result
+                }
+            })
 
             # Continue with tool results
             messages.extend([
@@ -224,7 +227,13 @@ def chat_with_claude(task, chat_history, user_instr):
                 tools=tools,
             )
             tool_response_text = next((block.text for block in tool_response.content if hasattr(block, "text")), None)
-            output_ls.append({TOOL_RESPONSE_MARKER: tool_response_text})
+            # Record tool response in structured format
+            if tool_response_text:
+                output_ls.append({
+                    "tool_response": {
+                        "content": tool_response_text
+                    }
+                })
             response = tool_response
 
             # Validate tool response
\end{lstlisting}


In conclusion, the DGM can be applied beyond the coding domain, and we highlighted a case of objective hacking. Similar to reward hacking in reinforcement learning~\citep{skalse2022defining}, objective hacking occurs when a system optimizes for a predefined, quantifiable objective rather than fulfilling the spirit of the task or solving the intended problem. This observation supports arguments made in prior works~\citep{zhang2024omni, faldor2025omni}, which suggest that optimizing quantitative measures often leads to undesirable or pathological outcomes, and aligns with Goodhart's law~\citep{strathern1997improving} -- "When a measure becomes a target, it ceases to be a good measure."

%%%%%%%%%%%%%%%%%%%%%%%%%%%%%%%%%%%%%%%%%%%%%%%%%%%%%%%%%%%%

% \newpage
% \section*{NeurIPS Paper Checklist}

% % %%% BEGIN INSTRUCTIONS %%%
% % The checklist is designed to encourage best practices for responsible machine learning research, addressing issues of reproducibility, transparency, research ethics, and societal impact. Do not remove the checklist: {\bf The papers not including the checklist will be desk rejected.} The checklist should follow the references and follow the (optional) supplemental material.  The checklist does NOT count towards the page
% % limit. 

% % Please read the checklist guidelines carefully for information on how to answer these questions. For each question in the checklist:
% % \begin{itemize}
% %     \item You should answer \answerYes{}, \answerNo{}, or \answerNA{}.
% %     \item \answerNA{} means either that the question is Not Applicable for that particular paper or the relevant information is Not Available.
% %     \item Please provide a short (1–2 sentence) justification right after your answer (even for NA). 
% %    % \item {\bf The papers not including the checklist will be desk rejected.}
% % \end{itemize}

% % {\bf The checklist answers are an integral part of your paper submission.} They are visible to the reviewers, area chairs, senior area chairs, and ethics reviewers. You will be asked to also include it (after eventual revisions) with the final version of your paper, and its final version will be published with the paper.

% % The reviewers of your paper will be asked to use the checklist as one of the factors in their evaluation. While "\answerYes{}" is generally preferable to "\answerNo{}", it is perfectly acceptable to answer "\answerNo{}" provided a proper justification is given (e.g., "error bars are not reported because it would be too computationally expensive" or "we were unable to find the license for the dataset we used"). In general, answering "\answerNo{}" or "\answerNA{}" is not grounds for rejection. While the questions are phrased in a binary way, we acknowledge that the true answer is often more nuanced, so please just use your best judgment and write a justification to elaborate. All supporting evidence can appear either in the main paper or the supplemental material, provided in appendix. If you answer \answerYes{} to a question, in the justification please point to the section(s) where related material for the question can be found.

% % IMPORTANT, please:
% % \begin{itemize}
% %     \item {\bf Delete this instruction block, but keep the section heading ``NeurIPS Paper Checklist"},
% %     \item  {\bf Keep the checklist subsection headings, questions/answers and guidelines below.}
% %     \item {\bf Do not modify the questions and only use the provided macros for your answers}.
% % \end{itemize} 
 

% % %%% END INSTRUCTIONS %%%


% \begin{enumerate}

% \item {\bf Claims}
%     \item[] Question: Do the main claims made in the abstract and introduction accurately reflect the paper's contributions and scope?
%     \item[] Answer: \answerYes{} % Replace by \answerYes{}, \answerNo{}, or \answerNA{}.
%     \item[] Justification: The abstract and introduction clearly state the core contributions of the paper: the Darwin G\"odel Machine, an open-ended, self-improving, self-referential system validated via empirical evaluation. They summarize the key experimental results - improving from 20.0\% to 50.0\% on SWE-bench and comparable gains on Polyglot - and explicitly note the scope limitation of only leveraging frozen pretrained LLMs (leaving full FM training to future work). These claims align precisely with the experimental findings presented, without overstating generalization beyond the evaluated benchmarks.
%     \item[] Guidelines:
%     \begin{itemize}
%         \item The answer NA means that the abstract and introduction do not include the claims made in the paper.
%         \item The abstract and/or introduction should clearly state the claims made, including the contributions made in the paper and important assumptions and limitations. A No or NA answer to this question will not be perceived well by the reviewers. 
%         \item The claims made should match theoretical and experimental results, and reflect how much the results can be expected to generalize to other settings. 
%         \item It is fine to include aspirational goals as motivation as long as it is clear that these goals are not attained by the paper. 
%     \end{itemize}

% \item {\bf Limitations}
%     \item[] Question: Does the paper discuss the limitations of the work performed by the authors?
%     \item[] Answer: \answerYes{} % Replace by \answerYes{}, \answerNo{}, or \answerNA{}.
%     \item[] Justification: The paper includes a dedicated "Conclusion and Limitations" section where it explicitly acknowledges the main limitations of the proposed framework. It discusses the high computational cost and two-week runtime on SWE-bench, the reliance on frozen pretrained FMs (limiting reasoning and requiring future work to self-train models), and the focus on coding benchmarks as a proxy for self-improvement. It further notes that the DGM currently lags behind closed-source SoTA, outlines safety considerations, and highlights assumptions about benchmark validity and domain scope, all of which demonstrate transparent reflection on its limitations.
%     \item[] Guidelines:
%     \begin{itemize}
%         \item The answer NA means that the paper has no limitation while the answer No means that the paper has limitations, but those are not discussed in the paper. 
%         \item The authors are encouraged to create a separate "Limitations" section in their paper.
%         \item The paper should point out any strong assumptions and how robust the results are to violations of these assumptions (e.g., independence assumptions, noiseless settings, model well-specification, asymptotic approximations only holding locally). The authors should reflect on how these assumptions might be violated in practice and what the implications would be.
%         \item The authors should reflect on the scope of the claims made, e.g., if the approach was only tested on a few datasets or with a few runs. In general, empirical results often depend on implicit assumptions, which should be articulated.
%         \item The authors should reflect on the factors that influence the performance of the approach. For example, a facial recognition algorithm may perform poorly when image resolution is low or images are taken in low lighting. Or a speech-to-text system might not be used reliably to provide closed captions for online lectures because it fails to handle technical jargon.
%         \item The authors should discuss the computational efficiency of the proposed algorithms and how they scale with dataset size.
%         \item If applicable, the authors should discuss possible limitations of their approach to address problems of privacy and fairness.
%         \item While the authors might fear that complete honesty about limitations might be used by reviewers as grounds for rejection, a worse outcome might be that reviewers discover limitations that aren't acknowledged in the paper. The authors should use their best judgment and recognize that individual actions in favor of transparency play an important role in developing norms that preserve the integrity of the community. Reviewers will be specifically instructed to not penalize honesty concerning limitations.
%     \end{itemize}

% \item {\bf Theory assumptions and proofs}
%     \item[] Question: For each theoretical result, does the paper provide the full set of assumptions and a complete (and correct) proof?
%     \item[] Answer: \answerNA{} % Replace by \answerYes{}, \answerNo{}, or \answerNA{}.
%     \item[] Justification: The paper does not include theoretical results.
%     \item[] Guidelines:
%     \begin{itemize}
%         \item The answer NA means that the paper does not include theoretical results. 
%         \item All the theorems, formulas, and proofs in the paper should be numbered and cross-referenced.
%         \item All assumptions should be clearly stated or referenced in the statement of any theorems.
%         \item The proofs can either appear in the main paper or the supplemental material, but if they appear in the supplemental material, the authors are encouraged to provide a short proof sketch to provide intuition. 
%         \item Inversely, any informal proof provided in the core of the paper should be complemented by formal proofs provided in appendix or supplemental material.
%         \item Theorems and Lemmas that the proof relies upon should be properly referenced. 
%     \end{itemize}

%     \item {\bf Experimental result reproducibility}
%     \item[] Question: Does the paper fully disclose all the information needed to reproduce the main experimental results of the paper to the extent that it affects the main claims and/or conclusions of the paper (regardless of whether the code and data are provided or not)?
%     \item[] Answer: \answerYes{} % Replace by \answerYes{}, \answerNo{}, or \answerNA{}.
%     \item[] Justification: The paper provides all the necessary information to reproduce its experiments. It specifies the exact benchmarks used (SWE-bench and Polyglot with version and task details), describes the DGM algorithm in sufficient detail (self-modification loop, archive management, evaluation protocols), and enumerates key hyperparameters (number of self-modification iterations, selection criteria, compute budget). It also reports environment specifics (API costs, two-week runtime on SWE-bench, FM versions) and points to detailed pseudocode and cost estimates in the appendix. Together, these elements enable another researcher to reconstruct the system and verify the reported performance gains.
%     \item[] Guidelines:
%     \begin{itemize}
%         \item The answer NA means that the paper does not include experiments.
%         \item If the paper includes experiments, a No answer to this question will not be perceived well by the reviewers: Making the paper reproducible is important, regardless of whether the code and data are provided or not.
%         \item If the contribution is a dataset and/or model, the authors should describe the steps taken to make their results reproducible or verifiable. 
%         \item Depending on the contribution, reproducibility can be accomplished in various ways. For example, if the contribution is a novel architecture, describing the architecture fully might suffice, or if the contribution is a specific model and empirical evaluation, it may be necessary to either make it possible for others to replicate the model with the same dataset, or provide access to the model. In general. releasing code and data is often one good way to accomplish this, but reproducibility can also be provided via detailed instructions for how to replicate the results, access to a hosted model (e.g., in the case of a large language model), releasing of a model checkpoint, or other means that are appropriate to the research performed.
%         \item While NeurIPS does not require releasing code, the conference does require all submissions to provide some reasonable avenue for reproducibility, which may depend on the nature of the contribution. For example
%         \begin{enumerate}
%             \item If the contribution is primarily a new algorithm, the paper should make it clear how to reproduce that algorithm.
%             \item If the contribution is primarily a new model architecture, the paper should describe the architecture clearly and fully.
%             \item If the contribution is a new model (e.g., a large language model), then there should either be a way to access this model for reproducing the results or a way to reproduce the model (e.g., with an open-source dataset or instructions for how to construct the dataset).
%             \item We recognize that reproducibility may be tricky in some cases, in which case authors are welcome to describe the particular way they provide for reproducibility. In the case of closed-source models, it may be that access to the model is limited in some way (e.g., to registered users), but it should be possible for other researchers to have some path to reproducing or verifying the results.
%         \end{enumerate}
%     \end{itemize}


% \item {\bf Open access to data and code}
%     \item[] Question: Does the paper provide open access to the data and code, with sufficient instructions to faithfully reproduce the main experimental results, as described in supplemental material?
%     \item[] Answer: \answerYes{} % Replace by \answerYes{}, \answerNo{}, or \answerNA{}.
%     \item[] Justification: We will open‐source the full DGM codebase and all generated data under a permissive license. The repository will include: (1) a detailed README with exact commands to set up the environment, install dependencies, and configure API keys; (2) scripts to download and pre-process SWE‐bench and Polyglot; (3) training and evaluation scripts for the DGM; and (4) the complete archive of generated agents and their evaluation logs. This ensures that anyone can faithfully reproduce our main experimental results.
%     \item[] Guidelines:
%     \begin{itemize}
%         \item The answer NA means that paper does not include experiments requiring code.
%         \item Please see the NeurIPS code and data submission guidelines (\url{https://nips.cc/public/guides/CodeSubmissionPolicy}) for more details.
%         \item While we encourage the release of code and data, we understand that this might not be possible, so “No” is an acceptable answer. Papers cannot be rejected simply for not including code, unless this is central to the contribution (e.g., for a new open-source benchmark).
%         \item The instructions should contain the exact command and environment needed to run to reproduce the results. See the NeurIPS code and data submission guidelines (\url{https://nips.cc/public/guides/CodeSubmissionPolicy}) for more details.
%         \item The authors should provide instructions on data access and preparation, including how to access the raw data, preprocessed data, intermediate data, and generated data, etc.
%         \item The authors should provide scripts to reproduce all experimental results for the new proposed method and baselines. If only a subset of experiments are reproducible, they should state which ones are omitted from the script and why.
%         \item At submission time, to preserve anonymity, the authors should release anonymized versions (if applicable).
%         \item Providing as much information as possible in supplemental material (appended to the paper) is recommended, but including URLs to data and code is permitted.
%     \end{itemize}


% \item {\bf Experimental setting/details}
%     \item[] Question: Does the paper specify all the training and test details (e.g., data splits, hyperparameters, how they were chosen, type of optimizer, etc.) necessary to understand the results?
%     \item[] Answer: \answerYes{} % Replace by \answerYes{}, \answerNo{}, or \answerNA{}.
%     \item[] Justification: \Cref{sec:experiments} and appendices fully describe the setup: the SWE-bench and Polyglot data splits and pre-processing steps; the self-modification loop parameters (number of iterations, archive sample size); the exact FM versions and prompts; and the evaluation metrics. This level of detail suffices to understand and reproduce the reported results.
%     \item[] Guidelines:
%     \begin{itemize}
%         \item The answer NA means that the paper does not include experiments.
%         \item The experimental setting should be presented in the core of the paper to a level of detail that is necessary to appreciate the results and make sense of them.
%         \item The full details can be provided either with the code, in appendix, or as supplemental material.
%     \end{itemize}

% \item {\bf Experiment statistical significance}
%     \item[] Question: Does the paper report error bars suitably and correctly defined or other appropriate information about the statistical significance of the experiments?
%     \item[] Answer: \answerNo{} % Replace by \answerYes{}, \answerNo{}, or \answerNA{}.
%     \item[] Justification: Running the full DGM evaluation on SWE-bench requires roughly two weeks of continuous compute and incurs around USD \$22,000 in API costs, making multiple independent trials for error‐bar estimation extremely costly. We therefore report single‐run results and acknowledge the absence of statistical significance measures as a limitation.
%     \item[] Guidelines:
%     \begin{itemize}
%         \item The answer NA means that the paper does not include experiments.
%         \item The authors should answer "Yes" if the results are accompanied by error bars, confidence intervals, or statistical significance tests, at least for the experiments that support the main claims of the paper.
%         \item The factors of variability that the error bars are capturing should be clearly stated (for example, train/test split, initialization, random drawing of some parameter, or overall run with given experimental conditions).
%         \item The method for calculating the error bars should be explained (closed form formula, call to a library function, bootstrap, etc.)
%         \item The assumptions made should be given (e.g., Normally distributed errors).
%         \item It should be clear whether the error bar is the standard deviation or the standard error of the mean.
%         \item It is OK to report 1-sigma error bars, but one should state it. The authors should preferably report a 2-sigma error bar than state that they have a 96\% CI, if the hypothesis of Normality of errors is not verified.
%         \item For asymmetric distributions, the authors should be careful not to show in tables or figures symmetric error bars that would yield results that are out of range (e.g. negative error rates).
%         \item If error bars are reported in tables or plots, The authors should explain in the text how they were calculated and reference the corresponding figures or tables in the text.
%     \end{itemize}

% \item {\bf Experiments compute resources}
%     \item[] Question: For each experiment, does the paper provide sufficient information on the computer resources (type of compute workers, memory, time of execution) needed to reproduce the experiments?
%     \item[] Answer: \answerYes{} % Replace by \answerYes{}, \answerNo{}, or \answerNA{}.
%     \item[] Justification: The paper reports that the SWE‐bench experiments run for approximately two weeks, and incurring about USD \$22,000 in API costs (\Cref{app:cost-estimate}). It also specifies the Polyglot evaluation setup and includes a detailed cost‐estimate table (\Cref{app:cost-estimate}). This level of detail enables readers to understand and provision the necessary compute resources to reproduce our results.
%     \item[] Guidelines:
%     \begin{itemize}
%         \item The answer NA means that the paper does not include experiments.
%         \item The paper should indicate the type of compute workers CPU or GPU, internal cluster, or cloud provider, including relevant memory and storage.
%         \item The paper should provide the amount of compute required for each of the individual experimental runs as well as estimate the total compute. 
%         \item The paper should disclose whether the full research project required more compute than the experiments reported in the paper (e.g., preliminary or failed experiments that didn't make it into the paper). 
%     \end{itemize}
    
% \item {\bf Code of ethics}
%     \item[] Question: Does the research conducted in the paper conform, in every respect, with the NeurIPS Code of Ethics \url{https://neurips.cc/public/EthicsGuidelines}?
%     \item[] Answer: \answerYes{} % Replace by \answerYes{}, \answerNo{}, or \answerNA{}.
%     \item[] Justification: Our work does not involve any human subjects or personal data, uses only publicly available benchmarks, and we explicitly discuss safety considerations in \Cref{sec:safety}. We adhere to responsible AI principles by providing open discussion of limitations, compute costs, and safety risks, and by planning to release code under a permissive license. Thus, our research fully conforms to the NeurIPS Code of Ethics.
%     \item[] Guidelines:
%     \begin{itemize}
%         \item The answer NA means that the authors have not reviewed the NeurIPS Code of Ethics.
%         \item If the authors answer No, they should explain the special circumstances that require a deviation from the Code of Ethics.
%         \item The authors should make sure to preserve anonymity (e.g., if there is a special consideration due to laws or regulations in their jurisdiction).
%     \end{itemize}


% \item {\bf Broader impacts}
%     \item[] Question: Does the paper discuss both potential positive societal impacts and negative societal impacts of the work performed?
%     \item[] Answer: \answerYes{} % Replace by \answerYes{}, \answerNo{}, or \answerNA{}.
%     \item[] Justification: The paper outlines positive societal impacts (e.g., accelerating AI-driven software development, lowering the barrier to high-quality code generation, and enabling broader access to AI innovation) in both the introduction (\Cref{sec:intro}) and conclusion (\Cref{sec:conclusion}). It also discusses negative impacts in the safety discussion (\Cref{sec:safety}) and the conclusion (\Cref{sec:conclusion}), including risks of unchecked self-improving AIs, potential misuse for malicious code generation or automation of harmful tasks. Moreover, we propose mitigation strategies (e.g., careful safety auditing, gated deployment, and human oversight) to address these risks.
%     \item[] Guidelines:
%     \begin{itemize}
%         \item The answer NA means that there is no societal impact of the work performed.
%         \item If the authors answer NA or No, they should explain why their work has no societal impact or why the paper does not address societal impact.
%         \item Examples of negative societal impacts include potential malicious or unintended uses (e.g., disinformation, generating fake profiles, surveillance), fairness considerations (e.g., deployment of technologies that could make decisions that unfairly impact specific groups), privacy considerations, and security considerations.
%         \item The conference expects that many papers will be foundational research and not tied to particular applications, let alone deployments. However, if there is a direct path to any negative applications, the authors should point it out. For example, it is legitimate to point out that an improvement in the quality of generative models could be used to generate deepfakes for disinformation. On the other hand, it is not needed to point out that a generic algorithm for optimizing neural networks could enable people to train models that generate Deepfakes faster.
%         \item The authors should consider possible harms that could arise when the technology is being used as intended and functioning correctly, harms that could arise when the technology is being used as intended but gives incorrect results, and harms following from (intentional or unintentional) misuse of the technology.
%         \item If there are negative societal impacts, the authors could also discuss possible mitigation strategies (e.g., gated release of models, providing defenses in addition to attacks, mechanisms for monitoring misuse, mechanisms to monitor how a system learns from feedback over time, improving the efficiency and accessibility of ML).
%     \end{itemize}
    
% \item {\bf Safeguards}
%     \item[] Question: Does the paper describe safeguards that have been put in place for responsible release of data or models that have a high risk for misuse (e.g., pretrained language models, image generators, or scraped datasets)?
%     \item[] Answer: \answerYes{} % Replace by \answerYes{}, \answerNo{}, or \answerNA{}.
%     \item[] Justification: In the safety discussion (\Cref{sec:safety}) and reiterated in the conclusion, the paper outlines concrete safeguards for responsible release of the DGM system. These measures address the dual-use risks inherent in powerful self-improving models.
%     \item[] Guidelines:
%     \begin{itemize}
%         \item The answer NA means that the paper poses no such risks.
%         \item Released models that have a high risk for misuse or dual-use should be released with necessary safeguards to allow for controlled use of the model, for example by requiring that users adhere to usage guidelines or restrictions to access the model or implementing safety filters. 
%         \item Datasets that have been scraped from the Internet could pose safety risks. The authors should describe how they avoided releasing unsafe images.
%         \item We recognize that providing effective safeguards is challenging, and many papers do not require this, but we encourage authors to take this into account and make a best faith effort.
%     \end{itemize}

% \item {\bf Licenses for existing assets}
%     \item[] Question: Are the creators or original owners of assets (e.g., code, data, models), used in the paper, properly credited and are the license and terms of use explicitly mentioned and properly respected?
%     \item[] Answer: \answerYes{} % Replace by \answerYes{}, \answerNo{}, or \answerNA{}.
%     \item[] Justification: We properly credit and cite all external assets: SWE-bench~\citep{jimenez2024swebench} (MIT License), Polyglot~\citep{gauthier2024polyglot} (Apache 2.0), and the pretrained FMs accessed via the OpenAI or Anthropic API under their published terms of service.
%     \item[] Guidelines:
%     \begin{itemize}
%         \item The answer NA means that the paper does not use existing assets.
%         \item The authors should cite the original paper that produced the code package or dataset.
%         \item The authors should state which version of the asset is used and, if possible, include a URL.
%         \item The name of the license (e.g., CC-BY 4.0) should be included for each asset.
%         \item For scraped data from a particular source (e.g., website), the copyright and terms of service of that source should be provided.
%         \item If assets are released, the license, copyright information, and terms of use in the package should be provided. For popular datasets, \url{paperswithcode.com/datasets} has curated licenses for some datasets. Their licensing guide can help determine the license of a dataset.
%         \item For existing datasets that are re-packaged, both the original license and the license of the derived asset (if it has changed) should be provided.
%         \item If this information is not available online, the authors are encouraged to reach out to the asset's creators.
%     \end{itemize}

% \item {\bf New assets}
%     \item[] Question: Are new assets introduced in the paper well documented and is the documentation provided alongside the assets?
%     \item[] Answer: \answerYes{} % Replace by \answerYes{}, \answerNo{}, or \answerNA{}.
%     \item[] Justification: All new assets introduced (i.e., the DGM codebase, the archive of generated agent variants, evaluation logs, and benchmark outputs) will be accompanied by comprehensive documentation. The release includes a structured README detailing directory organization, instructions for loading and interpreting evaluation logs, and example scripts to reproduce key figures. This ensures users can understand and utilize the new assets effectively.
%     \item[] Guidelines:
%     \begin{itemize}
%         \item The answer NA means that the paper does not release new assets.
%         \item Researchers should communicate the details of the dataset/code/model as part of their submissions via structured templates. This includes details about training, license, limitations, etc. 
%         \item The paper should discuss whether and how consent was obtained from people whose asset is used.
%         \item At submission time, remember to anonymize your assets (if applicable). You can either create an anonymized URL or include an anonymized zip file.
%     \end{itemize}

% \item {\bf Crowdsourcing and research with human subjects}
%     \item[] Question: For crowdsourcing experiments and research with human subjects, does the paper include the full text of instructions given to participants and screenshots, if applicable, as well as details about compensation (if any)? 
%     \item[] Answer: \answerNA{} % Replace by \answerYes{}, \answerNo{}, or \answerNA{}.
%     \item[] Justification: The paper does not involve crowdsourcing nor research with human subjects.
%     \item[] Guidelines:
%     \begin{itemize}
%         \item The answer NA means that the paper does not involve crowdsourcing nor research with human subjects.
%         \item Including this information in the supplemental material is fine, but if the main contribution of the paper involves human subjects, then as much detail as possible should be included in the main paper. 
%         \item According to the NeurIPS Code of Ethics, workers involved in data collection, curation, or other labor should be paid at least the minimum wage in the country of the data collector. 
%     \end{itemize}

% \item {\bf Institutional review board (IRB) approvals or equivalent for research with human subjects}
%     \item[] Question: Does the paper describe potential risks incurred by study participants, whether such risks were disclosed to the subjects, and whether Institutional Review Board (IRB) approvals (or an equivalent approval/review based on the requirements of your country or institution) were obtained?
%     \item[] Answer: \answerNA{} % Replace by \answerYes{}, \answerNo{}, or \answerNA{}.
%     \item[] Justification: The paper does not involve crowdsourcing nor research with human subjects.
%     \item[] Guidelines:
%     \begin{itemize}
%         \item The answer NA means that the paper does not involve crowdsourcing nor research with human subjects.
%         \item Depending on the country in which research is conducted, IRB approval (or equivalent) may be required for any human subjects research. If you obtained IRB approval, you should clearly state this in the paper. 
%         \item We recognize that the procedures for this may vary significantly between institutions and locations, and we expect authors to adhere to the NeurIPS Code of Ethics and the guidelines for their institution. 
%         \item For initial submissions, do not include any information that would break anonymity (if applicable), such as the institution conducting the review.
%     \end{itemize}

% \item {\bf Declaration of LLM usage}
%     \item[] Question: Does the paper describe the usage of LLMs if it is an important, original, or non-standard component of the core methods in this research? Note that if the LLM is used only for writing, editing, or formatting purposes and does not impact the core methodology, scientific rigorousness, or originality of the research, declaration is not required.
%     %this research? 
%     \item[] Answer: \answerYes{} % Replace by \answerYes{}, \answerNo{}, or \answerNA{}.
%     \item[] Justification: We explicitly detail the use of frozen pretrained LLMs as the core component of DGM's self-modification and evaluation loops. We specify model versions, prompt templates, and how these LLMs drive both code editing and evaluation on benchmarks (\Cref{sec:methods}), thereby clearly declaring their integral role in our methodology.
%     \item[] Guidelines:
%     \begin{itemize}
%         \item The answer NA means that the core method development in this research does not involve LLMs as any important, original, or non-standard components.
%         \item Please refer to our LLM policy (\url{https://neurips.cc/Conferences/2025/LLM}) for what should or should not be described.
%     \end{itemize}

% \end{enumerate}


\end{document}